\documentclass[a4paper,12pt,openany,notitlepage]{book}
\usepackage[margin=2.5cm]{geometry}
\usepackage{setspace}
\usepackage[utf8]{inputenc}
\usepackage{amsmath}
\usepackage{mathtools}
%\usepackage{amsfonts}
\usepackage{amssymb} % Pour les symboles mathématiques p109 de 'Latex par la pratique'
\usepackage[shortlabels]{enumitem}
\usepackage{float}




%*************************************************
%Pour la numérotation des exercices
\newcounter{num_exercice}
\newcommand{\exercice}
    {   \par
        \stepcounter{num_exercice}%
        \noindent
        \textbf{Exercise \arabic{num_exercice}}:
        \quad
    }
%*************************************************

\makeatletter
\newcommand*{\toccontents}{\@starttoc{toc}}
\makeatother

\begin{document}

\begin{titlepage}
\thispagestyle{empty}
\large Andrew Lehmann, \today \\
{\center
\rule{\linewidth}{0.5mm} \\[0.4cm]
{ \huge \bfseries Introduction to}\\[0.4cm] % Title of your document
{ \huge \bfseries Numerical Analysis:} \\[0.4cm] % Title of your document
{ \huge \bfseries Exercise Booklet} % Title of your document
\rule{\linewidth}{0.5mm} \\[0.2cm]
}
\toccontents
\thispagestyle{empty}
\clearpage
\end{titlepage}



\pagenumbering{arabic} 

%%%%%%%%%%%%%%%%%%%%%%%%%%%%%%%%%%%%%%%%%%%%%%%%%%%%%
%%%%%%%%%%%%%%%%%%%%%%%%%%%%%%%%%%%%%%%%%%%%%%%%%%%%%
%%%%%%%%%%%%%%%%%%%%%%%%%%%%%%%%%%%%%%%%%%%%%%%%%%%%%
\chapter{Introduction: square root methods}

\exercice{Calculation of $\sqrt{7}$}
%\begin{enumerate}
\begin{enumerate}[label=\alph*)]
	\item Define the function $f(x)$ so that the solution of $f(x)=0$ is $x=\sqrt{7}$. Is there a root of this equation in the interval $[1,2]$? What about $[2,3]$?
	
	\item Use 3 iterations of the bisection method, starting with the interval $[2,3]$, to estimate the value of $x=\sqrt{7}$.
	
	\item How many iterations of the bisection method are required to achieve an accuracy better than $10^{-5}$?
	
	\item Use 3 iterations of Heron’s method, starting at $x_0=3$, to estimate the value of $x=\sqrt{7}$.
\end{enumerate}


\exercice{Calculation of $\sqrt{5}$}
\begin{enumerate}[label=\alph*)]
	\item Give the terms of the Taylor expansion of $(1+x)^p$ for $x=4$ and $p=1/2$ up to the third derivative. Will this series converge on $\sqrt{5}$?
	
	\item Use 3 iterations of Heron’s method, starting at $x_0=2$, to estimate the value of $\sqrt{5}$.
\end{enumerate}


\exercice{Algorithm analysis}
\begin{enumerate}[label=\alph*)]
	\item For Theon of Smyrna’s method, by considering $p^2_{k+1} - 2q^2_{k+1}$ and $p_0=q_0=1$, prove this method converges to $x_0=2$.
	\item Prove that Heron’s method converges for any square root.
\end{enumerate}
%%%%%%%%%%%%%%%%%%%%%%%%%%%%%%%%%%%%%%%%%%%%%%%%%%%%%
%%%%%%%%%%%%%%%%%%%%%%%%%%%%%%%%%%%%%%%%%%%%%%%%%%%%%
%%%%%%%%%%%%%%%%%%%%%%%%%%%%%%%%%%%%%%%%%%%%%%%%%%%%%





%%%%%%%%%%%%%%%%%%%%%%%%%%%%%%%%%%%%%%%%%%%%%%%%%%%%%
%%%%%%%%%%%%%%%%%%%%%%%%%%%%%%%%%%%%%%%%%%%%%%%%%%%%%
%%%%%%%%%%%%%%%%%%%%%%%%%%%%%%%%%%%%%%%%%%%%%%%%%%%%%
\chapter{Rootfinding}

\exercice{Finding the intersection of $2\sin x=x$}
\begin{enumerate}[label=\alph*)]
	\item Sketch a graph of $y=2\sin x$ and $y=x$ in the domain $[-2\pi,2\pi]$. 
	
	\item Write a function $f(x)$ that has roots corresponding to the solutions of $2\sin x=x$.
	
	\item Use the Intermediate Value Theorem and the graph to find \textit{the number} of roots of $f(x)$, and give intervals surrounding them.
	
	\item Use the bisection algorithm to solve $2\sin x=x$ for as many iterations as needed       until the solution stops changing its first 4 decimal places. Do this for each       solution of $2\sin x=x$ that you found in part c.
	
	\item Define a function $g(x)$ so that we have an iterative scheme: $x_{k+1} = g(x_k)$ with fixed points at the roots of $f(x)$.
	
	\item Make a rough iteration figure exploring initial guesses $x_0$ and their subsequent      iterates using the iteration scheme of part e. Are there any unstable fixed-points? What value does $x_k$ converge to for initial $x_0$ where $\sin(x_0)>0$? What about initial $x_0$ where $\sin(x_0)<0$?
\end{enumerate}



\exercice{Finding the intersection of $\sin x=\cos x$}
\begin{enumerate}[label=\alph*)]
	\item Sketch a graph of $y=\sin x$ and $y=\cos x$ in the domain $[0,2\pi]$. 
	
	\item Write a function $f(x)$ that has roots corresponding to the solutions of $\sin x=\cos x$.
	
	\item Use the Intermediate Value Theorem and the graph to find \textit{the number} of roots of $f(x)$, and give intervals surrounding them.
	
	\item Use the bisection algorithm to solve $\sin x=x$ for as many iterations as needed       until the solution stops changing its first 4 decimal places. Do this for each       solution of $\sin x=\cos x$ that you found in part c.
\end{enumerate}



\exercice{Fixed-point analysis $g(x)=x(x^2-1)$}
\begin{enumerate}[label=\alph*)]
	\item How many fixed points, $\xi_k$, of $g(x)$ are there?
	
	\item Sketch a graph of $y=g(x)$ and $y=x$. Make rough iteration figures at different initial guesses $x_0$ on each side of the fixed points. Which fixed point(s) do you expect to be stable?
	
	\item Use the stability criterion on $|g(\xi_k)|$ to make a stability analysis of the fixed points.
\end{enumerate}



\exercice{Fixed-point analysis $f(x)=2e^{-x} + x - 2$}
\begin{enumerate}[label=\alph*)]
	\item Find 2 functions $g(x)$ for which the $x=g(x)$ has the same solutions as zeros of $f(x)$.
	
	\item Sketch separate graphs of the previous functions, as well as $y=2e^x$ and $y=2-x$ to get an intuition on the location of the roots or fixed points.
	
	\item Can you use Brouwer's theorem to guarantee the existence of any fixed points for either of the functions defined in part a?
	
	\item Make a stability analysis of the fixed points for both functions.
	
	\item For any stable fixed points, use the iterative scheme $x_{k+1}=g(x_k)$ for 5 iterations.
\end{enumerate}



\exercice{The intersection of $e^x=\cos x + 1$}
\begin{enumerate}[label=\alph*)]
	\item Sketch a graph of $y=e^x$ and $y=\cos x + 1$ in the domain $[-4\pi,4\pi]$. 
	
	\item Write a function $f(x)$ that has roots corresponding to the solutions of $e^x=\cos x + 1$.
	
	\item From the graph, how many positive roots and how many negative roots of $f(x)$ will there be?
	
	\item Use the Intermediate Value Theorem and the graph to prove there is a root of $f(x)$, for $x>0$.
	
	\item Use Netwon's method to propose an iterative scheme to find the roots of $f(x)$. Use this iterative scheme starting with $x_0=0$ to make a table of estimates $x_k$ for $k=0,1,\dots ,4$.
	
	\item Show that the function $g(x)=\log(\cos x + 1)$ gives an iterative scheme $x_{k+1} = g(x_k)$ with fixed points equal to the roots of $f(x)$. Determine a second function $h(x)$ which also satisfies these properties.
	
	\item Sketch a graph of $y=g(x)$ and $y=x$ in the domain $[-\pi,\pi]$. 
	
	\item Make a rough iteration figure exploring initial guesses x0 and their subsequent iterates using the iteration scheme with $g(x)$ of part f. Which fixed point(s) do you expect to be stable?
	
	\item Use the stability criterion on $|g(\xi_k)|$ to make a stability analysis of the fixed points $\xi_k \in [-\pi,\pi]$.
\end{enumerate}



\exercice{Fixed-point analysis $\log(x+2)=x^2$}
\begin{enumerate}[label=\alph*)]
	\item Sketch a graph of $y=\log(x+2)$ and $y=x^2$. 
	
	\item Write a function $f(x)$ that has roots corresponding to the solutions of $\log(x+2)=x^2$.
	
	\item From the graph, how many roots of $f(x)$ will there be?
	
	\item Use the Intermediate Value Theorem and the graph to locate the roots of $f(x)$.
	
	\item Use the Secant method to propose an iterative scheme to find the roots of $f(x)$. Use this iterative scheme twice, starting with $x_0=1$ and $x_1=2$ and then again with $x'_0=-1$ and $x_1=-1.5$ to make a table of estimates of $x_k$ and $x'_k$ until $k=4$.
	
	\item Show that the functions $g_1(x)=\exp(x^2)-2$ and $g_2(x)=( \log(x+2) )1/2$ give iterative schemes $x_{k+1} = g_i(x_k)$ with fixed points equal to the roots of $f(x)$.
	
	\item Sketch a graph of $y=g_1(x)$ and $y=x$. Sketch another graph of $y=g_2(x)$ and $y=x$. Can you use Brouwer's theorem to guarantee the existance of any fixed points for either $g_1(x)$ or $g_2(x)$?
	
	\item Make a rough iteration figure exploring initial guesses $x_0$ and their subsequent iterates using the iteration scheme with $g(x)$ of part f. Which fixed point(s) do you expect to be stable?
	
	\item Use the stability criterion on $|g_1(\xi_k)|$ and $|g_2(\xi_k)|$ to make a stability analysis of the fixed points $\xi_k$. For a stable fixed point of your choice, use the iterative scheme $x_{k+1}=g_i(x_k)$ to estimate the location of an intersection of $y=\log(x+2) $and $y=x_2$.
\end{enumerate}
%%%%%%%%%%%%%%%%%%%%%%%%%%%%%%%%%%%%%%%%%%%%%%%%%%%%%
%%%%%%%%%%%%%%%%%%%%%%%%%%%%%%%%%%%%%%%%%%%%%%%%%%%%%
%%%%%%%%%%%%%%%%%%%%%%%%%%%%%%%%%%%%%%%%%%%%%%%%%%%%%



%%%%%%%%%%%%%%%%%%%%%%%%%%%%%%%%%%%%%%%%%%%%%%%%%%%%%
%%%%%%%%%%%%%%%%%%%%%%%%%%%%%%%%%%%%%%%%%%%%%%%%%%%%%
%%%%%%%%%%%%%%%%%%%%%%%%%%%%%%%%%%%%%%%%%%%%%%%%%%%%%
\chapter{Polynomial interpolation}

\exercice{Given data $\{(1,2), (2,5), (3,3), (4,2)\}$}
\begin{enumerate}[label=\alph*)]
	\item Make a piece-wise linear interpolation of the given data.
	
	\item Find the unique polynomial of order less than 4 passing through each data point using the Lagrangian interpolation method.
	
	\item Find the unique polynomial of order less than 4 passing through each data point using the Newtonian interpolation method.
\end{enumerate}


\exercice{Given data $\{(1,2), (1.5,1.5), (3,2.5), (6,3)\}$}
\begin{enumerate}[label=\alph*)]
	\item Make a piece-wise linear interpolation of the given data.
	
	\item Find the unique polynomial of order less than 4 passing through each data point using the Lagrangian interpolation method.
	
	\item Find the unique polynomial of order less than 4 passing through each data point using the Newtonian interpolation method.
\end{enumerate}


\exercice{Estimate $f(x) = \log(x+2)$}
\begin{enumerate}[label=\alph*)]
	\item Make a piece-wise linear interpolation of $f(x)$ at positions $x \in [0,2,4,6]$.
	
	\item Give the Lagrangian polynomial of order less than 4 passing through $f(x)$ at positions $x \in [0,2,4,6]$.
	
	\item Use Newtonian interpolation to give the polynomial of order less than 4 passing through $f(x)$ at positions $x \in [0,1,5,6]$.
	
	\item Find the area under $f(x)$ between 0 and 6 by using the polynomials found in parts b and c. Compare to the analytic solution by directly integrating $f(x)$.
	
	\item Using your table of divided differences from part c, add 1 data point at $x=3$ to increase the polynomial order by 1.
\end{enumerate}


\exercice{Estimate $f(x) = e^x$}
\begin{enumerate}[label=\alph*)]
	\item Make a piece-wise linear interpolation of $f(x)$ at positions $x \in [-1,1,3,5]$.
	
	\item Give the Lagrangian polynomial of order less than 4 passing through $f(x)$ at positions $x \in [-1,0,4,5]$.
	
	\item Use Newtonian interpolation to give the polynomial of order less than 4 passing through $f(x)$ at positions $x \in [-1,1,3,5]$.
	
	\item Find the area under $f(x)$ between 0 and 6 by using the polynomials found in parts b and c. Compare to the analytic solution by directly integrating $f(x)$.
	
	\item Using your table of divided differences from part c, add 1 data point at $x=2$ to increase the polynomial order by 1.
\end{enumerate}
%%%%%%%%%%%%%%%%%%%%%%%%%%%%%%%%%%%%%%%%%%%%%%%%%%%%%
%%%%%%%%%%%%%%%%%%%%%%%%%%%%%%%%%%%%%%%%%%%%%%%%%%%%%
%%%%%%%%%%%%%%%%%%%%%%%%%%%%%%%%%%%%%%%%%%%%%%%%%%%%%



%%%%%%%%%%%%%%%%%%%%%%%%%%%%%%%%%%%%%%%%%%%%%%%%%%%%%
%%%%%%%%%%%%%%%%%%%%%%%%%%%%%%%%%%%%%%%%%%%%%%%%%%%%%
%%%%%%%%%%%%%%%%%%%%%%%%%%%%%%%%%%%%%%%%%%%%%%%%%%%%%
\chapter{Least-squares}

\exercice{Given data $\{ (1,5), (3,6), (5,8), (7,9) \}$}

Make a least-squares linear fit to the given data (generated by $y=0.7x + 4$). 


\exercice{Given data $\{ (1,7), (3,10), (4,12), (5,14),(7,18) \}$}

Make a least-squares linear fit to the given data (generated by $y=2.2x + 3$). 


\exercice{Given data $\{ (1,5), (2,11), (4,33), (5,59) \}$}

Make a least-squares exponential fit to the given data (generated by $y=3e^{0.6x}$). 

\exercice{Given data $\{ (1,1), (3,2), (5,5), (7,9), (8,12) \}$}

Make a least-squares exponential fit to the given data (generated by $y=e^{0.3x}$). 

\exercice{Given data $\{ (1,1), (3,3), (5,8), (7,15) \}$}

Make a least-squares quadratic fit to the given data (generated by $y=3x^2$). 

\exercice{Given data $\{ (1,10), (2,40), (3,89), (4,159), (6,359) \}$}

Make a least-squares quadratic fit to the given data (generated by y=$10x^2$). 
%%%%%%%%%%%%%%%%%%%%%%%%%%%%%%%%%%%%%%%%%%%%%%%%%%%%%
%%%%%%%%%%%%%%%%%%%%%%%%%%%%%%%%%%%%%%%%%%%%%%%%%%%%%
%%%%%%%%%%%%%%%%%%%%%%%%%%%%%%%%%%%%%%%%%%%%%%%%%%%%%




%%%%%%%%%%%%%%%%%%%%%%%%%%%%%%%%%%%%%%%%%%%%%%%%%%%%%
%%%%%%%%%%%%%%%%%%%%%%%%%%%%%%%%%%%%%%%%%%%%%%%%%%%%%
%%%%%%%%%%%%%%%%%%%%%%%%%%%%%%%%%%%%%%%%%%%%%%%%%%%%%
\chapter{Integration}

\exercice{Given the function $f(x) = 2 \log(x+1)$}
\begin{enumerate}[label=\alph*)]
	\item Use the Trapezium rule to estimate the integral of $f(x)$ from [0,6] using 2 subdivisions.
	
	\item Use the Trapezium rule to estimate the integral of $f(x)$ from [0,6] using 3 subdivisions.
	
	\item Use Simpson's rule to estimate the integral of $f(x)$ from [0,6] using 2 subdivisions.
	
	\item Use Simpson's rule to estimate the integral of $f(x)$ from [0,6] using 3 subdivisions.
	
	\item Integrate $f(x)$ analytically to compare to the previous results.
\end{enumerate}


\exercice{Given the function $f(x) = 3 e^{0.5x} - 3$}
\begin{enumerate}[label=\alph*)]
	\item Use the Trapezium rule to estimate the integral of $f(x)$ from [0,6] using 2 subdivisions.
	
	\item Use the Trapezium rule to estimate the integral of $f(x)$ from [0,6] using 3 subdivisions.
	
	\item Use Simpson's rule to estimate the integral of $f(x)$ from [0,6] using 2 subdivisions.
	
	\item Use Simpson's rule to estimate the integral of $f(x)$ from [0,6] using 3 subdivisions.
	
	\item Integrate $f(x)$ analytically to compare to the previous results.
\end{enumerate}


\exercice{Given the function $f(x) = e^{\cos x}$}
\begin{enumerate}[label=\alph*)]
	\item Use the Trapezium rule to estimate the integral of $f(x)$ from $[0,2\pi]$ using 2 subdivisions.
	
	\item Use the Trapezium rule to estimate the integral of $f(x)$ from $[0,2\pi]$ using 3 subdivisions.
	
	\item Use Simpson's rule to estimate the integral of $f(x)$ from $[0,2\pi]$ using 2 subdivisions.
	
	\item Use Simpson's rule to estimate the integral of $f(x)$ from $[0,2\pi]$ using 3 subdivisions.
	
	\textit{Note: $f(x)$ cannot be integrated analytically (into elementary functions) to compare to the previous results.}
\end{enumerate}
%%%%%%%%%%%%%%%%%%%%%%%%%%%%%%%%%%%%%%%%%%%%%%%%%%%%%
%%%%%%%%%%%%%%%%%%%%%%%%%%%%%%%%%%%%%%%%%%%%%%%%%%%%%
%%%%%%%%%%%%%%%%%%%%%%%%%%%%%%%%%%%%%%%%%%%%%%%%%%%%%



%%%%%%%%%%%%%%%%%%%%%%%%%%%%%%%%%%%%%%%%%%%%%%%%%%%%%
%%%%%%%%%%%%%%%%%%%%%%%%%%%%%%%%%%%%%%%%%%%%%%%%%%%%%
%%%%%%%%%%%%%%%%%%%%%%%%%%%%%%%%%%%%%%%%%%%%%%%%%%%%%
\chapter{Differentiation}
\exercice{Given the data}
\begin{figure}[h]
\begin{tabular}{l|lllll}
$x$    & 2.0000   & 2.0025   & 2.0050   & 2.0075   & 2.0100 \\  \hline
$f(x)$ & 1.202604 & 1.214698 & 1.226929 & 1.239299 & 1.251809
\end{tabular}
\end{figure}

\begin{enumerate}[label=\alph*)]
	\item Use the forward finite difference scheme with h=0.005 to estimate $f'(2.005)$.
	
	\item Use the forward finite difference scheme with h=0.0025 to estimate $f'(2.005)$.

	\item Use the backward finite difference scheme with h=0.005 to estimate $f'(2.005)$.

	\item Use the backward finite difference scheme with h=0.0025 to estimate $f'(2.005)$.
	
	\item Use the centered finite difference scheme with h=0.005 to estimate $f'(2.005)$.

	\item Use the centered finite difference scheme with h=0.0025 to estimate $f'(2.005)$.

	\item The data were generated with $f(x)=\exp(x^2 + 10)/1000000$. What is the true derivative at $x=2.005$?
\end{enumerate}


\exercice{Given the data}
\begin{figure}[h]
\begin{tabular}{l|lllll}
$x$    & 7.5000   & 7.5025   & 7.5050   & 7.5075   & 7.5100 \\  \hline
$f(x)$ & 384.375 & 384.785 & 385.194 & 385.604 & 386.015
\end{tabular}
\end{figure}          

\begin{enumerate}[label=\alph*)]
	\item Use the forward finite difference scheme with h=0.005 to estimate $f'(7.505)$.
	
	\item Use the forward finite difference scheme with h=0.0025 to estimate $f'(7.505)$.

	\item Use the backward finite difference scheme with h=0.005 to estimate $f'(7.505)$.

	\item Use the backward finite difference scheme with h=0.0025 to estimate $f'(7.505)$.
	
	\item Use the centered finite difference scheme with h=0.005 to estimate $f'(7.505)$.

	\item Use the centered finite difference scheme with h=0.0025 to estimate $f'(7.505)$.

	\item The data were generated with $f(x)=x^3 - 5x$. What is the true derivative at $x=7.505$?
\end{enumerate}


\exercice{Given the data}
\begin{figure}[h]
\begin{tabular}{l|lllll}
$x$    & 1.560   & 1.561   & 1.562   & 1.563   & 1.564 \\  \hline
$f(x)$ & 92.6204 & 102.076 & 113.681 & 128.263 & 147.136
\end{tabular}
\end{figure}

\begin{enumerate}[label=\alph*)]
	\item Use the forward finite difference scheme with h=0.002 to estimate $f'(1.562)$.
	
	\item Use the forward finite difference scheme with h=0.001 to estimate $f'(1.562)$.

	\item Use the backward finite difference scheme with h=0.002 to estimate $f'(1.562)$.

	\item Use the backward finite difference scheme with h=0.001 to estimate $f'(1.562)$.
	
	\item Use the centered finite difference scheme with h=0.002 to estimate $f'(1.562)$.

	\item Use the centered finite difference scheme with h=0.001 to estimate $f'(1.562)$.

	\item The data were generated with $f(x)=\tan x$. What is the true derivative at $x=7.505$?
\end{enumerate}
%%%%%%%%%%%%%%%%%%%%%%%%%%%%%%%%%%%%%%%%%%%%%%%%%%%%%
%%%%%%%%%%%%%%%%%%%%%%%%%%%%%%%%%%%%%%%%%%%%%%%%%%%%%
%%%%%%%%%%%%%%%%%%%%%%%%%%%%%%%%%%%%%%%%%%%%%%%%%%%%%




%%%%%%%%%%%%%%%%%%%%%%%%%%%%%%%%%%%%%%%%%%%%%%%%%%%%%
%%%%%%%%%%%%%%%%%%%%%%%%%%%%%%%%%%%%%%%%%%%%%%%%%%%%%
%%%%%%%%%%%%%%%%%%%%%%%%%%%%%%%%%%%%%%%%%%%%%%%%%%%%%
\chapter{Differential equations}

\exercice{Consider $y'(x) = -2y$}
\begin{enumerate}[label=\alph*)]
	\item Write the iteration scheme given by Euler's method.
	
	\item Write the iteration scheme given by the implicit Trapezium method. Define a function $F(z)$ so that the root of the equation $F(z)=0$ is equal to the next iteration of the solution in your iteration scheme.
	
	\item From the Trapezium method, propose an explicit iteration scheme to solve the differential equation.
	
	\item Write the iteration scheme given by the second order Runge-Kutta method. 
	
	\item Write the iteration scheme given by the fourth order Runge-Kutta method.
	
	\item What is the analytic solution of the differential equation?
\end{enumerate}



\exercice{Consider $y'(x) = xy$}
\begin{enumerate}[label=\alph*)]
	\item Write the iteration scheme given by Euler's method.
	
	\item Write the iteration scheme given by the implicit Trapezium method. Define a function $F(z)$ so that the root of the equation $F(z)=0$ is equal to the next iteration of the solution in your iteration scheme.
	
	\item From the Trapezium method, propose an explicit iteration scheme to solve the differential equation.
	
	\item Write the iteration scheme given by the second order Runge-Kutta method. 
	
	\item Write the iteration scheme given by the fourth order Runge-Kutta method.
	
	\item What is the analytic solution of the differential equation?
\end{enumerate}



\exercice{Consider $y'(x) = -2y^2$}

\textit{Keep 4 decimal places throughout your calculations in the following questions.}
\begin{enumerate}[label=\alph*)]
	\item Estimate $y(0.1)$ using Euler's method, with $y(0)=2.0$ and $h=0.1$.
	
	\item Estimate $y(0.1)$ using Euler's method, with $y(0)=2.0$ and $h=0.05$.
	
	\item Estimate $y(0.1)$ using the implicit Trapezium method, with $y(0)=2.0$ and $h=0.1$. Use Newton’s method to find the root to 4 decimal places.
	
	\item Estimate $y(0.1)$ using the implicit Trapezium method, with $y(0)=2.0$ and $h=0.05$. Use Newton’s method to find the root to 4 decimal places.
	
	\item From the Trapezium method, propose an explicit method for answering the previous 2 quesitons.
	
	\item Estimate $y(0.1)$ using the second order Runge-Kutta method, with $y(0)=2.0$ and $h=0.1$.
	
	\item Estimate $y(0.1)$ using the second order Runge-Kutta method, with $y(0)=2.0$ and $h=0.05$.
	
	\item Estimate $y(0.1)$ using the fourth order Runge-Kutta method, with $y(0)=2.0$ and $h=0.1$.
	
	\item Estimate $y(0.1)$ using the fourth order Runge-Kutta method, with $y(0)=2.0$ and $h=0.05$.
	
	\item Compare your answers to the analytic solution.
\end{enumerate}



\exercice{Consider $y'(x) = y - 1$}

\textit{Keep 4 decimal places throughout your calculations in the following questions.}
\begin{enumerate}[label=\alph*)]
	\item Estimate $y(0.1)$ using Euler's method, with $y(0)=1.1$ and $h=0.05$.
	
	\item Estimate $y(0.1)$ using Euler's method, with $y(0)=0.9$ and $h=0.05$.
	
	\item Estimate $y(0.1)$ using the implicit Trapezium method, with $y(0)=1.1$ and $h=0.05$.
	
	\item Estimate $y(0.1)$ using the implicit Trapezium method, with $y(0)=0.9$ and $h=0.05$.
	
	\item From the Trapezium method, propose an explicit method for answering the previous 2 quesitons.
	
	\item Estimate $y(0.1)$ using the second order Runge-Kutta method, with $y(0)=1.1$ and $h=0.05$.
	
	\item Estimate $y(0.1)$ using the fourth order Runge-Kutta method, with $y(0)=1.1$ and $h=0.05$.
	
	\item Estimate $y(0.1)$ using the second order Runge-Kutta method, with $y(0)=0.9$ and $h=0.05$.
	
	\item Estimate $y(0.1)$ using the fourth order Runge-Kutta method, with $y(0)=0.9$ and $h=0.05$.
	
	\item Compare your answers to the analytic solution.
\end{enumerate}
%%%%%%%%%%%%%%%%%%%%%%%%%%%%%%%%%%%%%%%%%%%%%%%%%%%%%
%%%%%%%%%%%%%%%%%%%%%%%%%%%%%%%%%%%%%%%%%%%%%%%%%%%%%
%%%%%%%%%%%%%%%%%%%%%%%%%%%%%%%%%%%%%%%%%%%%%%%%%%%%%



%%%%%%%%%%%%%%%%%%%%%%%%%%%%%%%%%%%%%%%%%%%%%%%%%%%%%
%%%%%%%%%%%%%%%%%%%%%%%%%%%%%%%%%%%%%%%%%%%%%%%%%%%%%
%%%%%%%%%%%%%%%%%%%%%%%%%%%%%%%%%%%%%%%%%%%%%%%%%%%%%
\chapter{Matrix methods}

\exercice{$2\times 2$ matrices}
\begin{enumerate}[label=\alph*)]
	\item For the following matrices, find $L$ and $U$ such that $LA_i=U$
	\begin{align*}
	A_1 = 
	\begin{pmatrix}
	1 & 2 \\
	1 & 1
	\end{pmatrix}, \,\,\,
	A_2 = 
	\begin{pmatrix}
	2 & 2 \\
	3 & -1
	\end{pmatrix}, \,\,\,
	A_3 = 
	\begin{pmatrix}
	-2 &  1 \\
	 1 & -1
	\end{pmatrix}
	\end{align*}
	
	\item For the previous matrices $A_i$ solve for $x_1$ and $x_2$, given
	\begin{align*}
	A_1
	\begin{pmatrix}
	x_1 \\
	x_2
	\end{pmatrix}
	=
	\begin{pmatrix}
	1 \\
	2
	\end{pmatrix}
	\end{align*}
	
	\item For the previous matrices $A_i$ solve for $x_1$ and $x_2$, given
	\begin{align*}
	A_1
	\begin{pmatrix}
	x_1 \\
	x_2
	\end{pmatrix}
	=
	\begin{pmatrix}
	-1 \\
	2
	\end{pmatrix}
	\end{align*}
\end{enumerate}




\exercice{$3\times 3$ matrices}
\begin{enumerate}[label=\alph*)]
	\item For the following matrices, find $L$ and $U$ such that $LA_i=U$
	\begin{align*}
	A_1 = 
	\begin{pmatrix}
	1 & 2 & 1 \\
	1 & 1 & 1 \\
	2 & 3 & 1
	\end{pmatrix}, \,\,\,
	A_2 = 
	\begin{pmatrix}
	1 & 2 & 1 \\
	-1 & 1 & 1 \\
	2 & 3 & -1
	\end{pmatrix}, \,\,\,
	A_3 = 
	\begin{pmatrix}
	 1 &  2 & -4 \\
	-2 &  2 &  1 \\
	-5 & -3 &  3
	\end{pmatrix}
	\end{align*}
	
	\item For the previous matrices $A_i$ solve for $x_1$ and $x_2$, given
	\begin{align*}
	A_1
	\begin{pmatrix}
	x_1 \\
	x_2 \\
	x_3
	\end{pmatrix}
	=
	\begin{pmatrix}
	1 \\
	2 \\
	3
	\end{pmatrix}
	\end{align*}
	
	\item For the previous matrices $A_i$ solve for $x_1$ and $x_2$, given
	\begin{align*}
	A_1
	\begin{pmatrix}
	x_1 \\
	x_2 \\
	x_3
	\end{pmatrix}
	=
	\begin{pmatrix}
	-1 \\
	2 \\
	-1
	\end{pmatrix}
	\end{align*}
\end{enumerate}



\exercice{Convergence}
\begin{enumerate}[label=\alph*)]
	\item For the following system, give a range of values that $k$ can take to guarantee the Jacobi iteration scheme converges on the true solution.
	\begin{align*}
	\begin{pmatrix}
	k & -1 \\
	1 & 2
	\end{pmatrix}
	\begin{pmatrix}
	x \\
	y
	\end{pmatrix}
	=
	\begin{pmatrix}
	3 \\
	-5
	\end{pmatrix}
	\end{align*}
	
	\item For the following system, give ranges of values that $\alpha$, $\beta$, and $\gamma$ can take to guarantee the Jacobi iteration scheme converges on the true solution?
	\begin{align*}
	\begin{pmatrix}
	3 & -1 & \alpha \\
	\beta & -3 & 2 \\
	1 & 1 & \gamma \\
	\end{pmatrix}
	\begin{pmatrix}
	x \\
	y \\
	z
	\end{pmatrix}
	=
	\begin{pmatrix}
	1 \\
	2 \\
	-2
	\end{pmatrix}
	\end{align*}
\end{enumerate}





\exercice{Schemes}

For the following matrices, write the Jacobi iterative system of equations.

\begin{align*}
	A_1 = 
	\begin{pmatrix}
	2 & 1 \\
	1 & 3
	\end{pmatrix}, \,\,\,
	A_2 = 
	\begin{pmatrix}
	3 & -1 \\
	1 & -2
	\end{pmatrix}, \,\,\,
	A_3 = 
	\begin{pmatrix}
	3 & -1 &  1 \\
	0 & -3 &  2 \\
	1 &  1 & -4
	\end{pmatrix}, \,\,\,
	A_4 = 
	\begin{pmatrix}
	 4 & -1 &  1 \\
	-1 &  5 & -2 \\
	 1 & -1 & -4
	\end{pmatrix}
\end{align*}




\exercice{Jacobi iteration}

For the following systems, use Jacobi iteration 3 times to find $X^{(3)}$.
\begin{enumerate}[label=\alph*)]
	\item \begin{align*}
	\begin{pmatrix}
	2 & 1 \\
	1 & 3
	\end{pmatrix}
	\begin{pmatrix}
	x \\
	y
	\end{pmatrix}
	=
	\begin{pmatrix}
	3 \\
	-5
	\end{pmatrix}, \,\,\,\,
	{\rm with} \,\,
	X^{(0)}=
	\begin{pmatrix}
	0 \\
	0
	\end{pmatrix}	
	\end{align*}
	
	\item \begin{align*}
	\begin{pmatrix}
	2 & 1 \\
	1 & 3
	\end{pmatrix}
	\begin{pmatrix}
	x \\
	y
	\end{pmatrix}
	=
	\begin{pmatrix}
	2 \\
	1
	\end{pmatrix}, \,\,\,\,
	{\rm with} \,\,
	X^{(0)}=
	\begin{pmatrix}
	1 \\
	1
	\end{pmatrix}
	\end{align*}
	
	
	\item \begin{align*}
	\begin{pmatrix}
	3 & -1 &  1 \\
	0 & -3 &  2 \\
	1 &  1 & -4
	\end{pmatrix}
	\begin{pmatrix}
	x \\
	y \\
	z
	\end{pmatrix}
	=
	\begin{pmatrix}
	1 \\
	2 \\
	-2
	\end{pmatrix}, \,\,\,\,
	{\rm with} \,\,
	X^{(0)}=
	\begin{pmatrix}
	0 \\
	0 \\
	0
	\end{pmatrix}
	\end{align*}
	
	
	\item \begin{align*}
	\begin{pmatrix}
	3 & -1 &  1 \\
	0 & -3 &  2 \\
	1 &  1 & -4
	\end{pmatrix}
	\begin{pmatrix}
	x \\
	y \\
	z
	\end{pmatrix}
	=
	\begin{pmatrix}
	-1 \\
	3 \\
	0
	\end{pmatrix}, \,\,\,\,
	{\rm with} \,\,
	X^{(0)}=
	\begin{pmatrix}
	1 \\
	2 \\
	3
	\end{pmatrix}
	\end{align*}
\end{enumerate}


\end{document}
