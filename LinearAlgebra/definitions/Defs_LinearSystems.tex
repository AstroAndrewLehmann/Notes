\chapter{Linear Systems} \label{ch:linearsystems}


\definition{Linear system of equations}{
A system of $m$ linear equations with $n$ unknowns, denoted $(S)$, has the general form
\begin{align*}
\begin{cases}
a_{11} x_1  + a_{12} x_2 + \cdots + a_{1n} x_n = y_1 \\
a_{21} x_1  + a_{22} x_2 + \cdots + a_{2n} x_n = y_2 \\
\vdots \\
a_{m1} x_1  + a_{m2} x_2 + \cdots + a_{mn} x_n = y_m
\end{cases}
\qquad (S)
\end{align*}
where the $x_j$ are the unknowns we want to find, $a_{ij}$ are the \textit{coefficients} and  the $y_i$ are the \textit{constant terms}.
}

\definition{Homogeneous linear system}{
For any system of linear equations, $(S)$, given by $AX=Y$, we associate the \textbf{homogeneous system}, denoted $(H)$:
\begin{align*}
 AX = 0_m
\end{align*} 
for the column
\begin{align*}
0_m = \begin{pmatrix} 0 \\ 0 \\ \vdots \\ 0 \end{pmatrix} \in \mathcal{M}_{m,1}
\end{align*}
We will denote the solution set of $(H)$ by $\mathcal{H}$.

\noindent \textit{Note}: the homogeneous system always admits \textit{at least one} solution, the trivial solution $X=0_n$.
}



\definition{Equivalent systems}{
Two systems of linear equations are \textbf{equivalent} if they share the same set of solutions.
}

\definition{Elementary operations}{

There are \textbf{elementary operations} that we can do to systems of equations that give new systems that remain equivalent to the old.

\centering

\underline{Exchanging two equations} \vspace{-0.4cm}
\begin{align*}
(S_1)
\begin{cases}
    I_1 +   I_2 -   I_3 =  0 \\
 13 I_1 - 6 I_2         = 20 
\end{cases}
\quad\equiv\quad
(S_2)
\begin{cases}
 13 I_1 - 6 I_2         = 20 \\
    I_1 +   I_2 -   I_3 =  0 
\end{cases}
\end{align*}
\underline{Multiplying one equation by a non-zero constant} \vspace{-0.4cm}
\begin{align*}
(S_1)
\begin{cases}
    I_1 +   I_2 -   I_3 =  0 \\
 13 I_1 - 6 I_2         = 20 
\end{cases}
\quad\equiv\quad
(S_2)
\begin{cases}
    I_1 +   I_2 -   I_3 =  0 \\
 I_1 - (6/13) I_2         = (20/13) 
\end{cases}
\end{align*}
\underline{Adding a multiple of one equation to another equation} \vspace{-0.4cm}
\begin{align*}
(S_1)
\begin{cases}
    I_1 +   I_2 -   I_3 =  0 \\
 13 I_1 - 6 I_2         = 20 
\end{cases}
\quad\equiv\quad
(S_2)
\begin{cases}
    I_1 +   I_2 -   I_3 =  0 \\
 15 I_1 - 4 I_2 - 2 I_3 = 20 
\end{cases}
\end{align*}
}


\definition{Overdetermined system}{
An overdetermined system has more equations than unknowns. We say “there are too many equations”. Such a system allows solutions only if certain conditions are met. 
}


\definition{Underdetermined system}{

An \textbf{underdetermined system} has less equations than unknowns. We say ``there are not enough equations''. Such a system has either no solutions, or infinitely many. 
}


\definition{Cramer system}{
Suppose we have the following linear system of equations (with unknowns equal to equations)
\begin{align*}
\begin{cases}
a_{11} x_1  + a_{12} x_2 + \cdots + a_{1n} x_n = y_1 \\
a_{21} x_1  + a_{22} x_2 + \cdots + a_{2n} x_n = y_2 \\
\vdots \\
a_{n1} x_1  + a_{n2} x_2 + \cdots + a_{nn} x_n = y_n
\end{cases}
\qquad (S)
\end{align*}
with the associated matrix form
\begin{align*}
\underbrace{
\begin{pmatrix}
a_{11} & a_{12} & \cdots & a_{1n} \\
a_{21} & a_{22} & \cdots & a_{2n} \\
\vdots & \vdots & \ddots & \vdots \\
a_{n1} & a_{n2} & \cdots & a_{nn}
\end{pmatrix}}_A
%%
%%
\underbrace{
\begin{pmatrix}
x_{1} \\
x_{2} \\
\vdots \\
x_{n}
\end{pmatrix}}_X
%%
=
\underbrace{
\begin{pmatrix}
y_{1} \\
y_{2} \\
\vdots \\
y_{n}
\end{pmatrix}}_Y
\end{align*}
We say that $(S)$ is a Cramer system if $\det(A) \neq 0$.
}
