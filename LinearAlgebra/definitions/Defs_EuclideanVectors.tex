\chapter{Euclidean Vectors} \label{ch:euclid}


\definition{Euclidean vector or tuple}{
A Euclidean vector is a list of $n$ real numbers, also called an $n$-tuple. We write this list in parentheses, for example $(1,3,-2, \dots, 0)$, and we say that this object belongs to $\mathbb{R}^n$. An arbitrary tuple can be written $\mathbf{v}=(v_1,v_2,\cdots,v_n)$ where the \textit{components} $v_i \in \mathbb{R}$ for any index $i$.
}


\definition{Tuple addition}{
Euclidean vectors are added to each other component by component. In symbols
\begin{align*}
(a_1, a_2, \dots, a_n) + (b_1, b_2, \dots, b_n) = (a_1+b_1, a_2+b_2, \dots, a_n + b_n).
\end{align*}
\textit{Note}: this means you can only add two tuples together \textit{of the same size}. It makes no sense to add a 3-tuple to a 5-tuple.
}


\definition{Scalar multiplication}{
Let $c\in\mathbb{R}$, called a scalar quantity, and $\mathbf{v} \in \mathbb{R}^n$ with components $v_i$. Then the \textit{scalar multiplication} $c\mathbf{v}$ gives a vector $\mathbf{w}$ with components $w_i = c v_i$ for every index $i$. In tuple form
\begin{align*}
c(v_1,v_2,\dots,v_n) = (cv_1,cv_2,\dots,cv_n).
\end{align*}
}


\definition{Canonical Euclidean unit vectors}{
The canonical Euclidean vectors in $\mathbb{R}^n$ are the $n$ vectors of the form
\begin{gather*}
\mathbf{e}_1 = (1,0,\dots,0) \\
\mathbf{e}_2 = (0,1,\dots,0) \\
\vdots \\
\mathbf{e}_n = (0,0,\dots,1).
\end{gather*}
More compactly
\begin{align*}
\mathbf{e}_k = (\alpha_1, \alpha_2, \dots, \alpha_n) \quad \text{where} \quad
\alpha_j=
\begin{cases}
1 & \text{for } j= k, \\
0 & \text{for } j\neq k.
\end{cases}
\end{align*}
}


\definition{Dot product}{
For two $n$-tuples $\mathbf{a}$ and $\mathbf{b}$, their \textit{dot product}, also called \textit{scalar} product and \textit{Euclidean inner} product, is the real number given by the addition of component by component multiplication
\begin{align*}
\mathbf{a}\cdot\mathbf{b} = a_1 b_1 + a_2 b_2 + \cdots + a_n b_n = \sum_{i=1}^n a_i b_i.
\end{align*}
}

\definition{Euclidean Norm}{
The norm of an $n$-tuple $\mathbf{v}$, denoted $\lVert \mathbf{v} \rVert$, is given by
\begin{align*}
\lVert \mathbf{v} \rVert = \sqrt{\mathbf{v}\cdot\mathbf{v}} = \sqrt{v_1^2 + v_2^2 + \cdots + v_2^2}.
\end{align*}
}

\definition{Orthogonal Euclidean vectors}{
Two vectors in $\mathbb{R}^n$ are orthogonal if and only if their dot product equals zero.
}

\definition{Displacement vector}{
Given two Euclidean vectors $\mathbf{a}$ and $\mathbf{b}$, the displacement vector pointing from $\mathbf{a}$ to $\mathbf{b}$ is given by $\mathbf{r}=\mathbf{b}-\mathbf{a}$ as pictured below. Of course we can also create the displacement vector in the other direction, from $\mathbf{b}$ to $\mathbf{a}$, given by $\mathbf{a}-\mathbf{b}$.
}


\definition{Vector form of a straight line}{
The set of vectors in $\mathbb{R}^n$ of the form $\mathbf{v} = \mathbf{a} + t \mathbf{r}$ for a parameter $t\in\mathbb{R}$ represents a straight line through the space $\mathbb{R}^n$. That is,
\begin{align*}
\{(x,y) \, | \, \forall x\in\mathbb{R} \, \text{and} \, y=mx+b  \} = \{ \mathbf{a} + t \mathbf{r} \, | \, \forall t\in\mathbb{R} \}
\end{align*}
where $\mathbf{a}$ is an arbitrary pair $(x,mx+b)$ and $\mathbf{r}$ is a displacement vector between any two distinct pairs $(x_1,mx_1+b)$ and $(x_2,mx_2+b)$.
}

