\chapter{Euclidean Vectors} \label{ch:euclid}

\section{Basic definitions}

\definition{Euclidean vector or tuple}{
A Euclidean vector is a list of $n$ real numbers, also called an $n$-tuple. We write this list in parentheses, for example $(1,3,-2, \dots, 0)$, and we say that this object belongs to $\mathbb{R}^n$. An arbitrary tuple can be written $\mathbf{v}=(v_1,v_2,\cdots,v_n)$ where the \textit{components} $v_i \in \mathbb{R}$ for any index $i$.
}

\noindent It will be particularly useful to build intuition on vectors in $\mathbb{R}^2$. These are pairs of numbers like $(1,2)$, $(-1,\pi)$ etc. We represent these vectors on a cartesian plane by associating an arrow pointing from the origin, $\mathcal{O}=(0,0)$, to the given pair of numbers $(x,y)$ as below

\begin{figure}[H]
\centering
\begin{tikzpicture}[% styles used in image code
         > = Straight Barb, % defined in "arrows.meta
dot/.style = {circle, fill,
              minimum size=2mm, inner sep=0pt, outer sep=0pt,
              node contents={}},
box/.style = {draw, thin, minimum  width=2mm, minimum height=4mm,
              inner sep=0pt, outer sep=0pt,
              node contents={}, sloped},
my angle/.style args = {#1/#2}{draw,->,
                               angle radius=#1,
                               angle eccentricity=#2,
                               } % angle label position!
                        ]
	% coordinate axis
	\draw[->] (-1, 0)   -- (5,0) node[below left] {$x$};
	\draw[->] ( 0,-0.5) -- (0,4) node[below left] {$y$};

	\coordinate (O) at (0,0);
	\coordinate (v1) at (3,1);
	\coordinate (v2) at (1,2);
	
	\draw (O) node[below left] {$\mathcal{O}$};
	\draw [->,magenta,very thick] (O) --(v1) node[above right,black] {$(3,1)$};
	\draw [->,red,very thick] (O) --(v2) node[above right,black] {$(1,2)$};
\end{tikzpicture} \caption*{Arrow representation of two euclidean vectors.}
\end{figure}

\noindent The components of euclidean vectors represented this way are generally called the $x$ and $y$ components (there will be a $z$ component for triples), and so we often write $\mathbf{v}=(v_x, v_y)$. For obvious reasons we also call these 2 dimensional vectors.

Now we would like to create a way to add vectors together. When we add two vectors together the result should also be a vector. 

\definition{Tuple addition}{
Euclidean vectors are added to each other component by component. In symbols
\begin{align*}
(a_1, a_2, \dots, a_n) + (b_1, b_2, \dots, b_n) = (a_1+b_1, a_2+b_2, \dots, a_n + b_n).
\end{align*}
\textit{Note}: this means you can only add two tuples together \textit{of the same size}. It makes no sense to add a 3-tuple to a 5-tuple.
}

\noindent Let's understand this tuple addition geometrically. For the two vectors we represented above, $\mathbf{a}=(1,2)$, $\mathbf{b}=(3,1)$ and so their addition is the vector $\mathbf{c}=\mathbf{a}+\mathbf{b}=(1+3,2+1)=(4,3)$. This component by component addition is represented below left. We can also understand this addition as though we move the vector $\mathbf{b}$ so that its tail is at the tip of $\mathbf{a}$. This tip-to-tail method is shown below to the right. It equally works by moving $\mathbf{a}$ to the tip of $\mathbf{b}$ meaning that $\mathbf{a}+\mathbf{b} = \mathbf{b}+\mathbf{a}$.

\begin{figure}[H]
\centering
\begin{subfigure}[b]{0.45\textwidth}
\begin{tikzpicture}[% styles used in image code
         > = Straight Barb, % defined in "arrows.meta
dot/.style = {circle, fill,
              minimum size=2mm, inner sep=0pt, outer sep=0pt,
              node contents={}},
box/.style = {draw, thin, minimum  width=2mm, minimum height=4mm,
              inner sep=0pt, outer sep=0pt,
              node contents={}, sloped},
my angle/.style args = {#1/#2}{draw,->,
                               angle radius=#1,
                               angle eccentricity=#2,
                               } % angle label position!
                        ]
	% coordinate axis
	\draw[->] (-1, 0)   -- (5,0) node[below left] {$x$};
	\draw[->] ( 0,-0.5) -- (0,4) node[below left] {$y$};

	\coordinate (O) at (0,0);
	\coordinate (a) at (1,2);
	\coordinate (ax) at (1,0);
	\coordinate (ay) at (0,2);
	\coordinate (b) at (3,1);
	\coordinate (bx) at (3,0);
	\coordinate (by) at (0,1);
	\coordinate (c) at (4,3);
	\coordinate (cx) at (4,0);
	\coordinate (cy) at (0,3);
	
	\draw (O) node[below left] {$\mathcal{O}$};
	\draw [->,red,very thick] (O) -- (a) node[black, above] {$(1,2)$};
	\draw [->,blue,very thick]  (O) -- (b) node[black, above] {$(3,1)$};
	\draw [->,very thick] (O) -- (c) node[above right] {$(4,3)$};
	
	
	\draw [black, decorate, decoration={brace,amplitude=5pt,raise=2pt,aspect=0.5}] 
		($(cy)+(0.1,0)$)--($(c)-(0.2,0)$) node[pos=0.5,black,above=5pt]{$c_x$};
	\draw [black, decorate, decoration={brace,mirror,amplitude=5pt,raise=2pt,aspect=0.5}] 
		($(cx)+(0,0.1)$) -- ($(c)-(0,0.2)$) node[pos=0.5,black,right=5pt]{$c_y$};
	
	\draw [red, decorate, decoration={brace,mirror,raise=5pt,amplitude=5pt,aspect=0.55}] 
		(O)--(bx) node[pos=0.5,black,below=5pt]{$b_x$};
	
	\draw [red, decorate, decoration={brace,raise=5pt,amplitude=5pt,aspect=0.55}] 
		(O)--(by) node[pos=0.5,black,left=7pt]{$b_y$};
	
	\draw [cyan, decorate, decoration={brace,mirror,raise=5pt,amplitude=5pt,aspect=0.55}] 
		($(bx)+(0.1,0)$)--($(bx)+(ax)$) node[pos=0.5,black,below=7pt]{$a_x$};
	
	\draw [cyan, decorate, decoration={brace,raise=5pt,amplitude=5pt,aspect=0.55}] 
		($(by)+(0,0.1)$)--($(by)+(ay)$) node[pos=0.5,black,left=7pt]{$a_y$};
		
\end{tikzpicture} \caption*{Component by component addition.}
\end{subfigure}
~
\begin{subfigure}[b]{0.45\textwidth}
\begin{tikzpicture}[% styles used in image code
         > = Straight Barb, % defined in "arrows.meta
dot/.style = {circle, fill,
              minimum size=2mm, inner sep=0pt, outer sep=0pt,
              node contents={}},
box/.style = {draw, thin, minimum  width=2mm, minimum height=4mm,
              inner sep=0pt, outer sep=0pt,
              node contents={}, sloped},
my angle/.style args = {#1/#2}{draw,->,
                               angle radius=#1,
                               angle eccentricity=#2,
                               } % angle label position!
                        ]
	% coordinate axis
	\draw[->] (-1, 0)   -- (5,0) node[below left] {$x$};
	\draw[->] ( 0,-0.5) -- (0,4) node[below left] {$y$};

	\coordinate (O) at (0,0);
	\coordinate (a) at (1,2);
	\coordinate (b) at (3,1);
	\coordinate (c) at (4,3);
	
	\draw (O) node[below left] {$\mathcal{O}$};
	\draw [->,red,very thick] (O) --(a) node[black,pos=0.5,left] {$\mathbf{a}$};
	\draw [->,blue,very thick] ($(a)+0.05*(b)$) --($(a)+0.9*(b)$) node[black,pos=0.4,rotate=18.4,above] {$\mathbf{b}$};
	\draw [->,blue,very thick] (O) --(b) node[black,pos=0.8,below] {$\mathbf{b}$};
	\draw [->,red,very thick] ($(b)+0.05*(a)$) --($(b)+0.9*(a)$) node[black,pos=0.5,pos=0.5,right] {$\mathbf{a}$};
	\draw [->,very thick] (O) --(c) node[above right] {$\mathbf{c}$};
\end{tikzpicture} \caption*{Tip-to-tail representation.}
\end{subfigure}
\end{figure}

\noindent Now lets think about multiplcation. We can naturally think of doubling or tripling a vector, which should be adding a vector to itself 2 or 3 times, and geometrically it just becomes longer. So we want $2\mathbf{v} = \mathbf{v} + \mathbf{v}$. In components this gives $2(v_x,v_y)=(v_x,v_y) + (v_x,v_y) = (2v_x,2v_y)$. So we see that multiplying a tuple by a number multiplies each of its components by that number.

\definition{Scalar multiplication}{
Let $c\in\mathbb{R}$, called a scalar quantity, and $\mathbf{v} \in \mathbb{R}^n$ with components $v_i$. Then the \textit{scalar multiplication} $c\mathbf{v}$ gives a vector $\mathbf{w}$ with components $w_i = c v_i$ for every index $i$. In tuple form
\begin{align*}
c(v_1,v_2,\dots,v_n) = (cv_1,cv_2,\dots,cv_n).
\end{align*}
}

\begin{figure}[H]
\centering
\begin{tikzpicture}[% styles used in image code
         > = Straight Barb, % defined in "arrows.meta
dot/.style = {circle, fill,
              minimum size=2mm, inner sep=0pt, outer sep=0pt,
              node contents={}},
box/.style = {draw, thin, minimum  width=2mm, minimum height=4mm,
              inner sep=0pt, outer sep=0pt,
              node contents={}, sloped},
my angle/.style args = {#1/#2}{draw,->,
                               angle radius=#1,
                               angle eccentricity=#2,
                               } % angle label position!
                        ]
	% coordinate axis
	\draw[->] (-3, 0)   -- (5,0) node[below left] {$x$};
	\draw[->] ( 0,-1) -- (0,3) node[below left] {$y$};

	\coordinate (O) at (0,0);
	\coordinate (v1) at (2,1);
	\coordinate (v2) at (1,0.5);
	\coordinate (v3) at (4,2);
	\coordinate (v4) at (-2,-1);
	
	\draw (O) node[below right] {$\mathcal{O}$};
	\draw [->,blue] (O) --(v3) node[pos=0.7,above,black] {$2\mathbf{v}$};
	\draw [->,black] (O) --(v1) node[pos=0.78,above,black] {$\mathbf{v}$};
	\draw [->,magenta,very thick] (O) --(v2) node[pos=0.5,above=5pt,black] {$0.5\mathbf{v}$};
	\draw [->,red] (O) --(v4) node[pos=0.7,above,black] {$-1\mathbf{v}$};
\end{tikzpicture} \caption*{Arrow representation of scalar multiplication.}
\end{figure}

\noindent We can use scalar multiplication and some manipulation to form a second way to represent Euclidean vectors:
\begin{align*}
(v_1,v_2,\dots,v_n) &= (v_1,0,\dots,0) + (0,v_2,\dots,0) + \cdots + (0,0,\dots,v_n) \\
&= v_1(1,0,\dots,0) + v_2(0,1,\dots,0) + \cdots + v_n(0,0,\dots,1)\\
&= v_1\mathbf{e}_1 + v_2\mathbf{e}_2 + \cdots + v_n\mathbf{e}_n
\end{align*}
where we have introduced the canonical Euclidean vectors.

\definition{Canonical Euclidean unit vectors}{
The canonical Euclidean vectors in $\mathbb{R}^n$ are the $n$ vectors of the form
\begin{gather*}
\mathbf{e}_1 = (1,0,\dots,0) \\
\mathbf{e}_2 = (0,1,\dots,0) \\
\vdots \\
\mathbf{e}_n = (0,0,\dots,1).
\end{gather*}
More compactly
\begin{align*}
\mathbf{e}_k = (\alpha_1, \alpha_2, \dots, \alpha_n) \quad \text{where} \quad
\alpha_j=
\begin{cases}
1 & \text{for } j= k, \\
0 & \text{for } j\neq k.
\end{cases}
\end{align*}
}

\noindent In two dimensions we have several notations for these unit vectors
\begin{align*}
\mathbf{v} = v_x \hat{x} + v_y \hat{y} = v_x \mathbf{e}_x + v_y \mathbf{e}_y = v_x \hat{i} + v_y \hat{j}
\end{align*}
but we will stick to the hat notation, $\hat{x}$, in this text when referring to the geometric representations in 2 or 3d ($\hat{z}=\mathbf{e}_z = \hat{k}$ being used for the third direction). We can understand these unit vectors as the building blocks of the arrow:
\begin{figure}[H]
\centering
\begin{tikzpicture}[% styles used in image code
         > = Straight Barb, % defined in "arrows.meta
dot/.style = {circle, fill,
              minimum size=2mm, inner sep=0pt, outer sep=0pt,
              node contents={}},
box/.style = {draw, thin, minimum  width=2mm, minimum height=4mm,
              inner sep=0pt, outer sep=0pt,
              node contents={}, sloped},
my angle/.style args = {#1/#2}{draw,->,
                               angle radius=#1,
                               angle eccentricity=#2,
                               } % angle label position!
                        ]
	% coordinate axis
	\draw[->] (-3, 0)   -- (5,0) node[below] {$x$};
	\draw[->] ( 0,-1) -- (0,3.5) node[left] {$y$};

	\coordinate (O) at (0,0);
	\coordinate (v) at (2,3);
	\coordinate (xhat) at (1,0);
	\coordinate (yhat) at (0,1);
	
	\draw (O) node[below left] {$\mathcal{O}$};
	\draw [->] (O) --(v) node[pos=0.9,above left,rotate=56.3] {$\mathbf{v}=(2,3)$};
	\draw [->, blue] (O) --(xhat) node[pos=0.5,below=5pt] {$\hat{x}$};
	\draw [->, blue] (xhat) --($2*(xhat)$) node[pos=0.5,below=5pt] {$\hat{x}$};
	\draw [->, magenta] (O) --(yhat) node[pos=0.5,left=5pt] {$\hat{y}$};
	\draw [->, magenta] ($2*(xhat)$) --($2*(xhat)+(yhat)$) node[pos=0.5,right=5pt] {$\hat{y}$};
	\draw [->, magenta] ($2*(xhat)+(yhat)$) --($2*(xhat)+2*(yhat)$) node[pos=0.5,right=5pt] {$\hat{y}$};
	\draw [->, magenta] ($2*(xhat)+2*(yhat)$) --($2*(xhat)+3*(yhat)$) node[pos=0.5,right=5pt] {$\hat{y}$};
\end{tikzpicture} \caption*{Unit vectors as building blocks of a Euclidean vector.}
\end{figure}

\section{Dot product}

\noindent Notice that with scalar multiplication we multiplied two different kinds of objects together, a scalar by a vector resulting in a vector. How about multiplying two vectors together? You might be tempted to simply form a new vector with component by component multiplication, as we did for addition. This turns out not to be so useful. One very useful way to define vector multiplication is as follows.

\definition{Dot product}{
For two $n$-tuples $\mathbf{a}$ and $\mathbf{b}$, their \textit{dot product}, also called \textit{scalar} product and \textit{Euclidean inner} product, is the real number given by the addition of component by component multiplication
\begin{align*}
\mathbf{a}\cdot\mathbf{b} = a_1 b_1 + a_2 b_2 + \cdots + a_n b_n = \sum_{i=1}^n a_i b_i.
\end{align*}
}

\noindent Notice that the result of this product is a real number, not another tuple. Let's first try to understand the dot product by considering a 2d vector ``dotted'' with itself:
\begin{align*}
\mathbf{v}\cdot\mathbf{v}=(v_x,v_y)\cdot(v_x,v_y)=v_x^2 + v_y^2.
\end{align*}
This dot product gives the square of the length of the right triangle made from the $x$ and $y$ components of $\mathbf{v}$:

\begin{figure}[H]
\centering
\begin{tikzpicture}[% styles used in image code
         > = Straight Barb, % defined in "arrows.meta
dot/.style = {circle, fill,
              minimum size=2mm, inner sep=0pt, outer sep=0pt,
              node contents={}},
box/.style = {draw, thin, minimum  width=2mm, minimum height=4mm,
              inner sep=0pt, outer sep=0pt,
              node contents={}, sloped},
my angle/.style args = {#1/#2}{draw,->,
                               angle radius=#1,
                               angle eccentricity=#2,
                               } % angle label position!
                        ]
	% coordinate axis
	%\draw[->] (-3, 0)   -- (5,0) node[below left] {$x$};
	%\draw[->] ( 0,-1) -- (0,3) node[below left] {$y$};

	\coordinate (O) at (0,0);
	\coordinate (v1) at (6,4);
	\coordinate (vx) at (6,0);
	\coordinate (vy) at (0,4);
	
	\draw [->] (O) --(v1) node[pos=0.5,above=5pt] {$\mathbf{v}$};
	\draw [dashed] (O) --($(vx)-(0.1,0)$) node[pos=0.5,below] {$v_x$};
	\draw [dashed] (vx) --($(v1)-(0,0.15)$) node[pos=0.5,right] {$v_y$};
\end{tikzpicture} \caption*{Components of the vector $\mathbf{v}$}
\end{figure}

\noindent So we can identify the dot product $\mathbf{v}\cdot\mathbf{v}$ with the square of the length of $\mathbf{v}$. We generalise this concept of vector length in the operation of the norm.

\definition{Norm}{
The norm of an $n$-tuple $\mathbf{v}$, denoted $\lVert \mathbf{v} \rVert$, is given by
\begin{align*}
\lVert \mathbf{v} \rVert = \sqrt{\mathbf{v}\cdot\mathbf{v}} = \sqrt{v_1^2 + v_2^2 + \cdots + v_2^2}.
\end{align*}
}

\noindent So far we have defined a vector by its components. Through the norm, we can have a method of \textit{calculating} the components from 2 other pieces of information: its length and its direction. Its direction will mean its angle with respect to the $x$-axis. Denote that angle $\theta$ and the norm as the non-bold font version of the vector $v=\lVert \mathbf{v} \rVert$. From the above figure, we see that $\cos\theta = v_x/v$ and $\sin\theta = v_y/v$. So, given a vector length and its direction we have
\begin{align*}
\mathbf{v} = (v\cos\theta, v\sin\theta) = v(\cos\theta,\sin\theta).
\end{align*}
This gives us the following result: for any 2d Euclidean vector $\mathbf{v}$ its components are given by
\begin{align*}
v_x &= \lVert \mathbf{v} \rVert \cos\theta \\
v_y &= \lVert \mathbf{v} \rVert \sin\theta
\end{align*}
where $\theta$ is the angle that $\mathbf{v}$ makes with the $x$ axis.

\noindent What happens if we dot two different vectors with this representation. Let $\mathbf{a}=a(\cos\alpha,\sin\alpha)$ and $\mathbf{b}=b(\cos\beta,\sin\beta)$. Then
\begin{align*}
\mathbf{a}\cdot\mathbf{b} &= [a(\cos\alpha,\sin\alpha)]\cdot [b(\cos\beta,\sin\beta)] \\
&= ab(\cos\alpha,\sin\alpha)\cdot (\cos\beta,\sin\beta) \\
&= ab(\cos\alpha\cos\beta + \sin\alpha\sin\beta) \\
&= ab \cos(\alpha-\beta)
\end{align*}
where we have used a trigonometric identity in the last line and a little trickery in the first line that you should verify: $(c_1\mathbf{v}_1)\cdot(c_2\mathbf{v}_2)=(c_1c_2)\mathbf{v}_1\cdot\mathbf{v}_2$. In the end we have shown the following theorem.

\theorem{Dot product with angle}{
The Euclidean dot product of two $n$-tuples $\mathbf{a}$ and $\mathbf{b}$ can be calculated solely from their norms and the angle between them, $\theta$, given by the formula
\begin{align*}
\mathbf{a}\cdot\mathbf{b} = \lVert \mathbf{a} \rVert \lVert \mathbf{b} \rVert \cos \theta
\end{align*}
}
\begin{figure}[H]
\centering
\begin{tikzpicture}[% styles used in image code
         > = Straight Barb, % defined in "arrows.meta
dot/.style = {circle, fill,
              minimum size=2mm, inner sep=0pt, outer sep=0pt,
              node contents={}},
box/.style = {draw, thin, minimum  width=2mm, minimum height=4mm,
              inner sep=0pt, outer sep=0pt,
              node contents={}, sloped},
my angle/.style args = {#1/#2}{draw,->,
                               angle radius=#1,
                               angle eccentricity=#2,
                               } % angle label position!
                        ]
	% coordinate axis
	%\draw[->] (-3, 0)   -- (5,0) node[below left] {$x$};
	%\draw[->] ( 0,-1) -- (0,3) node[below left] {$y$};

	\coordinate (O) at (0,0);
	\coordinate (a) at (5,1);
	\coordinate (b) at (2,3);
	
	\draw [->] (O) --(a) node[pos=0.5,below right=5pt] {$\mathbf{a}$};
	\draw [->] (O) --(b) node[pos=0.5,above=5pt] {$\mathbf{b}$};
	
	\begin{scope}
	\path[clip] (O) -- (a) -- (b);
	\fill[red, opacity=0.3, draw=black] (O) circle (10mm);
	\node at ($(O)+(35:15mm)$) {$\theta$};
	\end{scope}
\end{tikzpicture}\caption*{The angle between two vectors.}
\end{figure}

\noindent From this expression of the dot product we see immediately what happens in the special case of perpendicular vectors. When two vectors are perpendicular the angle between them is $\pi/2$ and the dot product is then zero (recall that $\cos(\pi/2)=0$). We can extend this result to define orthogonality for higher dimensional Euclidean vectors.

\definition{Orthogonal Euclidean vectors}{
Two vectors in $\mathbb{R}^n$ are orthogonal if and only if their dot product equals zero.
}

\section{Applications}

So far we have been considering Euclidean vectors as arrows pointing from the origin to the given tuple. It can be useful to consider arrows pointing from an arbitrary point in space to another.

\definition{Displacement vector}{
Given two Euclidean vectors $\mathbf{a}$ and $\mathbf{b}$, the displacement vector pointing from $\mathbf{a}$ to $\mathbf{b}$ is given by $\mathbf{r}=\mathbf{b}-\mathbf{a}$ as pictured below. Of course we can also create the displacement vector in the other direction, from $\mathbf{b}$ to $\mathbf{a}$, given by $\mathbf{a}-\mathbf{b}$.
}

\begin{figure}[H]
\centering
\begin{tikzpicture}[% styles used in image code
         > = Straight Barb, % defined in "arrows.meta
dot/.style = {circle, fill,
              minimum size=2mm, inner sep=0pt, outer sep=0pt,
              node contents={}},
box/.style = {draw, thin, minimum  width=2mm, minimum height=4mm,
              inner sep=0pt, outer sep=0pt,
              node contents={}, sloped},
my angle/.style args = {#1/#2}{draw,->,
                               angle radius=#1,
                               angle eccentricity=#2,
                               } % angle label position!
                        ]
	% coordinate axis
	\draw[->] (-0.5, 0) -- (4,0) node[below left] {$x$};
	\draw[->] ( 0,-0.5) -- (0,3) node[below left] {$y$};

	\coordinate (O) at (0,0);
	\coordinate (a) at (1,2);
	\coordinate (b) at (3,1);
	\coordinate (ar) at ($0.95*(a)+0.05*(b)$);
	\coordinate (br) at ($0.95*(b)+0.05*(a)$);
	
	\draw [->] (O) --(a) node[pos=0.5,above=5pt] {$\mathbf{a}$};
	\draw [->] (O) --(b) node[pos=0.5,right=5pt] {$\mathbf{b}$};
	\draw [->] (ar) --(br) node[pos=0.5,above=3pt,rotate=-26.5] {$\mathbf{r}= \mathbf{b}-\mathbf{a}$};
\end{tikzpicture}
\end{figure}

Why is this useful? Well we can, for example, represent straight lines in a new way. Recall the equation for a straight line $y=mx+b$ where $m$ gives the slope of the line and $b$ gives the $y$-intercept (or vertical offset). Let's consider the line $y=(0.5)x+1$. Now that means we can choose any two $x$ values, say $x=-1$ and $x=2$, and compute corresponding $y$ values, in this case $y=0.5$ and $y=2$. So the Euclidean vectors $\mathbf{a}=(-1,0.5)$ and $\mathbf{b}=(2,2)$ belong to the line, as pictured.

\begin{figure}[H]
\centering
\begin{tikzpicture}[% styles used in image code
         > = Straight Barb, % defined in "arrows.meta
dot/.style = {circle, fill,
              minimum size=2mm, inner sep=0pt, outer sep=0pt,
              node contents={}},
box/.style = {draw, thin, minimum  width=2mm, minimum height=4mm,
              inner sep=0pt, outer sep=0pt,
              node contents={}, sloped},
my angle/.style args = {#1/#2}{draw,->,
                               angle radius=#1,
                               angle eccentricity=#2,
                               } % angle label position!
                        ]
	% coordinate axis
	\draw[->] (-1, 0) -- (4,0) node[below left] {$x$};
	\draw[->] ( 0,-0.5) -- (0,3) node[below left] {$y$};
	
	\draw[domain=-1.5:4.5,variable=\x] plot({\x},{0.5*\x+1});

	\coordinate (O) at (0,0);
	\coordinate (a) at (-1,0.5);
	\coordinate (b) at (2,2);
	
	\draw [->] (O) --($0.9*(a)$) node[pos=0.5,above] {$\mathbf{a}$};
	\draw [->] (O) --($0.9*(b)$) node[pos=0.5,right=5pt] {$\mathbf{b}$};
	
	\filldraw [blue] (a) circle (2pt) node[above left,black] (a) {$(-1,0.5)$};
	\filldraw [magenta] (b) circle (2pt) node[above,black] (b) {$(2,2)$};
\end{tikzpicture}
\end{figure}

\noindent The displacement vector from $\mathbf{a}$ to $\mathbf{b}$ is given by $\mathbf{r}=\mathbf{b}-\mathbf{a}=(2,2)-(-1,0.5)=(3,1.5)$. This displacement vector can be used to represent any point on the line. We just have to start at $\mathbf{a}$ and add any multiple of $\mathbf{r}$:


\begin{figure}[H]
\centering
\begin{tikzpicture}[% styles used in image code
         > = Straight Barb, % defined in "arrows.meta
dot/.style = {circle, fill,
              minimum size=2mm, inner sep=0pt, outer sep=0pt,
              node contents={}},
box/.style = {draw, thin, minimum  width=2mm, minimum height=4mm,
              inner sep=0pt, outer sep=0pt,
              node contents={}, sloped},
my angle/.style args = {#1/#2}{draw,->,
                               angle radius=#1,
                               angle eccentricity=#2,
                               } % angle label position!
                        ]
	% coordinate axis
	\draw[->] (-2, 0) -- (4,0) node[below left] {$x$};
	\draw[->] ( 0,-1) -- (0,3) node[below left] {$y$};
	
	\draw[dashed,domain=-3:4.5,variable=\x] plot({\x},{0.5*\x+1});

	\coordinate (O) at (0,0);
	\coordinate (a) at (-1,0.5);
	\coordinate (r) at (3,1.5);
	
	\draw [->] (O) --($0.9*(a)$) node[pos=0.5,above] {$\mathbf{a}$};
	\draw [->] (a) --($(a)+1.3*(r)$) node[pos=0.9,above,rotate=26.5] {$\mathbf{a}+1.3\mathbf{r}$};
	\draw [->,red,very thick] (a) --($(a)+(r)$) node[pos=0.6,above,rotate=26.5,black] {$\mathbf{a}+\mathbf{r}$};
	\draw [->] (a) --($(a)-0.5*(r)$) node[pos=0.7,above,rotate=26.5] {$\mathbf{a}-0.5\mathbf{r}$};
	
	\filldraw [blue] (a) circle (2pt);
\end{tikzpicture}
\end{figure}
\noindent So this gives as the vector representation of a line.

\definition{Vector form of a straight line}{
The set of vectors in $\mathbb{R}^n$ of the form $\mathbf{v} = \mathbf{a} + t \mathbf{r}$ for a parameter $t\in\mathbb{R}$ represents a straight line through the space $\mathbb{R}^n$. That is,
\begin{align*}
\{(x,y) \, | \, \forall x\in\mathbb{R} \, \text{and} \, y=mx+b  \} = \{ \mathbf{a} + t \mathbf{r} \, | \, \forall t\in\mathbb{R} \}
\end{align*}
where $\mathbf{a}$ is an arbitrary pair $(x,mx+b)$ and $\mathbf{r}$ is a displacement vector between any two distinct pairs $(x_1,mx_1+b)$ and $(x_2,mx_2+b)$.
}


\section*{To do}
Unit vector

unit vector pointing in the direction of a given vector

Dot product as projection into another direction

\section{Summary of properties of Euclidean vectors}\label{sec:ch1_summary}

Let's collect a list of key properties of operations with tuples that will turn out not to be unique to tuples. The following are true for any tuples $\mathbf{a}$, $\mathbf{b}$ and $\mathbf{c} \in\mathbb{R}^n$ and any real numbers $k$ and $l\in\mathbb{R}$.

\begin{align*}
& \text{addition of tuples gives another tuple} && \quad \mathbf{a} + \mathbf{b} \in \mathbb{R}^n &\\
%
& \text{tuple addition is associative} && \quad (\mathbf{a} + \mathbf{b}) + \mathbf{c}  =\mathbf{a} + (\mathbf{b} + \mathbf{c}) &\\
%
& \text{the zero tuple is a tuple of zeros} && \quad \mathbf{a} + \mathbf{0} = \mathbf{0} + \mathbf{a} = \mathbf{a} &\\
%
& \text{the negative of a tuple is the additive inverse} && \quad \mathbf{a} + (-\mathbf{a}) = \mathbf{0} &\\
%
& \text{the order of tuple addition doesn't matter} && \quad \mathbf{a} + \mathbf{b} = \mathbf{b} + \mathbf{a} &\\
%
& \text{scalar multiplication of a tuple gives another tuple} && \quad k\mathbf{a} \in \mathbb{R}^n &\\
%
& \text{scalar multiplication distributes of tuple addition} && \quad k(\mathbf{a}+\mathbf{b})=k\mathbf{a}+k\mathbf{b} &\\
%
& \text{scalar addition distributes over tuples} && \quad (k+l)\mathbf{a}=k\mathbf{a}+l\mathbf{a} &\\
%
& \text{scalar multiplication order doesn't matter} && \quad k(l\mathbf{a})=(kl)\mathbf{a} &\\
%
& \text{scalar multiplication by 1 is an identity operation} && \quad 1\mathbf{a}=\mathbf{a} &
\end{align*}