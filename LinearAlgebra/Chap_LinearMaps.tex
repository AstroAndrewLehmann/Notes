\chapter{Linear Maps} \label{ch:linearmaps}


\section{Basic properties}
Let's build the concept of a function, call it $f$, taking vectors as inputs and giving vectors as outputs. We therefore write $f:V \to W$ for vector spaces $V$ and $W$. For example, imagine a function that takes any Euclidean vector in $\mathbb{R}^2$ and rotates it by $45^\circ$ anti-clockwise, keeping the length of the vector fixed. We could write $f(\mathbf{v})=\mathbf{w}$ for $\mathbf{v}, \mathbf{w}\in \mathbb{R}^2$. Below we sketch a couple of examples.

\begin{figure}[H]
\centering
\begin{tikzpicture}[% styles used in image code
         > = Straight Barb, % defined in "arrows.meta
dot/.style = {circle, fill,
              minimum size=2mm, inner sep=0pt, outer sep=0pt,
              node contents={}},
box/.style = {draw, thin, minimum  width=2mm, minimum height=4mm,
              inner sep=0pt, outer sep=0pt,
              node contents={}, sloped},
my angle/.style args = {#1/#2}{draw,->,
                               angle radius=#1,
                               angle eccentricity=#2,
                               } % angle label position!
                        ]
	% 1st coordinate axis
	\coordinate (O1) at (0,0);
	\draw[->] (-1, 0)   -- (5,0) node[below left] {$x$};
	\draw[->] ( 0,-0.5) -- (0,4) node[below left] {$y$};
	
	% 2nd coordinate axis
	\coordinate (O2) at (10,0);
	\draw[->] ($(O2)-(1,0)$)   -- ($(O2)+(5,0)$) node[below left] {$x$};
	\draw[->] ($(O2)-(0,0.5)$) -- ($(O2)+(0,4)$) node[below left] {$y$};
	
	
    \draw[->,ultra thick] (4,2.5) to[bend left] node[pos=0.5,above] {$f$}  (7,3);

	\coordinate (v1) at (3,1);
	\coordinate (v2) at (1,2);
	
	\coordinate (w1) at (1.414,2.828);
	\coordinate (w2) at (-2.121,2.121);
	
	\draw (O1) node[below left] {$\mathcal{O}$};
	\draw [->,blue,very thick] (O1) --(v1) node[above right,black] {$(3,1)$};
	\draw [->,red,very thick] (O1) --(v2) node[above right,black] {$(1,2)$};
	
	\draw [->,blue,very thick] (O2) --($(O2) + (w1)$) node[above right,black] {$f(3,1)$};
	\draw [->,red,very thick] (O2) --($(O2) + (w2)$) node[above right,black] {$f(1,2)$};
\end{tikzpicture}
\end{figure}

\noindent For an arbitrary vector $\mathbf{v}=(x,y)$, $f$ maps $\mathbf{v}$ to the coordinates $(x/\sqrt{2}-y/\sqrt{2}, \, x/\sqrt{2}+y/\sqrt{2})$. We then ask the question, if we have two arbitrary vectors $\mathbf{v}=(x,y)$ and $\mathbf{u}=(s,t)$ which add to the third vector $\mathbf{w} = \mathbf{v} + \mathbf{u}$, where does $f$ map the addition $\mathbf{w}$ to? Do we get the same answer as if we first rotate $\mathbf{v}$ and $\mathbf{u}$ and then add the rotated vectors together? Let's see
\begin{align*}
f(\mathbf{v}+\mathbf{u}) &= f(x+s, \, y+t) \\
&= \frac{1}{\sqrt{2}}(x+s - y-t, \, x+s+y+t) \\
&= \frac{1}{\sqrt{2}}(x-y, \, x+y) + \frac{1}{\sqrt{2}}(s-t, \, s+t) \\
&= f(x,y) + f(s,t)\\
&= f(\mathbf{v}) + f(\mathbf{u})
\end{align*}

\noindent We have indeed that we can rotate $\mathbf{v}$ and $\mathbf{u}$ and then add up the result, or we can add $\mathbf{v}$ and $\mathbf{u}$ and then rotate the result to get the same outcome. 

This lack of importance in the order of the application of the function is not necessarily true for any function we could think of. Consider a function, $g$, that takes a vector in $\mathbb{R}^2$ and gives another vector in the same direction with length equal to the square of the original vector's length. This would be represented by
\begin{align*}
g(x,y) = \sqrt{x^2 + y^2} (x,y).
\end{align*}
Now, for example, take two vectors $\mathbf{v}=(2,0)$ and $\mathbf{u}=(0,2)$. We have $g(\mathbf{v})=(4,0)$, $g(\mathbf{u})=(0,4)$ and $g(\mathbf{v}+\mathbf{u}) = g(2,2) = (4\sqrt{2},4\sqrt{2})$. So for this function, we have $g(\mathbf{v}+\mathbf{u}) \neq g(\mathbf{v})+g(\mathbf{u})$. The order matters. In linear algebra we study functions of the first type and not the second. These functions are called \textit{linear maps}, defined below:

\definition{Linear map}{
A mapping, $f$, from a vector space $V$ to a vector space $W$, denoted $f:V \to W$, is called a \textit{linear map} if it satisfies the following property:
\begin{align*}
& \forall \mathbf{u}, \, \mathbf{v} \in V, \, \forall \alpha,\beta \in \mathbb{R} \\
& f(\alpha\mathbf{u} + \beta\mathbf{v}) = \alpha f(\mathbf{u}) + \beta f(\mathbf{v}).
\end{align*}
We say that a linear map \textit{preserves linear combinations}.
}

\example{Differentiation as a linear map}{Let's define a mapping $f: \mathcal{P}_n \to \mathcal{P}_{n-1}$ that takes a polynomial of degree up to $n$ (a member of the vector space of polynomials of degree up to $n$) and differentiates it. For example
\begin{align*}
f(1 + 3x^2) &= 6x \\
f(3) &= 0 \\
f(2x + x^2) &= 2 + 2x 
\end{align*}
and you get the idea. Consider two arbitrary vectors
\begin{align*}
\mathbf{u} &= \alpha_0 + \alpha_1 x + \alpha_2 x^2 + \cdots + \alpha_n x^n \\
\mathbf{v} &= \beta_0 + \beta_1 x + \beta_2 x^2 + \cdots + \beta_n x^n.
\end{align*}
We consider, separately, the action of $f$ on these vectors and on their addition
\begin{align*}
f(\mathbf{u}) &= \alpha_1 + 2\alpha_2 x + \cdots + n\alpha_n x^{n-1} \\
f(\mathbf{v}) &= \beta_1 + 2\beta_2 x + \cdots + n\beta_n x^{n-1} \\
f(\mathbf{u} + \mathbf{v}) &= f\left((\alpha_0+\beta_0) + (\alpha_1 + \beta_1) x + (\alpha_2 + \beta_2) x^2 + \cdots + (\alpha_n + \beta_n) x^{n} \right) \\
&= (\alpha_1 + \beta_1) + 2(\alpha_2 + \beta_2) x + \cdots + n(\alpha_n + \beta_n) x^{n-1}
\end{align*}
This last expression can be split by collecting alphas and betas
\begin{align*}
(\alpha_1 + 2\alpha_2 x + \cdots + n\alpha_n x^{n-1}) + (\beta_1+ 2\beta_2 x + \cdots + n\beta_nx^{n-1}) = f(\mathbf{u}) + f(\mathbf{v})
\end{align*}
showing that the derivative of the addition is the addition of the derivatives. Hence we can consider differentiation of polynomials as a linear map.
}

\theorem{A linear map conserves the zero vector}{
For any linear map, $f:V \to W$, we have
\begin{align*}
f(\mathbf{0}_V) = \mathbf{0}_W
\end{align*}
where $\mathbf{0}_V$ is the zero vector of $V$ and $\mathbf{0}_W$ is the zero vector of $W$.
}

\noindent \textbf{Proof}. By the definition of a linear map, we can choose $\alpha=\beta=0 \in \mathbb{R}$ and we must have for any $\mathbf{u}, \, \mathbf{v} \in V$ the following
\begin{align*}
f(0\mathbf{u} + 0\mathbf{v}) = 0f(\mathbf{u}) + 0f(\mathbf{v}).
\end{align*}
We proved that the number 0 multiplied by any vector gives the zero vector in that space. Hence
\begin{align*}
0\mathbf{u} + 0\mathbf{v} = \mathbf{0}_V+ \mathbf{0}_V =\mathbf{0}_V \quad\text{and}\quad 0f(\mathbf{u}) + 0f(\mathbf{v}) = \mathbf{0}_W + \mathbf{0}_W =\mathbf{0}_W
\end{align*}
and so we have proved
\begin{align*}
f(\mathbf{0}_V) = \mathbf{0}_W.
\end{align*}

\definition{Image}{
The \textit{image of a linear map} $f: V \to W$, denoted $\text{im}(f)$, is the set of all possible ``output'' vectors of the map:
\begin{align*}
\text{im}(f) = \{ \mathbf{w} \in W \,\, | \,\, \exists \mathbf{v}\in V\, f(\mathbf{v}) = \mathbf{w} \} \subseteq W.
\end{align*}
}

\noindent This can be understood pictorially as so:
\begin{figure}[H]
\centering
\begin{tikzpicture}
	\coordinate (v) at (0,-1.45);
	\coordinate (w) at (5.5,-1.3);
	
	% domain set V
	\fill[pattern=north west lines, pattern color=blue,opacity=.5,draw] 
  (0,0) ellipse (1.5cm and 2.5cm);
    \draw[very thick,blue] (0,0) ellipse (1.5cm and 2.5cm);
    \node[align=left,scale=3] at (-2.5,0) {$V$};
    \fill[draw] (0,-1.5) circle (0.1cm and 0.1cm);
    \node[above,scale=1] at (v) {$\mathbf{v}$};
    
	% image set f(V) inside W
	\fill[pattern=north west lines, pattern color=blue,opacity=.5,draw] 
  (6,-1) ellipse (1.5cm and 1cm);
    \draw[thick] (6,-1) ellipse (1.5cm and 1cm);
    \node[align=left,scale=1.5] at (6,-0.7) {$\text{im}(f)$};
  
	% codomain set W
    \draw[very thick] (6,0) ellipse (2cm and 2.7cm);
    \node[align=right,scale=3] at (9,0) {$W$};
    \fill[draw] (5.5,-1.3) circle (0.1cm and 0.1cm);
    \node[right,scale=1] at (w) {$\mathbf{w}$};
    
    % mapping lines
	\draw[->, thick, blue, dashed] (0,2.5) to[bend left] node[pos=0.5,above, black,scale=1] {$f$}  (6,0);
	\draw[->, thick, blue, dashed] (0,-2.5) to[bend right] (6,-2);
	\draw[->, thick, blue] (v) to[bend right] node[pos=0.5,above, black,scale=1] {$f(\mathbf{v})$}  ($(w)-(0.1,0.1)$);
\end{tikzpicture}
\end{figure}

\noindent The image is a vector subspace of $W$. Let's prove this. Firstly, we previously showed that any linear map takes the zero vector of $V$ to the zero vector of $W$. So $\mathbf{0}_W \in \text{im}(f)$ and hence it is not an empty set. Let $\mathbf{u}$, $\mathbf{v}\in \text{im}(f)$ and $\alpha$, $\beta \in \mathbb{R}$. There must exist corresponding vectors in $V$, $\mathbf{u}'$ and $\mathbf{v}'$ such that $f(\mathbf{u}')=\mathbf{u}$ and $f(\mathbf{v}')=\mathbf{v}$. Hence
\begin{align*}
\alpha \mathbf{u} + \beta \mathbf{v} = \alpha f(\mathbf{u}') + \beta f(\mathbf{v}') = f(\alpha \mathbf{u}' + \beta \mathbf{v}')
\end{align*}
where we have used the definition of a linear map in the last step. This shows that for any linear combination of vectors in the image of $f$, we can find a corresponding vector in $V$. This means the $\alpha \mathbf{u} + \beta \mathbf{v} \in \text{im}(f)$. Hence the image is a non-empty subset of $W$ that is closed under linear combinations, that is, it is a vector subspace of $W$.

\theorem{Generator of the image}{The image of any linear map, $f:V\to W$, has a generator set that can be found by the action of $f$ on any basis of $V$. That is, let $V$ be an $n$ dimensional vector space and $\mathcal{B}_V=\{\mathbf{b}_1, \dots, \mathbf{b}_n\}$ be a basis of $V$. Then the set $\mathcal{C}=\{f(\mathbf{b}_1),\dots, f(\mathbf{b}_n)\}\subset W$ generates $\text{im}(f)$, that is, 
\begin{align*}
\text{im}(f) = \text{SPAN}\left( f(\mathbf{b}_1),\dots, f(\mathbf{b}_n) \right)
\end{align*}
}

\begin{proof}
Let $\mathbf{w}\in \text{im}(f)$. This means there must exist a $\mathbf{v}\in V$ such that $f(\mathbf{v})=\mathbf{w}$. As $\mathcal{B}_V$ is a basis of $V$, the vector $\mathbf{v}$ can be expressed as a linear combination of the basis vectors
\begin{align*}
\mathbf{v} = \alpha_1 \mathbf{b}_1 + \cdots + \alpha_n \mathbf{b}_n
\end{align*}
hence
\begin{align*}
\mathbf{w} = f(\mathbf{v}) = f(\alpha_1 \mathbf{b}_1 + \cdots + \alpha_n \mathbf{b}_n) = \alpha_1 f(\mathbf{b}_1) + \cdots + \alpha_n f(\mathbf{b}_n).
\end{align*}
This shows that any vector of the image can be expressed as a linear combination of vectors in the set $\mathcal{C}=\{f(\mathbf{b}_1),\dots, f(\mathbf{b}_n)\}$, i.e.
\begin{align*}
\text{im}(f) \subset \text{SPAN}\left( f(\mathbf{b}_1),\dots, f(\mathbf{b}_n) \right).
\end{align*}
The other direction is trivial, but worth the practice. Let $\mathbf{w}\in \text{SPAN}\left( f(\mathbf{b}_1),\dots, f(\mathbf{b}_n) \right)$. This means it can be written as a linear combination of these vectors
\begin{align*}
\mathbf{w} &= \alpha_1 f(\mathbf{b}_1) + \cdots + \alpha_n f(\mathbf{b}_n) \\
&= f(\alpha_1 \mathbf{b}_1 + \cdots + \alpha_n \mathbf{b}_n)
\end{align*}
which means we have found a corresponing vector $\mathbf{v}=\alpha_1 \mathbf{b}_1 + \cdots + \alpha_n \mathbf{b}_n \in V$ such that $f(\mathbf{v}) = \mathbf{w}$. Hence $\mathbf{w}\in \text{im}(f)$. This proves 
\begin{align*}
\text{SPAN}\left( f(\mathbf{b}_1),\dots, f(\mathbf{b}_n) \right) \subset \text{im}(f)
\end{align*}
and so we have the equality of these two sets.
\end{proof}

\definition{Rank}{
The \textit{rank of a linear map} is the dimension of its image: $\text{rank}(f)=\dim(\text{im}(f))$.
}

\example{Rank of a linear map from $\mathbb{R}^3$ to $\mathbb{R}^3$}{Let's find the dimension of a given linear map. Let $f:\mathbb{R}^3 \to \mathbb{R}^3$ be the linear map defined by
\begin{align*}
f(x,y,z) = (x+z, z-y, y-x-2z).
\end{align*}
We take the canonical basis of $\mathbb{R}^3$, $\mathcal{C}=\{\mathbf{e}_1,\mathbf{e}_2,\mathbf{e}_3\}=\{(1,0,0),\, (0,1,0), \, (0,0,1)\}$. From theorem ... we can form the generator set
\begin{align*}
\mathcal{B} &= \{f(\mathbf{e}_1),\, f(\mathbf{e}_2), \, f(\mathbf{e}_3) \} \\
&= \{(1,0,-1), \, (0,-1,1), \, (1,1,-2) \}.
\end{align*}
\noindent Now this set is not a basis, because the third vector is the subtraction of the second from the first (and thus the set $\mathcal{B}$ is not a set of linearly independent vectors). So we can drop the third vector to find
\begin{align*}
\text{im}(f) = \text{SPAN}((1,0,-1), \, (0,-1,1), \, (1,1,-2) ) = \text{SPAN}((1,0,-1), \, (0,-1,1))
\end{align*}
and thus we have the basis of the image
\begin{align*}
\mathcal{C} &= \{(1,0,-1), \, (0,-1,1) \}
\end{align*}
which means the dimension is 2 and hence $\text{rank}(f)=2$.
}

\definition{Kernel}{
The \textit{kernel of a linear map}  $f: V \to W$, denoted $\ker(f)$, is the set of vectors that $f$ maps to the zero vector, $\mathbf{0}_W$, of $W$. That is,
\begin{align*}
\ker(f) = \{\mathbf{v} \in V \,\, | \,\, f(\mathbf{v}) = \mathbf{0}_W \}.
\end{align*}
}

\noindent This can be understood pictorially as so:
\begin{figure}[H]
\centering
\begin{tikzpicture}
	\coordinate (v) at (0,-1.45);
	\coordinate (Ow) at (5.2,-1.1);
	
	% domain set V
	%\fill[pattern=north west lines, pattern color=blue,opacity=.5,draw] 
		%(0,0) ellipse (1.5cm and 2.5cm);
    \draw[very thick,blue,rotate around={-15:(0,0)}] (0,0) ellipse (1.7cm and 2.5cm);
    \node[align=left,scale=3] at (-2.5,0) {$V$};
    %\fill[draw] (0,-1.5) circle (0.1cm and 0.1cm);
    %\node[above,scale=1] at (v) {$\mathbf{v}$};
  
	% codomain set W
    \draw[very thick] (6,0) ellipse (2cm and 2.7cm);
    \node[align=right,scale=3] at (9,0) {$W$};
    
	% kernel set ker(f) inside V
	\fill[pattern=north west lines, pattern color=red,opacity=.5,draw] 
  		(0.3,1) ellipse (1.2cm and 0.9cm);
    \draw[thick,red] (0.3,1) ellipse (1.2cm and 0.9cm);
    \node[align=left,scale=1.5] at (0.35,1) {$\ker(f)$};
    
	% image set f(V) inside W
    \draw[thick] (6,-1) ellipse (1.5cm and 1cm);
    \node[align=left,scale=1.5] at (6.5,0.5) {$\text{im}(f)$};
    
    % zero vector in W
    \fill[draw,red] (Ow) circle (0.1cm and 0.1cm);
    \node[right=2pt,scale=1] at (Ow) {$\mathbf{0}_W$};
    
    % mapping lines
	\draw[->, thick, red, dashed] (0.5,1.9) to[bend left] node[pos=0.5,above, black,scale=1] {$f$}  ($(Ow)+(-0.12,0.3)$);
	\draw[->, thick, red, dashed] (-0.4,0.2) to[bend right] ($(Ow)+(-0.17,-0.1)$);
\end{tikzpicture}
\end{figure}



\example{Kernel of a linear map from $\mathbb{R}^3$ to $\mathbb{R}^3$}{Let's find the kernel of the linear map from the previous example, the map $f:\mathbb{R}^3 \to \mathbb{R}^3$ defined by
\begin{align*}
f(x,y,z) = (x+z, z-y, y-x-2z).
\end{align*}
We want to find all the triples that $f$ takes to the zero vector of $\mathbb{R}^3$, which is $(0,0,0)$. Hence we are looking to solve
\begin{align*}
f(x,y,z)=(0,0,0) \quad \implies \quad (x+z, z-y, y-x-2z) = (0,0,0).
\end{align*}
Equating components gives us the linear system
\begin{align*}
\begin{cases}
x+z = 0 \\
z-y=0 \\ 
y-x-2z =0
\end{cases}
%
\quad \implies \quad
%
\begin{cases}
x = -z \\
y=z \\ 
y-x-2z =0
\end{cases}.
\end{align*}
So $x$ and $y$ can each be expressed in terms of $z$ and the 3rd equation gives no extra constraint. $z$ is therefore a free variable, denote it $z=t\in\mathbb{R}$, and we have
\begin{align*}
f(x,y,z) = (0,0,0)\quad \implies \quad (x,y,z)=(-t,t,t)=(-1,1,1)t.
\end{align*}
So we can write the kernel as the set
\begin{align*}
\ker(f) = \{(-1,1,1)t \, | \, t\in\mathbb{R} \}.
\end{align*}
}

\noindent The kernel is a vector subspace of $W$. It is non-empty because the zero vector necessarily maps to the zero, $f(\mathbf{0}_V)=\mathbf{0}_W$, and so $\mathbf{0}_V\in \ker(f)$. Let $\mathbf{u},\mathbf{v}\in \ker(f)$ and $\alpha,\beta \in\mathbb{R}$. Then $f$ maps the linear combination as
\begin{align*}
f( \alpha \mathbf{u} + \beta \mathbf{v}) =  \alpha f(\mathbf{u}) + \beta f(\mathbf{v}) = \alpha \mathbf{0}_W + \beta \mathbf{0}_W = \mathbf{0}_W
\end{align*}
and so $\alpha \mathbf{u} + \beta \mathbf{v}$ is also in the kernel of $f$. Thus $\ker(f)$ is closed under linear combinations and is a non-empty subset of $V$. Therefore it is a vector subspace of $V$.

\definition{Nullity}{
The \textit{nullity of a linear map} is the dimension of its kernel: $\text{nullity}(f)=\dim(\ker(f))$.
}





\example{Nullity of a linear map from $\mathbb{R}^3$ to $\mathbb{R}^3$}{Let's retake the linear map from the previous example, $f:\mathbb{R}^3 \to \mathbb{R}^3$ defined by
\begin{align*}
f(x,y,z) = (x+z, z-y, y-x-2z).
\end{align*}
We found the kernel as the set
\begin{align*}
\ker(f) = \{(-1,1,1)t \, | \, t\in\mathbb{R} \} = \text{SPAN}( \, (-1,1,1) \, ).
\end{align*}
This means that the set $\mathcal{B} = \{ (-1,1,1) \}$ is a basis for the kernel. Hence the dimension of kernel is 1, and so
\begin{align*}
\text{nullity}(f) = \dim(\ker(f)) = 1.
\end{align*}
}

\theorem{Rank-Nullity}{
For any linear map $f: V \to W$ we have
\begin{gather*}
\text{rank}(f) + \text{nullity}(f) = \dim(V) \\
\text{or} \\
\dim(\text{im}(f)) + \dim(\ker(f)) = \dim(V). 
\end{gather*}
}

\example{Projection map onto the $x$-$y$ plane}{
Consider the linear map $f:\mathbb{R}^3 \to \mathbb{R}^3$ defined by
\begin{align*}
f(x,y,z) = (x,y,0).
\end{align*}
This map takes any vector in 3d space and gives you the component of that vector in the $x$-$y$ plane. Here's a sketch of the action of $f$ on a couple of example vectors

\begin{figure}[H]
\centering
\tdplotsetmaincoords{105}{-30}
\begin{tikzpicture}[tdplot_main_coords,font=\sffamily]
  \begin{scope}[canvas is xy plane at z=0]
    \fill[blue,fill opacity=0.1] (-3,-3) rectangle (3,4); 
   
   \pgflowlevelsynccm
  \end{scope}
    
  \coordinate (O) at (0,0);
  \coordinate (u) at (2,5,2);
  \coordinate (fu) at (2,5,0);
  \coordinate (v) at (3,-1,2);
  \coordinate (fv) at (3,-1,0);
  \draw [->,red,thick] (O) --(u) node[pos=0.6,below=1pt] {$\mathbf{u}$};
  \draw [->,blue,dashed,thick] (O) --(fu) node[pos=0.6,right=1pt] {$f(\mathbf{u})$};
  \draw [-,black,dashed] ($0.1*(fu)+0.9*(u)$) --($0.8*(fu)+0.2*(u)$);
  
  
  \draw [->,red,thick] (O) --(v) node[pos=0.6,above=1pt] {$\mathbf{v}$};
  \draw [->,blue,dashed,thick] (O) --(fv) node[pos=0.6,above=1pt] {$f(\mathbf{v})$};
  \draw [-,black,dashed] ($0.1*(fv)+0.9*(v)$) --($0.9*(fv)+0.1*(v)$);
    
  \pgfmathsetmacro{\Radius}{1.5}
  \draw[-stealth] (O)-- (2.5*\Radius,0,0) node[pos=1.15] {$y$};
  \draw[-stealth] (O) -- (0,3.5*\Radius,0) node[pos=1.15] {$x$};
  \draw[-stealth] (O) -- (0,0,2*\Radius) node[pos=1.05] {$z$};
\end{tikzpicture}
\end{figure}

\noindent Now the image of this function is clearly all of the $x$-$y$ plane, but let's show that mathematically. Take the canonical basis of $\mathbb{R}^3$: $\mathcal{C}=\{\mathbf{e}_1,\mathbf{e}_2,\mathbf{e}_3\}=\{(1,0,0),\, (0,1,0), \, (0,0,1)\}$. We have the generator set
\begin{align*}
\mathcal{B} &= \{f(\mathbf{e}_1),\, f(\mathbf{e}_2), \, f(\mathbf{e}_3) \} \\
&= \{(1,0,0), \, (0,1,0), \, (0,0,0) \}.
\end{align*}
The span of this set is the same as if we drop the zero vector, so
\begin{align*}
\mathcal{C} = \{(1,0,0), \, (0,1,0) \}
\end{align*}
is a set of linearly independent vectors such that $\text{im}(f)=\text{SPAN}(\mathcal{C})$, which means $\mathcal{C}$ is a basis for the image. With two vectors in this basis, we have the dimension of the image, and therefore the rank of $f$, is 2.

Now for the kernel we must solve
\begin{align*}
f(x,y,z)=(0,0,0).
\end{align*}
This gives us $(x,y,0) = (0,0,0)$ so that $x=0$ and $y=0$. There is no restriction on $z$, so that the kernel can be written as the set
\begin{align*}
\ker(f) = \{ (0,0,1)t \, | \, t\in\mathbb{R}\} = \text{SPAN}((0,0,1)).
\end{align*}
So we can form the obvious basis $\mathcal{D}=\{ (0,0,1) \}$, showing the dimension of the kernel, and hence the nullity of $f$, is 1. We thus verify that for this linear map, we have $\text{rank}(f) + \text{nullity}(f) = 2 + 1 = 3 = \dim(\mathbb{R}^3)$.
}

\definition{Injectivity}{
Let $f:V\to W$ be a linear map. We say $f$ is injective if no two vectors of $V$ are mapped to the same vector of $W$. In symbols we have two equivalent expressions
\begin{gather*}
\forall \, \mathbf{x},\mathbf{y}\in V, \quad \left(f(\mathbf{x})=f(\mathbf{y}) \implies \mathbf{x}=\mathbf{y}\right) \\
%
\text{or} \\
%
\forall \, \mathbf{x},\mathbf{y}\in V, \quad \left( \mathbf{x} \neq \mathbf{y} \implies f(\mathbf{x}) \neq f(\mathbf{y})\right)  
\end{gather*}
}

\definition{Surjectivity}{
Let $f:V\to W$ be a linear map. We say that $f$ is surjective if every vector in the output space has a corresponding input vector. In symbols
\begin{align*}
\forall \, \mathbf{w} \in W \quad \exists \mathbf{v}\in V \, \text{such that} \, f(\mathbf{v})=\mathbf{w}.
\end{align*}
}

\definition{Categories of linear maps}{
Let $f:V\to W$ be a linear map.
\begin{itemize}
\item If $W=V$ we call $f$ an \textit{endomorphism}.
\item If $f$ is both injective and surjective then we say it is bijective and we call it an \textit{isomorphism}.
\item If $f$ is both an isomorphism and an endomorphism we call it an \textit{automorphism}.
\end{itemize}
}

\definition{Composition of linear maps}{
Composition of linear maps works exactly as you would expect if you remember the composition of regular functions. We must have a coherence between the output of one linear map and the input of another. So, two linear maps $f:A\to B$ and $g:U\to V$ can be composed as a well defined linear map $g\circ f$ (``$g$ of $f$'') if and only if the output space of $f$ is the input space of $g$: $U=B$. For any $\mathbf{u}\in A$ the composition is written
\begin{align*}
g\circ f: A \to V \quad \text{and} \quad (g\circ f)(\mathbf{u}) = g(f(\mathbf{u})).
\end{align*}
}

\noindent The composition can be represented pictorially as so:
\begin{figure}[H]
\centering
\begin{tikzpicture}
	\coordinate (a) at (-0.5,-1.45);
	\coordinate (b) at (5.5,-1.3);
	\coordinate (v) at (11,0);
	
	% domain set A
	\fill[pattern=north west lines, pattern color=blue,opacity=.3,draw] 
  (0,0) ellipse (1.5cm and 2.5cm);
    \draw[very thick,blue] (0,0) ellipse (1.5cm and 2.5cm);
    \node[align=center,scale=3] at (0,3.5) {$A$};
    \fill[draw] (a) circle (0.1cm and 0.1cm);
    \node[left,scale=1] at (a) {$\mathbf{a}$};
  
	% codomain set B=U
    \draw[very thick] (6,0) ellipse (2cm and 2.7cm);
    \node[align=center,scale=3] at (6,3.5) {$B=U$};
    \fill[draw] (b) circle (0.1cm and 0.1cm);
    \node[right,scale=1] at (b) {$\mathbf{b}$};
  
	% codomain set V
    \draw[very thick] (12,0) ellipse (2cm and 1.5cm);
    \node[align=center,scale=3] at (12,3.5) {$V$};
    \fill[draw] (v) circle (0.1cm and 0.1cm);
    \node[right,scale=1] at (v) {$\mathbf{v}$};
    
	% image set g(A) inside B
	\fill[pattern=north west lines, pattern color=blue,opacity=.3,draw] 
  (6,-1) ellipse (1.5cm and 1cm);
    \draw[very thick,blue] (6,-1) ellipse (1.5cm and 1cm);
    \node[align=left,scale=1.5] at (6,-0.7) {$\text{im}(f)$};
  
	% iamge set f(g(A))
	\fill[pattern=north west lines, pattern color=blue,opacity=.3,draw] 
  (12,0.5) ellipse (1.5cm and 0.9cm);
    \draw[very thick,blue] (12,0.5) ellipse (1.5cm and 0.9cm);
    \node[align=left,scale=1.5] at (12,0.7) {$\text{im}(g\circ f)$};
    
    % mapping lines
    % A to B
	\draw[->, thick, blue, dashed] (0,2.5) to[bend left] node[pos=0.5,above, black,scale=1.5] {$f$}  (5.8,0);
	\draw[->, thick, blue, dashed] (0,-2.5) to[bend right] (5.8,-2);
	
    % B to V
	\draw[->, thick, blue, dashed] (6.2,0) to[bend left] node[pos=0.5,above, black,scale=1.5] {$g$}  (12,1.4);
	\draw[->, thick, blue, dashed] (6.2,-2) to[bend right] (12,-0.4);
	
	% example functions f(a)=b, g(b)=v, and g(f(a))=v
	\draw[->, thick, blue] (a) to[bend right] node[pos=0.5,above, black,scale=1] {$f(\mathbf{a})$}  ($(b)-(0.1,0.1)$);
	\draw[->, thick, blue] ($(b)+(0.5,-0.15)$) to[bend right]
	 node[pos=0.5,above, black,scale=1] {$g(\mathbf{b})$}  ($(v)-(0.15,0.15)$);
	\draw[->, thick, blue] ($(a)+(0.1,0.1)$) to[bend left]
	 node[pos=0.3,above,rotate=15,black,scale=1] {$(g\circ f)(\mathbf{a})$}  ($(v)+(-0.15,0.15)$);
\end{tikzpicture}
\end{figure}


\example{Composition of linear maps}{
Let's come up with a couple of linear maps and then take their composition, why not? Consider $f:\mathbb{R}^2 \to \mathcal{P}_3[\mathbb{R}]$ and $g: \mathcal{P}_3[\mathbb{R}]\to \mathbb{R}^3$ defined by
\begin{align*}
& f(\alpha,\beta) = \alpha + (\beta-\alpha)x + (\alpha + 2\beta)x^3 \\
& g(a_0 + a_1 x + a_2 x^2 + a_3 x^3) = (a_3, \, a_2-a_1, \, a_0+a_3).
\end{align*}
Now the composition of these linear maps will skip the polynomial space (the output of $f$ and input of $g$):
\begin{align*}
g \circ f: & \mathbb{R}^2 \to \mathbb{R}^3 \\
(g \circ f)(\alpha,\beta) & = g \left(f(\alpha,\beta)\right) \\
 & = g \left(\alpha + (\beta-\alpha)x + (\alpha + 2\beta)x^3\right) \\
 & = (\alpha + 2\beta, \, \alpha-\beta, \, 2\alpha + 2\beta)
\end{align*}
So let's see for example what happens to the vector $\mathbf{v}=(1,1)$. If we consider the step-by-step process, first act with $f$ to obtain a polynomial, then hit that polynomial with $g$, we find
\begin{align*}
f(\mathbf{v}) & = f(1,1) = 1 + 3x^3 \\
\implies g(f(\mathbf{v})) &= g(1+3x^3) = \left(3, \, 0, \, 4 \right).
\end{align*}
If we want to avoid this 2-step process we can use the composition rule as we found above
\begin{align*}
(g \circ f)(1,1) = (1 + 2, \, 1-1, \, 2 + 2) = (3, \, 0, \, 4).
\end{align*}
Of course we get the same answer, but the lesson here is that we never had to think about polynomials this way.
}



\theorem{Inverse linear map}{
Let $f:V\to W$ be a linear map. If $f$ is bijective then there exists a linear map $g:W\to V$ such that
\begin{gather*}
\forall \, \mathbf{v}\in V \, \text{we have} \, (g \circ f)(\mathbf{v}) = \mathbf{v}\\
\text{and}\\
\forall \, \mathbf{w}\in W \, \text{we have} \, (f \circ g)(\mathbf{w}) = \mathbf{w}
\end{gather*}
$g$ is called the inverse of $f$ and is denoted $f^{-1}$.
}

\section{The vector space of linear maps}


\theorem{The set of linear maps as a vector space}{
Consider the set of all possible linear maps from the vector space $V$ to the vector space $W$, denoted $\mathcal{L}(V,W)$. This set satisfies all of the vector space axioms if we define vector addition and scalar multiplication as follows:
\begin{gather*}
\forall f,g \in \mathcal{L}(V,W), \quad f+g=h \quad \text{such that} \quad h(\mathbf{v})=f(\mathbf{v}) + g(\mathbf{v}) \\
\text{and} \\
\forall \alpha \in \mathbb{R}\, \quad \alpha f = f_\alpha \quad \text{such that} \quad f_\alpha(\mathbf{v})=\alpha \left( f(\mathbf{v}) \right).
\end{gather*}
}

\noindent \textbf{Proof}. Let $f,g,h\in \mathcal{L}(V,W)$ and $\alpha,\beta\in\mathbb{R}$. Let's go through the vector space axioms in order:

\noindent \underline{VA1 - closure under vector addition: $f+g\in \mathcal{L}(V,W)$} \\
\noindent Let the addition be denoted $h = f+g$. We have to show that $h$ is a linear map from $V$ to $W$. For every $\mathbf{v},\mathbf{u}\in V$ and $\alpha,\beta\in\mathbb{R}$ we have
\begin{align*}
h(\alpha\mathbf{v} + \beta\mathbf{u}) &= f(\alpha\mathbf{v} + \beta\mathbf{u}) + g(\alpha\mathbf{v} + \beta\mathbf{u}) \\
%
&= \alpha f(\mathbf{v}) + \beta f (\mathbf{u}) + \alpha g(\mathbf{v}) + \beta g(\mathbf{u}) \quad \text{(because $f$ and $g$ are linear maps)}\\
%
&= \alpha \left( f(\mathbf{v}) + g(\mathbf{v})\right)  + \beta\left( f (\mathbf{u})+ g(\mathbf{u})\right) \\
%
&= \alpha h(\mathbf{v}) + \beta h(\mathbf{u}).
\end{align*}
So $f+g$ preserves linear combinations and is therefore a linear map. Hence $\mathcal{L}(V,W)$ is closed under vector addition. \\
%%%%%%%


%%%%%%%
\noindent \underline{VA2 - associativity of vector addition: $f+(g+h) = (f+g)+h \in \mathcal{L}(V,W)$}

\noindent For every $\mathbf{v}\in V$ we have
\begin{align*}
(f+(g+h))(\mathbf{v}) &= f(\mathbf{v})+(g+h)(\mathbf{v})  &  (\text{definition of vector addition of linear maps})\\
 &= f(\mathbf{v})+(g(\mathbf{v})+h(\mathbf{v}) )   &  (\text{definition of vector addition of linear maps})\\
 &= (f(\mathbf{v})+g(\mathbf{v}))+h(\mathbf{v})    &  (\text{associativity of vectors in $W$})\\
 &= (f+g)(\mathbf{v})+h(\mathbf{v})    &  (\text{definition of vector addition of linear maps})\\
 &= ((f+g)+h)(\mathbf{v})    &  (\text{definition of vector addition of linear maps}).
\end{align*}
This shows that $f+(g+h)$ is the same linear map as $(f+g)+h$. Hence the addition of vectors in $\mathcal{L}(V,W)$ is associative. \\
%%%%%%%


%%%%%%%
\noindent \underline{VA3 - additive identity: $\exists \, f_0 \in \mathcal{L}(V,W), \, \text{such that} \, f + f_0 = f_0 + f = f$}

\noindent Define the zero map $f_0: V \to W$ by
\begin{align*}
f_0(\mathbf{v}) = \mathbf{0}_W
\end{align*}
for every $\mathbf{v}\in V$. Firstly, is this a linear map? Let's see if it preserves linear combinations
\begin{align*}
f_0(\alpha \mathbf{v} + \beta \mathbf{u}) &= \mathbf{0}_W  &  (\text{definition of the zero map}) \\
%
 &= \mathbf{0}_W + \mathbf{0}_W  &  (\text{definition of the zero vector of $W$}) \\
%
 &= \alpha \mathbf{0}_W + \beta\mathbf{0}_W  &  (\text{a number multiplied by the zero vector is the zero vector})  \\
%
 &= \alpha f_0(\mathbf{v}) + \beta f_0(\mathbf{u})  &  (\text{reverse definition of the zero map}) .
\end{align*}
Indeed, this zero map is a linear map from $V$ to $W$, that is $f_0 \in \mathcal{L}(V,W)$. Now we must show that this map acts as the zero vector of $\mathcal{L}(V,W)$. For every $\mathbf{v}\in V$ we have
\begin{align*}
(f_0 + f)(\mathbf{v}) &= f_0(\mathbf{v})+f(\mathbf{v})  &  (\text{definition of vector addition of linear maps})\\
%
&= \mathbf{0}_W+f(\mathbf{v})  &  (\text{definition of zero map})\\
%
&= f(\mathbf{v})+\mathbf{0}_W  &  (\text{commutativity of vector addition in $W$})\\
%
&= f(\mathbf{v})  &  (\text{definition of zero vector of $W$})
\end{align*}
Hence this zero map is a member of $\mathcal{L}(V,W)$ and satisifes $f + f_0 = f_0 + f = f$. This proves the zero map is the zero \textit{vector} of $\mathcal{L}(V,W)$. \\
%%%%%%%


%%%%%%%
\noindent \underline{VA4 - additive inverse: $\exists \, f_- \in \mathcal{L}(V,W) \, \text{such that} \, f + f_- = f_0$}

\noindent This can easily become symbolically ambiguous, so I will try to be pedantically careful here. Define the additive inverse of any map $f$, denote it $f_-$, as the scalar multiplication of that map (as defined) by the real number $-1$. So that the map $f_-:V\to W$ is given by
\begin{align*}
f_-(\mathbf{v}) = -1 \times f(\mathbf{v}).
\end{align*}
First let's show this map belongs to $\mathcal{L}(V,W)$ by checking the preservation of linear combinations
\begin{align*}
f_-(\alpha \mathbf{v} + \beta \mathbf{u}) &= -1\times f(\alpha \mathbf{v} + \beta \mathbf{u})  &  (\text{definition of this map}) \\
%
 &= -1\times \left( \alpha f(\mathbf{v)} + \beta f(\mathbf{u}) \right)  &  (\text{linearity of the map $f$}) \\
%
 &= -1\times \alpha f(\mathbf{v)} -1\times \beta f(\mathbf{u})  &  (\text{distributivity of the reals}) \\
%
 &= \alpha \times -1\times f(\mathbf{v)} + \beta \times -1\times f(\mathbf{u})  &  (\text{commutativity of real multiplication}) \\
%
 &= \alpha f_-(\mathbf{v)} + \beta f_-(\mathbf{u})  &  (\text{definition of this map}).
\end{align*}
Indeed, this map is a linear map from $V$ to $W$, that is $f_- \in \mathcal{L}(V,W)$. Now we show that it acts as an additive inverse. For every $\mathbf{v}\in V$ we have
\begin{align*}
(f_- + f)(\mathbf{v}) &= f_-(\mathbf{v}) + f(\mathbf{v})  &  (\text{definition of vector addition of linear maps}) \\
%
 &= -1\times f(\mathbf{v}) + f(\mathbf{v})  &  (\text{definition of this map}) \\
%
 &= -f(\mathbf{v}) + f(\mathbf{v})  &  (\text{multiplication by 1 for vectors in $W$}) \\
%
 &= \mathbf{0}_W  &  (\text{additive inverse of vectors in $W$})
\end{align*}
With commutativity of vectors in $W$ you can also show $f_- + f = f + f_-$. Hence this additive inverse map exists in $\mathcal{L}(V,W)$.
\\
%%%%%%%


%%%%%%%
\noindent \underline{VA5 - commutativity of vector addition: $f + g  = g + f$}

\noindent This one is pretty simple. For every $\mathbf{v}\in V$ we have
\begin{align*}
(f + g)(\mathbf{v}) &= f(\mathbf{v}) + g(\mathbf{v})  &  (\text{definition of vector addition of linear maps}) \\
%
&= g(\mathbf{v}) + f(\mathbf{v})  &  (\text{commutativity of vector addition in $W$}) \\
%
&= (g+ f)(\mathbf{v})  &  (\text{definition of vector addition of linear maps}).
\end{align*}
Hence the addition of vectors in $\mathcal{L}(V,W)$ is commutative. \\
%%%%%%%


%%%%%%%
\noindent \underline{SM1 - closure under scalar multiplication: $\alpha f \in \mathcal{L}(V,W)$}

\noindent Recall the definition of scalar multiplication of the linear map $f$ by a real number $k$ gives a new map, denote it $f_k$, such that
\begin{align*}
f_k(\mathbf{v}) = k f(\mathbf{v}).
\end{align*}
Let's show this map belongs to $\mathcal{L}(V,W)$ by checking the preservation of linear combinations
\begin{align*}
f_k(\alpha \mathbf{v} + \beta \mathbf{u}) &= k f(\alpha \mathbf{v} + \beta \mathbf{u})  &  (\text{definition of this map}) \\
%
 &= k \left( \alpha f(\mathbf{v)} + \beta f(\mathbf{u}) \right)  &  (\text{linearity of the map $f$}) \\
%
 &= k \alpha f(\mathbf{v)} +k \beta f(\mathbf{u})  &  (\text{distributivity of the reals}) \\
%
 &= \alpha \times kf(\mathbf{v)} + \beta \times kf(\mathbf{u})  &  (\text{commutativity of real multiplication}) \\
%
 &= \alpha f_k(\mathbf{v)} + \beta f_k(\mathbf{u})  &  (\text{definition of this map}).
\end{align*}
Indeed, this map is a linear map from $V$ to $W$, and so $\mathcal{L}(V,W)$ is closed under scalar multiplication. \\
%%%%%%%


%%%%%%%
\noindent \underline{SM2 - distributivity over vector addition: $\alpha(f+g)=\alpha f+\alpha g$}

\noindent For every $\mathbf{v}\in V$ we have
\begin{align*}
(\alpha(f+g)) (\mathbf{v}) &= \alpha((f+g) (\mathbf{v}))  &  (\text{definition of scalar multiplication for linear maps}) \\
%
&=  \alpha(f(\mathbf{v})+g(\mathbf{v})) &  (\text{definition of vector addition for linear maps}) \\
%
&=  \alpha \left(f(\mathbf{v})\right) +\alpha \left( g(\mathbf{v})\right)  &  (\text{distributivity over vector addition in $W$})\\
%
&=  \left(\alpha f\right)(\mathbf{v}) +\left(\alpha  g\right)(\mathbf{v})  &  (\text{definition of scalar multiplication for linear maps})\\
%
&=  \left(\alpha f + \alpha  g\right)(\mathbf{v})  &  (\text{definition of addition  for linear maps}).
\end{align*}
This shows that $\alpha(f+g)$ is the same linear map as $\alpha f+\alpha g$. Hence scalar multiplication for linear maps distributes over vector addition of linear maps. \\
%%%%%%%


%%%%%%%
\noindent \underline{SM3 - distributivity with field addition: $(\alpha+\beta)f=\alpha f+\beta g$}

\noindent For every $\mathbf{v}\in V$ we have
\begin{align*}
\left( (\alpha+\beta)f \right)(\mathbf{v}) &= (\alpha+\beta)\left( f(\mathbf{v}) \right) 
 & (\text{definition of scalar multiplication for linear maps}) \\
%
 &= \alpha  f(\mathbf{v}) +\beta  f(\mathbf{v})
 & (\text{distributivity over field addition for $W$}) \\
%
 &= \left(\alpha f\right)(\mathbf{v}) + \left(\beta f\right) (\mathbf{v})
 & (\text{definition of scalar multiplication for linear maps}) \\
%
 &= \left(\alpha f + \beta f\right) (\mathbf{v})
 & (\text{definition of vector addition for linear maps}).
\end{align*}
This shows that $(\alpha+\beta)f$ is the same linear map as $\alpha f + \beta f$. Hence field addition distributes over linear maps. 
\\
%%%%%%%


%%%%%%%
\noindent \underline{SM4 - compatibility of scalar and field multiplication: $\alpha(\beta f)=(\alpha\beta) f$}

\noindent For every $\mathbf{v}\in V$ we have
\begin{align*}
\left( \alpha(\beta f) \right)(\mathbf{v}) &= \alpha\left( (\beta f) (\mathbf{v}) \right)
 & (\text{definition of scalar multiplication for linear maps}) \\
%
 &= \alpha\left( \beta (f (\mathbf{v}))  \right)
 & (\text{definition of scalar multiplication for linear maps}) \\
%
 &= \left(\alpha \beta \right)(f (\mathbf{v}))
 & (\text{compatibility of scalar and field multiplication for $W$}) \\
%
 &= \left(\left( \alpha \beta \right)f \right) (\mathbf{v})
 & (\text{definition of scalar multiplication for linear maps})
\end{align*}
This shows that $\alpha(\beta f)$ is the same linear map as $(\alpha\beta) f$. Hence scalar multiplication is compatible with field multiplication for linear maps. 
\\
%%%%%%%


%%%%%%%
\noindent \underline{SM5 - multiplicative identity: $1f=f$}

\noindent For every $\mathbf{v}\in V$ we have
\begin{align*}
\left(1 f \right)(\mathbf{v}) &= 1\left( f(\mathbf{v}) \right)
 & (\text{definition of scalar multiplication for linear maps}) \\
%
 &= f (\mathbf{v})
 & (\text{multiplicative identity for $W$})
\end{align*}
This shows that $1f$ is the same linear map as $f$. Hence the real number $1$ is the scalar multiplicative identity for linear maps. \vspace{0.5cm} \\
\noindent It was long and perhaps tedious but we have now completed the proof that with this definition of linear map vector addition and scalar multiplication, the set $\mathcal{L}(V,W)$ satisfies all 10 of the vector space axioms.






%%%%%%%%%%%%%%%%%%%%
%%%%%%%%%%%%%%%%%%%%
%%%%%%%%%%%%%%%%%%%%
\section{Linear maps as matrices}

Let's recall the definition of coordinates of a vector. Given any vector $\mathbf{v}$ in some $n$-dimensional vector space $V$, if we have a basis $\mathcal{B}=\{ \mathbf{b}_1, \dots, \mathbf{b}_n \}$ then the coordinates of $\mathbf{v}$ \textit{in this basis}, denoted $[\mathbf{v}]_\mathcal{B}$, is a column of $n$ numbers which are the coefficients of the linear combination of $\mathbf{v}$ in the basis vectors
\begin{align*}
\mathbf{v}=\alpha_1 \mathbf{b}_1 + \cdots + \alpha_n \mathbf{b}_n \quad\implies\quad [\mathbf{v}]_\mathcal{B} = \begin{pmatrix} \alpha_1 \\ \vdots \\ \alpha_n \end{pmatrix}.
\end{align*}

\noindent I hope to show in this section something quite powerful about linear algebra, that any $n$-dimensional vector space can be mapped to $\mathbb{R}^n$, via the coordinates, and that linear maps take on a particularly simple to use form when mapping between coordinates. We'll start by an example.

\example{Rotation linear map}{
Let's retake the first example of this chapter, the linear map, $f:\mathbb{R}^2 \to \mathbb{R}^2$, that rotates 2d vectors by $45^\circ$ anti-clockwise, defined in detail by
\begin{align*}
f(x,y) = \frac{1}{\sqrt{2}} \left(x-y, x+y \right).
\end{align*}
Let $\mathcal{C}=\{\mathbf{e}_x, \, \mathbf{e}_y\} = \{(1,0), \, (0,1)\}$. We will want to represent $f$ by some operation, call it $F$, that takes the coordinates of any vector to the coordinates of the rotated vector. Let's write this desired property as
\begin{align*}
F[\mathbf{v}]_\mathcal{C} = [f(\mathbf{v})]_\mathcal{C}.
\end{align*}
Now let $\mathbf{v}=(x,y)$ be some arbitrary vector, so that 
\begin{align*}
[f(\mathbf{v})]_\mathcal{C} &= [f(x\mathbf{e}_x + y \mathbf{e}_y)]_\mathcal{C} \\
%
 &= [ \frac{1}{\sqrt{2}} \left(x-y, x+y \right)]_\mathcal{C} \\
%
 &= [\left(\frac{x - y}{\sqrt{2}}\right)\mathbf{e}_x + \left(\frac{x + y}{\sqrt{2}}\right)\mathbf{e}_y]_\mathcal{C} \\
%
 &= \begin{pmatrix}
 \dfrac{x - y}{\sqrt{2}} \\ \dfrac{x + y}{\sqrt{2}}
 \end{pmatrix}.
\end{align*}
These coordinates can be rearranged to focus on the $x$ and $y$
\begin{align*}
\begin{pmatrix}
 \dfrac{x - y}{\sqrt{2}} \\ \dfrac{x + y}{\sqrt{2}}
\end{pmatrix}
%
=
%
\begin{pmatrix}
 \dfrac{x}{\sqrt{2}} \\ \dfrac{x}{\sqrt{2}}
\end{pmatrix}
+
\begin{pmatrix}
 \dfrac{- y}{\sqrt{2}} \\ \dfrac{y}{\sqrt{2}}
\end{pmatrix}
%
=
%
\begin{pmatrix}
 \dfrac{1}{\sqrt{2}} \\ \dfrac{1}{\sqrt{2}}
\end{pmatrix}x
+
\begin{pmatrix}
 \dfrac{-1}{\sqrt{2}} \\ \dfrac{1}{\sqrt{2}}
\end{pmatrix}y.
\end{align*}
So at this point we have
\begin{align*}
F[\mathbf{v}]_\mathcal{C} = [f(\mathbf{v})]_\mathcal{C} \quad\implies\quad 
F
\begin{pmatrix}
 x \\ y
\end{pmatrix}
%
=
%
\begin{pmatrix}
 \dfrac{1}{\sqrt{2}} \\ \dfrac{1}{\sqrt{2}}
\end{pmatrix}x
+
\begin{pmatrix}
 \dfrac{-1}{\sqrt{2}} \\ \dfrac{1}{\sqrt{2}}
\end{pmatrix}y.
\end{align*}
We will define a certain rearrangement of the right hand side
\begin{align*}
\begin{pmatrix}
 \dfrac{1}{\sqrt{2}} \\ \dfrac{1}{\sqrt{2}}
\end{pmatrix}x
+
\begin{pmatrix}
 \dfrac{-1}{\sqrt{2}} \\ \dfrac{1}{\sqrt{2}}
\end{pmatrix}y
=
\begin{pmatrix}
 \dfrac{1}{\sqrt{2}} & \dfrac{-1}{\sqrt{2}} \\ 
 \dfrac{1}{\sqrt{2}} & \dfrac{1}{\sqrt{2}}
\end{pmatrix}
\begin{pmatrix}
 x \\ y
\end{pmatrix}
\end{align*}
so that the equation
\begin{align*}
F
\begin{pmatrix}
 x \\ y
\end{pmatrix}
%
=
%
\begin{pmatrix}
 \dfrac{1}{\sqrt{2}} & \dfrac{-1}{\sqrt{2}} \\ 
 \dfrac{1}{\sqrt{2}} & \dfrac{1}{\sqrt{2}}
\end{pmatrix}
\begin{pmatrix}
 x \\ y
\end{pmatrix}
\end{align*}
lets us identify the operation of $F$ with this array of 4 numbers
\begin{align*}
F
%
=
%
\begin{pmatrix}
 \dfrac{1}{\sqrt{2}} & \dfrac{-1}{\sqrt{2}} \\ 
 \dfrac{1}{\sqrt{2}} & \dfrac{1}{\sqrt{2}}
\end{pmatrix}
\end{align*}
\noindent This kind of array of numbers is called a \textit{matrix}. In this case we say that $F$ is a matrix representation of the linear map $f$. Its job is to represent the same mapping as $f$ but on the \textit{coordinates} of vectors in the domain and codomain of $f$.
}

\noindent Let's take another example to show that the matrix representation is not always an array of 4 numbers.

\example{Non-square matrix}{
Consider a linear map $f:\mathbb{R}^2 \to \mathcal{P}_2[\mathbb{R}]$ defined by 
\begin{align*}
f(\alpha,\beta) = \alpha + \beta x + (\alpha-\beta)x^2.
\end{align*}
As in the previous example, we desire that the matrix $F$ maps the coordinates of an arbitrary vector $\mathbf{v}=(\alpha, \beta)$ to the coordinates of $f(\mathbf{v})$. The coordinates depend on the choices of bases for the domain and codomain vector spaces. Let's take the canonical bases
\begin{align*}
\mathcal{A} =& \{ \mathbf{e}_x, \, \mathbf{e}_y \} = \{(1,0),\, (0,1) \} \\
\mathcal{B} =& \{ \mathbf{p}_0, \, \mathbf{p}_1, \, \mathbf{p}_2\} = \{1,\, x, \, x^2 \}
\end{align*}
so that
\begin{align*}
F[(\alpha, \beta)]_\mathcal{A} = [f(\alpha, \beta)]_\mathcal{B} 
%
\quad \implies \quad 
%
F
\begin{pmatrix}
\alpha \\ \beta
\end{pmatrix}
&=
[\alpha + \beta x + (\alpha-\beta)x^2]_\mathcal{B} 
\\
&=
\begin{pmatrix}
\alpha \\ \beta \\ \alpha-\beta
\end{pmatrix}\\
&=
\begin{pmatrix}
1 \\ 0 \\ 1
\end{pmatrix}
\alpha
+
\begin{pmatrix}
0 \\ 1 \\ -1
\end{pmatrix}
\beta.
\end{align*}
We then use this last term to define a matrix
\begin{align*}
\begin{pmatrix}
1 \\ 0 \\ 1
\end{pmatrix}
\alpha
+
\begin{pmatrix}
0 \\ 1 \\ -1
\end{pmatrix}
\beta.
=
\begin{pmatrix}
 1 &  0 \\
 0 &  1 \\
 1 & -1
\end{pmatrix}
\begin{pmatrix}
\alpha \\ \beta
\end{pmatrix}
.
\end{align*}
So we see that we have a matrix with 3 rows and 2 columns
\begin{align*}
F
\begin{pmatrix}
\alpha \\ \beta
\end{pmatrix}
=
\begin{pmatrix}
 1 &  0 \\
 0 &  1 \\
 1 & -1
\end{pmatrix}
\begin{pmatrix}
\alpha \\ \beta
\end{pmatrix}
\quad \implies \quad
F=
\begin{pmatrix}
 1 &  0 \\
 0 &  1 \\
 1 & -1
\end{pmatrix}.
\end{align*}
}

\noindent So now that we perhaps can see that matrices can take any shape we may as well define them clearly.

\definition{Matrix}{
A matrix, denoted $A$, is an array of numbers organised into rows and columns. 
\begin{align*}
A = 
\begin{pmatrix}
a_{11} & \cdots & a_{1j} & \cdots & a_{1n}   \\
\vdots & \ddots & \vdots &        & \vdots    \\
a_{i1} & \cdots & a_{ij} & \cdots & a_{in}   \\
\vdots &        & \vdots & \ddots & \vdots    \\
a_{m1} & \cdots & a_{mj} & \cdots & a_{mn} 
\end{pmatrix} \\
\end{align*}
We say $A$ is an $m \times n$ matrix if it has $m$ rows and $n$ columns. The numbers in the array are called coefficients or elements of $A$, and are often denoted by their indices
\begin{align*}
a_{ij} = \left(A\right)_{ij} = a_{i,j}.
\end{align*}
}

\noindent In the previous examples, there was a key step in the creation of the matrices, when we claimed
\begin{align*}
\begin{pmatrix}
\alpha \\ \gamma
\end{pmatrix}
x
+
\begin{pmatrix}
\beta \\ \delta
\end{pmatrix}
y
=
\begin{pmatrix}
\alpha &\beta \\ 
\delta & \gamma
\end{pmatrix}
\begin{pmatrix}
x \\ y
\end{pmatrix}
%%%%%
%%%%%
\quad \text{or} \quad
%%%%%
%%%%%
\begin{pmatrix}
\alpha \\ \gamma \\ \epsilon
\end{pmatrix}
x
+
\begin{pmatrix}
\beta \\ \delta \\ \zeta
\end{pmatrix}
y
=
\begin{pmatrix}
 \alpha   & \beta \\
 \gamma   & \delta  \\
 \epsilon & \zeta
\end{pmatrix}
\begin{pmatrix}
x \\ y
\end{pmatrix}
\end{align*}
\noindent We will simply reverse the direction of the equality and take this as a defining statement of how to multiply a matrix by a column:

\definition{Matrix multiplication by a column}{
The multiplication of an $m \times n$ matrix, $A$, by a column, $X$, is defined only if the column contains as many elements as the columns of $A$. It is given by
\begin{align*}
AX = 
\begin{pmatrix}
a_{11} & \cdots & a_{1j} & \cdots & a_{1n}   \\
\vdots & \cdots & \vdots & \cdots & \vdots    \\
a_{i1} & \cdots & a_{ij} & \cdots & a_{in}   \\
\vdots & \cdots & \vdots & \cdots & \vdots    \\
a_{m1} & \cdots & a_{mj} & \cdots & a_{mn} 
\end{pmatrix}
%%%
\begin{pmatrix}
x_1 \\ 
\vdots \\
x_n
\end{pmatrix}
%%%
&=
%%%
\begin{pmatrix}
a_{11} \\ 
\vdots \\
a_{j1} \\ 
\vdots \\
a_{m1}
\end{pmatrix}
x_1
+
\cdots
+
\begin{pmatrix}
a_{1n} \\ 
\vdots \\
a_{jn} \\ 
\vdots \\
a_{mn}
\end{pmatrix}
x_n \\
%%%
&=
%%%
\begin{pmatrix}
a_{11}x_1 + \cdots + a_{1j}x_j + \cdots + a_{1n}x_n   \\
\vdots  \\
a_{i1}x_1 + \cdots + a_{ij}x_j + \cdots + a_{in}x_n   \\
\vdots   \\
a_{m1}x_1 + \cdots + a_{mj}x_j + \cdots + a_{mn}x_n 
\end{pmatrix}
\end{align*}
This gives an expression for the $i^{th}$ element of the resulting column that you may have seen in other textbooks
\begin{align*}
(AX)_i = \sum_{j=1}^n a_{ij}x_j.
\end{align*}
}

\noindent Now let's be very general and consider an undefined linear map. Let $f\in \mathcal{L}(V,W)$, $\mathcal{A}=\{ \mathbf{a}_1, \dots , \mathbf{a}_n\}$ be a basis for the $n$-dimensional vector space $V$ and $\mathcal{B}=\{ \mathbf{b}_1, \dots , \mathbf{b}_m\}$ be a basis for the $m$-dimensional vector space $W$. Then let $F$ be the operation of $f$ that takes the coordinates of an arbitrary vector $\mathbf{v}\in V$ to the coordinates of $f(\mathbf{v}) \in W$
\begin{align*}
F[\mathbf{v}]_\mathcal{A} = [f(\mathbf{v})]_\mathcal{B}.
\end{align*}
As $\mathcal{A}$ is basis of $V$, the vector $\mathbf{v}$ can be expressed as a linear combination of these vectors and the coefficients give its coordinates in this basis
\begin{align*}
\exists \alpha_1, \dots, \alpha_n \in \mathbb{R}\quad\text{such that} \quad \mathbf{v} = \alpha_1 \mathbf{a}_1 + \cdots + \alpha_n \mathbf{a}_n \quad\implies\quad [\mathbf{v}]_\mathcal{A} = 
\begin{pmatrix}
\alpha_1 \\ \vdots \\ \alpha_n
\end{pmatrix}.
\end{align*}
Hence we have
\begin{align*}
F
\begin{pmatrix}
\alpha_1 \\ \vdots \\ \alpha_n
\end{pmatrix}
%
&=[f(\alpha_1 \mathbf{a}_1 + \alpha_2 \mathbf{a}_2 + \cdots + \alpha_n \mathbf{a}_n)]_\mathcal{B} \\
%
&= [\alpha_1 f(\mathbf{a}_1) + \alpha_2  f(\mathbf{a}_2) + \cdots + \alpha_n f(\mathbf{a}_n)]_\mathcal{B}  \\
%
&= \alpha_1 [f(\mathbf{a}_1)]_\mathcal{B}  + \alpha_2 [f(\mathbf{a}_2)]_\mathcal{B} + \cdots + \alpha_n [f(\mathbf{a}_n)]_\mathcal{B} .
\end{align*}
If we remember that the $[f(\mathbf{a}_k)]_\mathcal{B}$ are columns, then this last line gives us the form  of a matrix multiplied by a column
\begin{align*}
F
\begin{pmatrix}
\alpha_1 \\ \vdots \\ \alpha_n
\end{pmatrix}
=
\alpha_1 [f(\mathbf{a}_1)]_\mathcal{B}  + \cdots + \alpha_n [f(\mathbf{a}_n)]_\mathcal{B} = 
\begin{pmatrix}
| & | & & | \\
[f(\mathbf{a}_1)]_\mathcal{B} & [f(\mathbf{a}_2)]_\mathcal{B} & \dots & [f(\mathbf{a}_n)]_\mathcal{B}\\
| & | & & | 
\end{pmatrix}
\begin{pmatrix}
\alpha_1 \\ \vdots \\ \alpha_n
\end{pmatrix}
\end{align*}
This gives us a general method of finding the matrix representation of any linear map. Let's formalise what we did first with a definition:



\definition{Matrix representation of a linear map}{
Let $f:V \to W$ be a linear map, $\mathcal{A}$ be a basis of $V$ and $\mathcal{B}$ be a basis of $W$. The \textit{matrix representation} of $f$ in bases $\mathcal{A}$ and $\mathcal{B}$ is defined by the operation
\begin{align*}
\mathcal{M}(f,\mathcal{A}\to\mathcal{B})[\mathbf{v}]_\mathcal{A} = [f(\mathbf{v})]_\mathcal{B}
\end{align*}
for every $\mathbf{v}$ in $V$. In words, the matrix takes the coordinates of $\mathbf{v}$ to the coordinates of $f(\mathbf{v})$ in their respective bases.
}

\noindent Now we have already proven the following theorem before expressing it:

\theorem{Matrix representation of a linear map}{
Let $f:V \to W$ be a linear map, $\mathcal{A}=\{\mathbf{a}_1,\dots,\mathbf{a}_n\}$ be a basis of $V$ and $\mathcal{B}=\{\mathbf{b}_1,\dots,\mathbf{b}_m\}$ be a basis of $W$. Then the \textit{matrix representation} of $f$ in bases $\mathcal{A}$ and $\mathcal{B}$ is a unique $m\times n$ matrix calculated by expressing the coordinates of the linear map acting on the basis vectors of the input space as columns
\begin{align*}
\mathcal{M}(f,\mathcal{A}\to\mathcal{B})=
\begin{pmatrix}
| &  & | \\
[f(\mathbf{a}_1)]_\mathcal{B} & \dots & [f(\mathbf{a}_n)]_\mathcal{B}\\
| &  & | 
\end{pmatrix}
\end{align*}
where the vertical lines are reminders that the coordinates of the $f(\mathbf{a}_k)$ vectors are columns. We often shorten ``matrix representation of $f$'' to just ``matrix of $f$''. If the input and output vector spaces are the same, i.e. if $f$ is an endomorphism, we can use the same basis for both spaces and we may shorten the notation: $\mathcal{M}(f,\mathcal{A}\to\mathcal{A}) = \mathcal{M}(f,\mathcal{A})$.
}

\example{Matrix of a linear map in canonical bases}{

\noindent Consider a linear map $f:\mathbb{R}^3 \to \mathbb{R}^2$ defined by 
\begin{align*}
f(x,y,z) = (x-z, 2y+3z)
\end{align*}
and canonical bases $\mathcal{C}_3$ and $\mathcal{C}_2$ of $\mathbb{R}^3$ and $\mathbb{R}^2$, respectively. What is the matrix representation of $f$ in the canonical bases? \\

\noindent To build the matrix of $f$ we use $f$ on the basis vectors of the domain, $\mathcal{C}_3=\{(1,0,0), \, (0,1,0), \, (0,0,1)\}$, and find the coordinates of the results in the given basis of the codomain, $\mathcal{C}_2=\{(1,0), \, (0,1)\}$. That is, we want the $2 \times 3$ matrix:
\begin{align*}
\mathcal{M}(f,\mathcal{C}_3\to\mathcal{C}_2)=
\begin{pmatrix}
| & | & | \\
[f(1,0,0)]_{\mathcal{C}_2} & [f(0,1,0)]_{\mathcal{C}_2} & [f(0,0,1)]_{\mathcal{C}_2}\\
| & | & | 
\end{pmatrix}
\end{align*}
So we have
\begin{align*}
f(1,0,0) &= (1,0) \quad\implies\quad [f(1,0,0)]_{\mathcal{C}_2} = \begin{pmatrix} 1 \\ 0 \end{pmatrix} \\
%
f(0,1,0) &= (0,2) \quad\implies\quad [f(0,1,0)]_{\mathcal{C}_2} = \begin{pmatrix} 0 \\ 2 \end{pmatrix} \\
%
f(0,0,1) &= (-1,3) \quad\implies\quad [f(0,0,1)]_{\mathcal{C}_2} = \begin{pmatrix} -1 \\ 3 \end{pmatrix}.
\end{align*}
These coordinates form the columns of the matrix of $f$:
\begin{align*}
\mathcal{M}(f,\mathcal{C}_3\to\mathcal{C}_2)=
\begin{pmatrix}
1 & 0 & -1 \\
0 & 2 & 3
\end{pmatrix}.
\end{align*}
We can illustrate the usage of the matrix by showing that matrix multiplication by a coordinate column gives the same answer as applying the linear map directly
\begin{gather*}
f(1,2,3) = (-2, 13) = -2(1,0) + 13(0,1) \quad\implies\quad [f(1,2,3)]_{\mathcal{C}_2} = 
\begin{pmatrix}
-2 \\ 13
\end{pmatrix}
\end{gather*}
and
\begin{align*}
\mathcal{M}(f,\mathcal{C}_3\to\mathcal{C}_2) [(1,2,3)]_{\mathcal{C}_3} &=
\begin{pmatrix}
1 & 0 & -1 \\
0 & 2 & 3
\end{pmatrix}
\begin{pmatrix}
1 \\ 2 \\  3
\end{pmatrix}
\\
&=
1 \begin{pmatrix}  1 \\ 0 \end{pmatrix}
+
2 \begin{pmatrix}  0 \\ 2 \end{pmatrix}
+
3 \begin{pmatrix} -1 \\ 3 \end{pmatrix}\\
&=
\begin{pmatrix} -2 \\ 13 \end{pmatrix}\\
&=
[(-2,13)]_{\mathcal{C}_2}
\end{align*}
}


\example{Matrix of a linear map in non-canonical bases}{

\noindent Consider the same linear map as the previous example but instead let's use bases 
\begin{align*}
\mathcal{A}=\{(1,0,1), \, (0,1,1), \, (1,0,-1)\} \quad\text{and}\quad \mathcal{B}=\{(1,1), \, (0,1)\}
\end{align*} 
of $\mathbb{R}^3$ and $\mathbb{R}^2$, respectively. What is the matrix representation of that linear map in these new bases? \\

\noindent Again, we must use $f$ on the basis vectors of the domain and find the coordinates of the results in the given basis of the codomain:
\begin{align*}
f(1,0,1) &= (0,3) =0(1,1) + 3(0,1) \quad\implies\quad [f(1,0,0)]_{\mathcal{B}} = \begin{pmatrix} 0 \\ 3 \end{pmatrix} \\
%
f(0,1,1) &= (-1,5) =-1(1,1) + 6(0,1) \quad\implies\quad [f(0,1,0)]_{\mathcal{B}} = \begin{pmatrix} -1 \\ 6 \end{pmatrix} \\
%
f(1,0,-1) &= (2,-3) =2(1,1) -5(0,1) \quad\implies\quad [f(0,0,1)]_{\mathcal{B}} = \begin{pmatrix} 2 \\ -5 \end{pmatrix}.
\end{align*}
Hence the matrix of $f$ in these bases is
\begin{align*}
\mathcal{M}(f,\mathcal{A}\to\mathcal{B})=
\begin{pmatrix}
0 & -1 & 2 \\
3 &  6 & 5
\end{pmatrix}.
\end{align*}
Now let's check that matrix multiplication by a coordinate column gives the same answer as applying the linear map directly
\begin{gather*}
f(1,2,3) = (-2, 13) = -2(1,1) + 15(0,1) \quad\implies\quad [f(1,2,3)]_{\mathcal{B}} = 
\begin{pmatrix}
-2 \\ 15
\end{pmatrix} 
%
\\ \quad\text{and}\quad \\
%
(1,2,3) = 1(1,0,1) + 2(0,1,1) + 0(1,0,-1)\quad\implies\quad [(1,2,3)]_{\mathcal{A}} = 
\begin{pmatrix} 1 \\ 2 \\ 0 \end{pmatrix} 
%
\\ \quad\text{so}\quad \\
%
\begin{pmatrix}
0 & -1 & 2 \\
3 &  6 & 5
\end{pmatrix}
\begin{pmatrix}
1 \\ 2 \\  0
\end{pmatrix}
=
\begin{pmatrix}
-2 \\ 15
\end{pmatrix}.
\end{gather*}
}

\example{Matrix of differentiation of polynomials}{

\noindent Let $\mathcal{C}_n$ and $\mathcal{C}_{n-1}$ be the canonical bases of the vector spaces of polynomials with degree up to $n$, $\mathcal{P}_n$, and up to $n-1$, $\mathcal{P}_{n-1}$, respectively. What is the matrix representing the differentation linear map $D:\mathcal{P}_n \to \mathcal{P}_{n-1}$ which maps a polynomial to its derivative? \\

\noindent We consider the linear map on the basis vectors of the domain:
\begin{align*}
D(1) &= 0 \\
D(x) &= 1 \\
D(x^2) &= 2x \\
&\vdots \\
D(x^n) &= nx^{n-1}
\end{align*}
The pattern is quite clear I hope. If we write the coordinates of these resulting vectors generally we have
\begin{align*}
[D(x^k)]_{\mathcal{C}_{n-1}} = 
\begin{pmatrix} 
\alpha_0 \\ 
\alpha_1 \\ 
\vdots \\ 
\alpha_{n-1}
\end{pmatrix}
\end{align*}
%\begin{align*}
%[D(x^k)]_{\mathcal{C}_{n-1}} = 
%\begin{pmatrix} 
%0 \\ 
%\vdots \\ 
%k \\ 
%\vdots \\ 
%0 \end{pmatrix}
%\leftarrow \text{in the $(k-1)^{th}$ position}
%\end{align*}
where
\begin{align*}
\alpha_j = \begin{cases}
k & \text{for } j=k-1\\
0 & \text{for } j\neq k-1
\end{cases}
\end{align*}
The matrix representation of $D$ is therefore the following matrix with $n$ columns and $n-1$ rows
\begin{align*}
\mathcal{M}(D,\mathcal{C}_n\to\mathcal{C}_{n-1})
=
\begin{pmatrix}
| & | &  & | \\
[D(1)]_{\mathcal{C}_{n-1}} & [D(x)]_{\mathcal{C}_{n-1}} & \dots & [D(x^n)]_{\mathcal{C}_{n-1}} \\
| & | &  & | 
\end{pmatrix}
=
\begin{pmatrix}
  0    &   1    &   0    &   0    & \cdots &   0 \\
  0    &   0    &   2    &   0    & \cdots &   0 \\
  0    &   0    &   0    &   3    & \cdots &   0 \\
  0    &   0    &   0    &   0    & \cdots &   0 \\
\vdots & \vdots & \vdots & \vdots & \ddots & \vdots \\
  0    &   0    &   0    & \cdots &   n-1  &   0 \\
  0    &   0    &   0    & \cdots &   0    &   n \\
\end{pmatrix}
\end{align*}
For example, if we want to use this matrix to differentiate the polynomial $2 - x^2 + 3x^3$ we perform the following calculation
\begin{align*}
\mathcal{M}(D,\mathcal{C}_3\to\mathcal{C}_{2}) [2 - x^2 + 3x^3]_{\mathcal{C}_3}
&=
\begin{pmatrix}
  0  &  1  &  0  &  0\\
  0  &  0  &  2  &  0\\
  0  &  0  &  0  &  3  
\end{pmatrix}
\begin{pmatrix}
2 \\ 0 \\ -1 \\ 3
\end{pmatrix} \\
&=
\begin{pmatrix}
0 \\ -2 \\ 9 
\end{pmatrix} \\
&=
[-2x + 9x^2]_{\mathcal{C}_2}
\end{align*}
}

\theorem{Identity map and identity matrix}{
The identity map is the endomorphism $f_I: V \to V$ for any $n$-dimensional vector space $V$, defined by
\begin{align*}
f_I(\mathbf{v}) = \mathbf{v} \quad \forall \, \mathbf{v}\in V.
\end{align*}
Let $\mathcal{B} = \{ \mathbf{b}_1, \dots, \mathbf{b}_n \}$ be some basis of $V$. Then the matrix of $f_I$ in this basis for both the input and output space is given by
\begin{align*}
\mathcal{M}(f_I,\mathcal{B})
=
\begin{pmatrix}
| &  & | \\
[f_I(\mathbf{b}_1)]_\mathcal{B} & \dots & [f_I(\mathbf{b}_n)]_\mathcal{B}\\
| &  & | 
\end{pmatrix}
=
\begin{pmatrix}
| &  & | \\
[\mathbf{b}_1]_\mathcal{B} & \dots & [\mathbf{b}_n]_\mathcal{B}\\
| &  & | 
\end{pmatrix}
=
\begin{pmatrix}
1      & 0      & \cdots & 0 \\
0      & 1      & \cdots & 0 \\
\vdots & \vdots & \ddots & \vdots \\
0      &  0     & \cdots & 1 
\end{pmatrix}
\end{align*}
This matrix is called the identity matrix of size $n$ and is denoted $I_n$ or just $I$ when the context is clear.
}

\theorem{Map composition and matrix multiplication}{
\begin{align*}
F[\mathbf{v}]_\mathcal{A} = [f(\mathbf{v})]_\mathcal{B}, \quad
G[\mathbf{w}]_\mathcal{B} = [g(\mathbf{w})]_\mathcal{C}, \quad 
M[\mathbf{v}]_\mathcal{A} = [g\circ f(\mathbf{v})]_\mathcal{C} \quad \implies \quad M = GF
\end{align*}
}

\theorem{Inverse linear map and matrix inversion}{
\begin{align*}
F[\mathbf{v}]_\mathcal{A} = [\mathbf{w}]_\mathcal{B}, \quad
G[\mathbf{w}]_\mathcal{B} = [\mathbf{v}]_\mathcal{A}, \quad 
\implies \quad G = F^{-1}
\end{align*}
}


%%%%%%%%%%%%%%%%%%%%
%%%%%%%%%%%%%%%%%%%%
%%%%%%%%%%%%%%%%%%%%
\section{Transition matrices}


\definition{Transition matrix (change-of-basis matrix)}{
The \textit{transition matrix} changes the representation of the coordinates of a vector from one basis into another. Let $\mathcal{A}$ and $\mathcal{B}$ be two bases of the same vector space, $V$, and let $\mathbf{v} \in V$. The transition matrix from $\mathcal{A}$ to $\mathcal{B}$, denoted $P_{\mathcal{A}\to \mathcal{B}}$ is defined by the relation
\begin{align*}
P_{\mathcal{A}\to \mathcal{B}} [\mathbf{v}]_\mathcal{A} = [\mathbf{v}]_\mathcal{B}.
\end{align*}
}





\theorem{Transition matrix (change-of-basis matrix)}{
Let $\mathcal{A}=\{\mathbf{a}_1,\dots,\mathbf{a}_n\}$ and $\mathcal{B}=\{\mathbf{b}_1,\dots,\mathbf{b}_m\}$ be two bases of a vector space $V$. Then transition matrix from $\mathcal{A}$ to $\mathcal{B}$ can be calculated by
\begin{align*}
P_{\mathcal{A}\to\mathcal{B}} =
\begin{pmatrix}
| & | & & | \\
[\mathbf{a}_1]_\mathcal{B} & [\mathbf{a}_2]_\mathcal{B} & \dots & [\mathbf{a}_n]_\mathcal{B}\\
| & | & & | 
\end{pmatrix}
\end{align*}
where the vertical lines are reminders that the coordinates of the $A$ basis vectors are columns.
}

\theorem{Transition matrix is the matrix representation of the identity map}{
For a vector space $V$ with two bases $\mathcal{A}$ and $\mathcal{B}$, the transition matrix from $\mathcal{A}$ to $\mathcal{B}$ is exactly the matrix representing the identity map of $V$, $I_V$, in bases $\mathcal{A}$ and $\mathcal{B}$. That is,
\begin{align*}
P_{\mathcal{A}\to \mathcal{B}} = \mathcal{M}(I_V,\mathcal{A}\to \mathcal{B}).
\end{align*}
}

\begin{proof}
This one is pretty simple. Let $\mathcal{A} = \left\{ \mathbf{a}_1, \dots, \mathbf{a}_n \right\}$. Then
\begin{align*}
\mathcal{M}(I_V,\mathcal{A}\to \mathcal{B})
&=
\begin{pmatrix}
| & | & & | \\
[I_V(\mathbf{a}_1)]_\mathcal{B} & [I_V(\mathbf{a}_2)]_\mathcal{B} & \dots & [I_V(\mathbf{a}_n)]_\mathcal{B}\\
| & | & & | 
\end{pmatrix} 
&=
\begin{pmatrix}
| & | & & | \\
[\mathbf{a}_1]_\mathcal{B} & [\mathbf{a}_2]_\mathcal{B} & \dots & [\mathbf{a}_n]_\mathcal{B}\\
| & | & & | 
\end{pmatrix}
=
P_{\mathcal{A}\to\mathcal{B}}
\end{align*}
\end{proof}

\example{Transition matrix in $\mathbb{R}^2$}{
\noindent Find the transition matrix from the canonical basis of $\mathbb{R}^2$, $\mathcal{C}_2$, to the basis $\mathcal{B}=\{ (2,1), \, (1,3) \}$. Use this matrix to find the coordinates of $(-2,3)$ in the basis $\mathcal{B}$. \\

\noindent Let $\mathbf{b}_1 = (2,1)$ and $\mathbf{b}_2 = (1,3)$. Then the transition matrix $P_{\mathcal{C}_2\to\mathcal{B}}$ has columns $[(1,0)]_\mathcal{B}$ and $[(0,1)]_\mathcal{B}$. For the first, we have
\begin{align*}
(1,0) &= \alpha \mathbf{b}_1 + \beta \mathbf{b}_2 \\
\implies & 
\begin{cases}
 2\alpha + \beta = 1 \\
 \alpha + 3\beta = 0
\end{cases} \\
\implies &
\begin{cases}
 2\alpha + \beta = 1 \\
 \alpha  = -3\beta
\end{cases} \\
\implies &
\begin{cases}
 -6\beta + \beta = 1 \\
 \alpha  = -3\beta
\end{cases} \\
\implies &
\begin{cases}
 \beta = -\dfrac{1}{5} \\
 \alpha  = \dfrac{3}{5}
\end{cases}
\end{align*}
Hence 
\begin{align*}
[(1,0)]_\mathcal{B}
=
\begin{pmatrix}
3/5 \\
-1/5
\end{pmatrix}
\end{align*}
With similar working we find
\begin{align*}
[(0,1)]_\mathcal{B}
=
\begin{pmatrix}
-1/5\\
2/5
\end{pmatrix}
\end{align*}
and so the transition matrix is given by
\begin{align*}
P_{\mathcal{C}_2\to\mathcal{B}}
=
\begin{pmatrix}
3/5  & -1/5 \\
-1/5 & 2/5
\end{pmatrix}
\end{align*}
Now we use this matrix to calculate $[(-2,3)]_\mathcal{B}$:
\begin{align*}
[(-2,3)]_\mathcal{B} = P_{\mathcal{C}_2\to\mathcal{B}}[(-2,3)]_{\mathcal{C}_2}
= 
\begin{pmatrix}
3/5  & -1/5 \\
-1/5 & 2/5
\end{pmatrix}
\begin{pmatrix}
-2 \\
3
\end{pmatrix}
=
\begin{pmatrix}
-9/5 \\
8/5
\end{pmatrix}
\end{align*}
}



\theorem{Reverse transition matrix is the inverse}{
For a vector space $V$ with two bases $\mathcal{A}$ and $\mathcal{B}$, the transition matrix from $\mathcal{B}$ to $\mathcal{A}$ is the inverse of the transition matrix from $\mathcal{A}$ to $\mathcal{B}$
\begin{align*}
P_{ \mathcal{B}\to \mathcal{A}} = P_{\mathcal{A}\to \mathcal{B}}^{-1}.
\end{align*}
}


\theorem{Changing the bases of a matrix representation}{ \label{thm:map_diff_bases}
Let $f:U \to V$ be a linear map, $\mathcal{B}_U$ and $\mathcal{B}'_U$ be two bases of $U$, $\mathcal{B}_V$ and $\mathcal{B}'_V$ be two bases of $V$, and $F=\mathcal{M}(f,\mathcal{B}_U\to\mathcal{B}_V)$ be the matrix representation of $f$ from basis $\mathcal{B}_U$ to basis $\mathcal{B}_V$. 

Then $F'=\mathcal{M}(f,\mathcal{B}'_U\to\mathcal{B}'_V)$, the matrix representation of $f$ from basis $\mathcal{B}'_U$ to basis $\mathcal{B}'_V$, is given by
\begin{align*}
F' = P_{\mathcal{B}_V\to\mathcal{B}_V'} \, F \, P_{\mathcal{B}'_U\to\mathcal{B}_U}
\end{align*}

The following diagram may help visualise this relation
\begin{figure}[H]
\centering
\begin{tikzpicture}
	\coordinate (U)  at (-2.5,+2);
	\coordinate (Ud) at (-2.5,-2);
	\coordinate (V)  at (+2.5,+2);
	\coordinate (Vd) at (+2.5,-2);
	
	\coordinate (F)  at (+0,+2.7);
	\coordinate (Fd) at (+0,-2.7);
	\coordinate (P1) at (-4,+0);
	\coordinate (P2) at (+4,+0);
	
	\node[black,scale=2.5] at (U)  {$\mathcal{B}_U$};
	\node[black,scale=2.5] at (Ud) {$\mathcal{B}'_U$};
	\node[black,scale=2.5] at (V)  {$\mathcal{B}_V$};
	\node[black,scale=2.5] at (Vd) {$\mathcal{B}'_V$};
	
	\node[black,scale=2.5] at (F)  {$F$};
	\node[black,scale=2.5] at (Fd) {$F'$};
	\node[black,scale=2] at (P1) {$P_{\mathcal{B}'_U\to\mathcal{B}_U}$};
	\node[black,scale=2] at (P2) {$P_{\mathcal{B}_V\to\mathcal{B}_V'}$};
	
	\draw[->,ultra thick] ($(U)+(1,0)$)--($(V)-(1,0)$);
	\draw[->,ultra thick] ($(Ud)+(1,0)$)--($(Vd)-(1,0)$);
	\draw[->,ultra thick] ($(Ud)+(0,0.8)$)--($(U)-(0,0.8)$);
	\draw[->,ultra thick] ($(V)-(0,0.8)$)--($(Vd)+(0,0.8)$);
\end{tikzpicture}
\end{figure}
To read this schematic, consider that the arrow for $F'$ has the same input and output as following the other three arrows to go up, then right (through $F$) then down again. This ordered path is the matrix multiplication given above.
}

\example{Changing the bases of a matrix representation}{

\noindent Consider the linear map $f:\mathbb{R}^2 \to \mathcal{P}_2$ given by
\begin{align*}
f(\alpha,\beta) = \beta + \alpha x^2.
\end{align*}
What is the matrix representation of $f$ using the canonical bases of $\mathbb{R}^2$, call it $\mathcal{A}$, and $\mathcal{P}_2$, call it $\mathcal{B}$? Consider the two bases $\mathcal{A}'=\{(-1,1),\, (1,2)\}$ and $\mathcal{B}'=\{ 2, \,1- x, \, 3 + x^2 \}$. Use transition matrices to find the matrix representation of $f$ in these new bases: $\mathcal{M}(f,\mathcal{A}'\to\mathcal{B}')$. \\

\noindent For the matrix in the canonical bases, we use the linear map on the basis vectors:
\begin{align*}
f(1,0) &= x^2 \implies [f(1,0)]_{\mathcal{B}} = \begin{pmatrix} 0 \\ 0 \\ 1\end{pmatrix} \\
f(0,1) &= 1 \implies [f(0,1)]_{\mathcal{B}} = \begin{pmatrix} 1 \\ 0 \\ 0\end{pmatrix}
\end{align*}
and so that matrix of $f$ in the canonical bases is given by
\begin{align*}
\mathcal{M}(f,\mathcal{A}\to\mathcal{B})
=
\begin{pmatrix} 
0 & 1 \\
0 & 0 \\
1 & 0
\end{pmatrix}
\end{align*}
We need transition matrices $P_{\mathcal{A}'\to\mathcal{A}}$ and $P_{\mathcal{B}\to\mathcal{B}'}$. The first is really simple
\begin{align*}
(-1,1) &= -1(1,0) + 1(0,1) \implies[(1,1)]_{\mathcal{A}} = \begin{pmatrix} -1 \\ 1 \end{pmatrix} \\
(1,2) &= 1(1,0) + 2(0,1) \implies[(0,2)]_{\mathcal{A}} = \begin{pmatrix} 1 \\ 2 \end{pmatrix} \\
\implies P_{\mathcal{A}'\to\mathcal{A}}
&=
\begin{pmatrix} 
 -1 & 1\\ 
  1 & 2
\end{pmatrix}
\end{align*}
The second transition matrix, from $\mathcal{B}=\{1,\,x,\,x^2\}$ to $\mathcal{B}'=\{2,\,1-x,\,3+x^2\}$, requires a bit more effort
\begin{align*}
1 &= \alpha(2) + \beta(1-x) + \gamma(3+x^2) \\
\implies &
\begin{cases}
2\alpha + \beta + 3\gamma = 1 \\
-\beta = 0 \\
\gamma = 0
\end{cases} \\
\implies & \alpha = 1/2 \\
\implies & [1]_{\mathcal{B}'} = \begin{pmatrix} 1/2 \\ 0 \\ 0 \end{pmatrix} \\ \\
%%
%%
x &= \alpha(2) + \beta(1-x) + \gamma(3+x^2) \\
\implies &
\begin{cases}
2\alpha + \beta + 3\gamma = 0 \\
-\beta = 1 \\
\gamma = 0
\end{cases} \\
\implies & \alpha = 0 \\
\implies & [x]_{\mathcal{B}'} = \begin{pmatrix} 0 \\ -1 \\ 0 \end{pmatrix} \\ \\
%%
%%
x^2 &= \alpha(2) + \beta(1-x) + \gamma(3+x^2) \\
\implies &
\begin{cases}
2\alpha + \beta + 3\gamma = 0 \\
-\beta = 0 \\
\gamma = 1
\end{cases} \\
\implies & \alpha = -3/2 \\
\implies & [x^2]_{\mathcal{B}'} = \begin{pmatrix} -3/2 \\ 0 \\ 1 \end{pmatrix}
\end{align*}
so we have transition matrix
\begin{align*}
P_{\mathcal{B}\to\mathcal{B}'}
&=
\begin{pmatrix} 
  1/2 & 0 & -3/2 \\
  0 & -1 & 0 \\
  0 & 0 & 1
\end{pmatrix}
\end{align*}
Now we can use these transition matrices to compute the matrix representation of $f$ in these alternate bases:
\begin{align*}
F' &= P_{\mathcal{B}\to\mathcal{B}'} \, F \, P_{\mathcal{A}'\to\mathcal{A}} \\
%%%
%%%
%%%
&= \begin{pmatrix} 
  1/2 & 0 & -3/2 \\
  0 & -1 & 0 \\
  0 & 0 & 1
\end{pmatrix}
\begin{pmatrix} 
  0 & 1 \\
  0 & 0 \\
  1 & 0
\end{pmatrix}
\begin{pmatrix} 
 -1 & 1 \\ 
  1 & 2
\end{pmatrix} \\
%%%
%%%
%%%
&=
\begin{pmatrix} 
  2 & -1/2 \\
  0 & 0 \\
 -1 & 1
\end{pmatrix}
\end{align*}
}