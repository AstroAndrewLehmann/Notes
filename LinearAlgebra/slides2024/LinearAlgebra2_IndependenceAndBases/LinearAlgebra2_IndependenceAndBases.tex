\documentclass[usenames,dvipsnames,aspectratio=169,10pt]{beamer}
%\usetheme{default}
\usetheme[progressbar=frametitle]{metropolis}

\def \EXAMPLEVERSION {3} % 1 for examples, 2 to hide examples (they are in textbook), 3 to hide examples but leave blank slide

\def \SCHOOLVERSION {2} % 1 for neutral, 2 for ISEP

%\documentclass[12pt]{book}
\usepackage{amsfonts}
\usepackage{amsmath}
\usepackage{amssymb}
\usepackage{graphicx}
\usepackage[authoryear]{natbib}
%\usepackage[margin=2.5cm]{geometry}
%\usepackage{hyperref}
\usepackage[font=footnotesize]{caption}
\usepackage{float}
\usepackage{caption}
\usepackage{subcaption}
\usepackage{setspace}
\usepackage{cleveref}
\usepackage{lscape}
\usepackage{multirow}
\usepackage{nicematrix}
\usepackage{mathtools}

% for tikz
\usepackage{tikz}
\usetikzlibrary{angles, arrows.meta, calc, quotes}
\usetikzlibrary{decorations.pathreplacing,calligraphy}
\usetikzlibrary{patterns}
\usetikzlibrary{bending,matrix,positioning}
\usetikzlibrary{arrows, fit, shapes, backgrounds}

\usepackage{tikz-3dplot}
\usepackage{xcolor,colortbl}


% for red line canceling diagonally
\usepackage{cancel}
\renewcommand{\CancelColor}{\color{red}}

\captionsetup{font=small,labelfont=bf,singlelinecheck=off,margin=2cm,justification=justified}
\numberwithin{equation}{section}

\newcommand{\defbox}[3]
	{
		\vspace{0.5cm} 
		\noindent \fbox{\begin{minipage}{\linewidth}
		\textbf{{#1}DEFINITION{#2}} #3
		\end{minipage}}
		\vspace{0.5cm}
	}
	
	
\newcommand{\defnobox}[3]
	{
		\vspace{0.5cm} 
		\noindent \begin{minipage}{\linewidth}
		\textbf{{#1}DEFINITION{#2}} #3
		\end{minipage}
		\vspace{0.5cm}
	}
	
%%%% To get a nice colourful box around an equation
\newcommand*{\colourboxed}{}
\def\colourboxed#1#{%
  \colourboxedAux{#1}%
}
\newcommand*{\colourboxedAux}[3]{%
  % #1: optional argument for color model
  % #2: color specification
  % #3: formula
  \begingroup
    \colorlet{cb@saved}{.}%
    \color#1{#2}%
    \boxed{%
      \color{cb@saved}%
      #3%
    }%
  \endgroup
}


%%%%%% COLOURS %%%%%%%%%%
\definecolor{airforceblue}{rgb}{0.36, 0.54, 0.66}
\definecolor{battleshipgrey}{rgb}{0.52, 0.52, 0.51}
\definecolor{brightmaroon}{rgb}{0.76, 0.13, 0.28}
\definecolor{nicegreen}{RGB}{133, 204, 111}

% \textcolor{MidnightBlue}{}
% \textcolor{Purple}{}
% \textcolor{BurntOrange}{}
% \textcolor{Maroon}{}
% \textcolor{Mahogany}{}

% isep colours
\definecolor{isepblue1}{RGB}{0, 97, 161}      % teinte à 100%
\definecolor{isepblue2}{RGB}{77, 144, 189}    % teinte à 70%
\definecolor{isepblue3}{RGB}{179, 208, 227}   % teinte à 30%
\definecolor{iseporange1}{RGB}{244, 161, 0}   % teinte à 100%
\definecolor{iseporange2}{RGB}{234, 189, 100} % teinte à 70%
\definecolor{iseporange3}{RGB}{252, 227, 179} % teinte à 30%
%%%%%%%%%%%%%%%%

% beamer stuff
\setbeamertemplate{navigation symbols}{}
\setbeamersize{text margin left=1.0cm,text margin right=1.0cm}
\setbeamercolor{background canvas}{bg=white}


\ifnum \SCHOOLVERSION = 2
	%%%% ISEP COLOURS %%%%
	\setbeamercolor{frametitle}{bg=isepblue1, fg=white}
	\setbeamercolor{progress bar}{fg=iseporange1}
	\setbeamercolor{itemize item}{fg=iseporange1,bg=iseporange1}
	%%%%%%%%%%%%%%%%%%%%%%
\fi

\setbeamerfont{frametitle}{family=\fontfamily{qag}\selectfont} % choose font for frame titles
\setbeamerfont{title}{family=\fontfamily{qag}\selectfont} % choose font for title
\setbeamerfont{subtitle}{family=\fontfamily{qag}\selectfont} % choose font for subtitle
\setbeamerfont{section title}{family=\fontfamily{qag}\selectfont} % choose font for titles
%\fontfamily{qag}\selectfont %choose font for main text % put after begin{document}

% to make the progress bar a little thicker
\makeatletter
\setlength{\metropolis@titleseparator@linewidth}{1.5pt}
\setlength{\metropolis@progressonsectionpage@linewidth}{1.5pt}
\setlength{\metropolis@progressinheadfoot@linewidth}{1.5pt}
\makeatother

\begin{document}

\title{Linear Algebra}
\subtitle{Linear Algebra - Vector Spaces}
\author{Andrew Lehmann}
\ifnum \SCHOOLVERSION = 2
	\institute{\'{E}cole d'ing\'{e}nieurs du num\'{e}rique}
\fi
\date{\textit{Last updated: \today}}

% logo of university
\ifnum \SCHOOLVERSION = 2
	\titlegraphic{\includegraphics[width=3cm]{/home/andrew/Dropbox/ISEP/admin/logo-isep-2023.png} }
\fi


\begin{frame}
\titlepage
\end{frame}




\section{Linear Dependence and Independence}


%%%%%%%%%%%%%%%%%%%%%%%%%%%%%%%%%%%%%%%%%%%%%%%%%%%%%%%%%%%%%%%%%%%%%%%%
\begin{frame}
\frametitle{Definition - Linear dependence}
A set of vectors $\{\mathbf{v}_1, \dots, \mathbf{v}_n\}$ from a vector space $V$ is said to be \textit{linearly dependent} if there exists a set of constants $\{ \alpha_1, \dots, \alpha_n \}$ \textit{not all zero} such that
\begin{align*}
\alpha_1 \mathbf{v}_1 + \cdots + \alpha_n \mathbf{v}_n = \mathbf{0}_V.
\end{align*}
\textit{Note}: the right hand side of the equation is the \textit{zero vector}, not the real number $0$.
\end{frame}
%%%%%%%%%%%%%%%%%%%%%%%%%%%%%%%%%%%%%%%%%%%%%%%%%%%%%%%%%%%%%%%%%%%%%%%%




\ifnum \EXAMPLEVERSION = 1
%%%%%%%%%%%%%%%%%%%%%%%%%%%%%%%%%%%%%%%%%%%%%%%%%%%%%%%%%%%%%%%%%%%%%%%%
\begin{frame}
\frametitle{Example - Linear dependence of two Euclidean vectors}

\noindent Let $\mathbf{u}=(1,2)$ and $\mathbf{v}=(10.2,20.4)$. Show that $\mathbf{u}$ and $\mathbf{v}$ are linearly dependent. \\

\noindent To show this we must demonstrate that there are two constants, $a$ and $b$, such that $a\mathbf{u} + b\mathbf{v} = \mathbf{0}$. If this was true, we would have
\begin{align*}
a(1,2) + b(10.2,20.4) = (0,0)  \quad \implies \quad (a + 10.2b,2a + 20.4b) = (0,0)
\end{align*}
giving the system of equations
$
\begin{cases}
a + 10.2b = 0 \\
2a + 20.4b = 0
\end{cases}
$

Both equations give the same relation between $a$ and $b$:
$
b = -\frac{a}{10.2}
$

We are free to choose an $a$, for example if $a=10.2$ then $b=-1$ and we have
$
10.2 \mathbf{u} - 1 \mathbf{v} = \mathbf{0}
$

showing that the two vectors are linearly dependent. This example show that the linear dependence of 2 vectors means that 1 vector is a scalar multiple of the other. In this example:
\begin{align*}
\mathbf{v} = 10.2 \mathbf{u}.
\end{align*}
\end{frame}
%%%%%%%%%%%%%%%%%%%%%%%%%%%%%%%%%%%%%%%%%%%%%%%%%%%%%%%%%%%%%%%%%%%%%%%%
\fi


\ifnum \EXAMPLEVERSION = 3
%%%%%%%%%%%%%%%%%%%%%%%%%%%%%%%%%%%%%%%%%%%%%%%%%%%%%%%%%%%%%%%%%%%%%%%%
\begin{frame}
\frametitle{Example - Linear dependence of two Euclidean vectors}
\end{frame}
%%%%%%%%%%%%%%%%%%%%%%%%%%%%%%%%%%%%%%%%%%%%%%%%%%%%%%%%%%%%%%%%%%%%%%%%
\fi 



\ifnum \EXAMPLEVERSION = 1
%%%%%%%%%%%%%%%%%%%%%%%%%%%%%%%%%%%%%%%%%%%%%%%%%%%%%%%%%%%%%%%%%%%%%%%%
\begin{frame}
\frametitle{Example}

The vectors $\mathbf{u}=(2,3)$ and $\mathbf{v}=(5,7.5)$ are linearly dependent.

What is the relationship between $\mathbf{u}$ and $\mathbf{v}$?
\end{frame}
%%%%%%%%%%%%%%%%%%%%%%%%%%%%%%%%%%%%%%%%%%%%%%%%%%%%%%%%%%%%%%%%%%%%%%%%


%%%%%%%%%%%%%%%%%%%%%%%%%%%%%%%%%%%%%%%%%%%%%%%%%%%%%%%%%%%%%%%%%%%%%%%%
\begin{frame}
\frametitle{Example}

Consider the vectors $\mathbf{u}=(2,3,0,4,-10)$ and $\mathbf{v}=(4,6,1,8,-20)$.

Are $\mathbf{u}$ and $\mathbf{v}$ linearly dependent?
\end{frame}
%%%%%%%%%%%%%%%%%%%%%%%%%%%%%%%%%%%%%%%%%%%%%%%%%%%%%%%%%%%%%%%%%%%%%%%%
\fi 


%%%%%%%%%%%%%%%%%%%%%%%%%%%%%%%%%%%%%%%%%%%%%%%%%%%%%%%%%%%%%%%%%%%%%%%%
\begin{frame}
\frametitle{Properties}

\begin{minipage}{0.55\textwidth}
For Euclidean vectors, when we have two vectors that are linearly dependent we say that they are \textit{colinear}.
\end{minipage}
\begin{minipage}{0.4\textwidth}
\begin{figure}[H]
\begin{center}
    \begin{tikzpicture}[line cap=round, line join=round, >=Triangle,scale=1]
		% coordinate system
		\coordinate (O) at (0,0);
		\draw [->,black] (-1.5,0)--(+2.5,0) node[right] {$x$}; % x-axis
		\draw [->,black] (0,-0.5)--(0,+2.5) node[right] {$y$}; % y-axis
		
    	\coordinate (u) at (0.5,+0.4);
    	\coordinate (v) at (1.5,1.2);
    	
    	\draw [-,nicegreen,line width=1pt] ($-0.2*(v)$)--($2*(v)$)node[left=3pt,black,pos=0.8, scale=1.2] {$\text{SPAN}(\textbf{u},\textbf{v})$};
		
		\draw [->,brightmaroon,line width=1.25pt] (O)--(v) node[left,black,pos=0.8] {$\textbf{v}$};
		\draw [->,airforceblue,line width=1.25pt] (O)--(u) node[left=10pt,black,pos=0.8] {$\textbf{u}$};
    \end{tikzpicture}
\end{center}
\end{figure}
\end{minipage}



\begin{minipage}{0.55\textwidth}
When we have three 3d Euclidean vectors that are linearly dependent we say that they are \textit{coplanar}. Any two of the vectors give a plane as their span and then adding the 3rd dependent vector gives linear combinations that remain in that plane.
\end{minipage}
\begin{minipage}{0.4\textwidth}
\begin{figure}[H]
\begin{center}
    \begin{tikzpicture}[line cap=round, line join=round, >=Triangle,scale=0.8]

		% coordinate system
		\coordinate (O) at (0,0);
		\draw [->,black] (O)--(-1.5,-1.5) node[right] {$x$}; % x-axis
		\draw [->,black] (O)--(+2.5,+0.0) node[right] {$y$}; % y-axis
		\draw [->,black] (O)--(+0.0,+2.0) node[left] {$z$}; % z-axis
    
    	
	    % plane vertices positions
    	\coordinate (A) at (-1.5,-0.8);
    	\coordinate (B) at (-0.5,+0.8);
    	\coordinate (C) at (+2.3,+1.1);
    	\coordinate (D) at (+1.3,-0.7);
    	
    	
		\draw [-,airforceblue,line width=1.2pt] (A)--(B)node[left=3pt,black,pos=0,scale=1.2] {$\text{SPAN}(\textbf{u},\textbf{v},\textbf{w})$};
		\draw [-,airforceblue,line width=1.2pt] (B)--(C);
		\draw [-,airforceblue,line width=1.2pt] (C)--(D);
		\draw [-,airforceblue,line width=1.2pt] (D)--(A);

		% vectors
		\coordinate (u) at (0.5,-0.1);
		\coordinate (v) at (1,0.5);
		\coordinate (w) at (-0.5,0.2);
		\draw [->,nicegreen,line width=1.25pt] (O)--(u) node[below=3pt,black,pos=0.9] {$\textbf{u}$};
		\draw [->,airforceblue,line width=1.25pt] (O)--(v) node[above=3pt,black,pos=0.9] {$\textbf{v}$};
		\draw [->,brightmaroon,line width=1.25pt] (O)--(w) node[above=3pt,black,pos=0.9] {$\textbf{w}$};
    \end{tikzpicture}
\end{center}
\end{figure}
\end{minipage}




\end{frame}
%%%%%%%%%%%%%%%%%%%%%%%%%%%%%%%%%%%%%%%%%%%%%%%%%%%%%%%%%%%%%%%%%%%%%%%%



%%%%%%%%%%%%%%%%%%%%%%%%%%%%%%%%%%%%%%%%%%%%%%%%%%%%%%%%%%%%%%%%%%%%%%%%
\begin{frame}
\frametitle{Definition - Linear independence}

A set of vectors $\{\mathbf{v}_1, \dots, \mathbf{v}_n\}$ from $V$ is said to be \textit{linearly independent} if they are not linearly dependent. That is, the equation
\begin{align*}
\alpha_1 \mathbf{v}_1 + \cdots + \alpha_n \mathbf{v}_n =  \mathbf{0}_V.
\end{align*}
implies that the constants $\alpha_1, \dots, \alpha_n$ \textit{are all zero}.
\end{frame}
%%%%%%%%%%%%%%%%%%%%%%%%%%%%%%%%%%%%%%%%%%%%%%%%%%%%%%%%%%%%%%%%%%%%%%%%




\ifnum \EXAMPLEVERSION = 1
%%%%%%%%%%%%%%%%%%%%%%%%%%%%%%%%%%%%%%%%%%%%%%%%%%%%%%%%%%%%%%%%%%%%%%%%
\begin{frame}
\frametitle{Example - Linear independence of three Euclidean vectors}

\noindent Consider the following vectors in $\mathbb{R}^3$:
\begin{align*}
\mathbf{u} = (1,-1,1), \quad \mathbf{v} = (1,1,1), \quad \text{and} \quad \mathbf{w} = (2,2,4).
\end{align*}
Show that these three vectors are linearly independent. \\

\noindent We start with the equation
\begin{align*}
\alpha \mathbf{u} + \beta \mathbf{v} + \gamma \mathbf{w} = \mathbf{0}
\end{align*}
with the goal of finding possible solutions for $\alpha$, $\beta$ and $\gamma$. Writing the vector equation in full gives
\begin{align*}
& \alpha (1,-1,1) + \beta (1,1,1) + \gamma (2,2,4) = (0,0,0) \\
& (\alpha + \beta + 2\gamma, \, -\alpha + \beta + 2\gamma, \, \alpha + \beta + 4 \gamma) = (0,0,0)
\end{align*}
\end{frame}
%%%%%%%%%%%%%%%%%%%%%%%%%%%%%%%%%%%%%%%%%%%%%%%%%%%%%%%%%%%%%%%%%%%%%%%%






%%%%%%%%%%%%%%%%%%%%%%%%%%%%%%%%%%%%%%%%%%%%%%%%%%%%%%%%%%%%%%%%%%%%%%%%
\begin{frame}
\frametitle{Example - Linear independence of three Euclidean vectors (cont.)}
Component by component this vector equation is actually a system of equations
\begin{align*}
\begin{cases}
\alpha + \beta + 2\gamma = 0 \\
-\alpha + \beta + 2\gamma = 0 \\
\alpha + \beta + 4\gamma = 0 
\end{cases}
\begin{matrix}
 \\
 L_2 \to L_2 + L_1 \\
 L_3 \to L_3 - L_1
\end{matrix}
 \quad \implies \quad
\begin{cases}
\alpha + \beta + 2\gamma = 0 \implies \alpha = -\beta - 2\gamma \\
2\beta + 4\gamma = 0 \implies \beta = -2 \gamma \\
2\gamma = 0
\end{cases} 
\end{align*}
The last equation implies $\gamma=0$, which them implies $\beta=0$, which then implies $\alpha=0$. So the assumption that 
\begin{align*}
\alpha \mathbf{u} + \beta \mathbf{v} + \gamma \mathbf{w} = \mathbf{0}
\end{align*}
implies that $\alpha=\beta=\gamma=0$. That is exactly the definition of linear independence.
\end{frame}
%%%%%%%%%%%%%%%%%%%%%%%%%%%%%%%%%%%%%%%%%%%%%%%%%%%%%%%%%%%%%%%%%%%%%%%%
\fi 


\ifnum \EXAMPLEVERSION = 3
%%%%%%%%%%%%%%%%%%%%%%%%%%%%%%%%%%%%%%%%%%%%%%%%%%%%%%%%%%%%%%%%%%%%%%%%
\begin{frame}
\frametitle{Example - Linear independence of three Euclidean vectors}
\end{frame}
%%%%%%%%%%%%%%%%%%%%%%%%%%%%%%%%%%%%%%%%%%%%%%%%%%%%%%%%%%%%%%%%%%%%%%%%

%%%%%%%%%%%%%%%%%%%%%%%%%%%%%%%%%%%%%%%%%%%%%%%%%%%%%%%%%%%%%%%%%%%%%%%%
\begin{frame}
\end{frame}
%%%%%%%%%%%%%%%%%%%%%%%%%%%%%%%%%%%%%%%%%%%%%%%%%%%%%%%%%%%%%%%%%%%%%%%%
\fi 



%%%%%%%%%%%%%%%%%%%%%%%%%%%%%%%%%%%%%%%%%%%%%%%%%%%%%%%%%%%%%%%%%%%%%%%%
\begin{frame}
\frametitle{Methodology}

Take note of the simple logic or methodology of the previous examples. We always start with the equation
\begin{align*}
\alpha_1 \mathbf{v}_1 + \alpha_2 \mathbf{v}_2 + \cdots + \alpha_n \mathbf{v}_n = \mathbf{0}_V
\end{align*}
and show either that by necessity all of the coefficients are zero (then the vectors are linearly independent), or that we can find at least one non-zero coefficient (then the vectors are linearly dependent). The exact method by which we determine these coefficients depends on which type of vectors we are considering.

\end{frame}
%%%%%%%%%%%%%%%%%%%%%%%%%%%%%%%%%%%%%%%%%%%%%%%%%%%%%%%%%%%%%%%%%%%%%%%%


\ifnum \EXAMPLEVERSION = 3
%%%%%%%%%%%%%%%%%%%%%%%%%%%%%%%%%%%%%%%%%%%%%%%%%%%%%%%%%%%%%%%%%%%%%%%%
\begin{frame}
\frametitle{Exercise - Linear dependence/independence}

Determine whether the following vectors are linearly dependent or independent
\begin{align*}
\mathbf{u}=(1,2,0), \quad \mathbf{v}=(0,1,-1) \quad \mathbf{w}=(1,0,3).
\end{align*}
\end{frame}
%%%%%%%%%%%%%%%%%%%%%%%%%%%%%%%%%%%%%%%%%%%%%%%%%%%%%%%%%%%%%%%%%%%%%%%%
\fi 




%%%%%%%%%%%%%%%%%%%%%%%%%%%%%%%%%%%%%%%%%%%%%%%%%%%%%%%%%%%%%%%%%%%%%%%%
\begin{frame}
\frametitle{Properties}

Given a set of linearly independent vectors $\{ \mathbf{v}_1, \, \mathbf{v}_2, \, \dots, \mathbf{v}_n \}$, if a new set of vectors $\{ \mathbf{v}_1, \, \mathbf{v}_2, \, \dots, \mathbf{v}_n, \mathbf{w} \}$ is linearly dependent, then $\mathbf{w}$ can be given as a linear combination of the other vectors. That is, there exist constants $\alpha_1, \alpha_2, \dots, \alpha_n \in \mathbb{R}$ such that
\begin{align*}
\mathbf{w} = \alpha_1\mathbf{v}_1 + \alpha_2 \mathbf{v}_2 + \dots + \alpha_n \mathbf{v}_n
\end{align*}
Additionally, given a set of independent vectors, any linear combination of a subset of these vectors is also linearly independent.

\end{frame}
%%%%%%%%%%%%%%%%%%%%%%%%%%%%%%%%%%%%%%%%%%%%%%%%%%%%%%%%%%%%%%%%%%%%%%%%



%%%%%%%%%%%%%%%%%%%%%%%%%%%%%%%%%%%%%%%%%%%%%%%%%%%%%%%%%%%%%%%%%%%%%%%%
\begin{frame}
\frametitle{Theorem - The span of a dependent set of vectors can be reduced}


Let $\mathcal{B}=\{\mathbf{v}_1, \, \mathbf{v}_2, \, \dots, \mathbf{v}_n\}$ be a set of vectors of $V$. If $\mathcal{B}$ is a set of linearly dependent vectors, then we can always remove one of the vectors to form a new set, $\mathcal{B}'=\mathcal{B} \backslash \{\mathbf{v}_k\}$ for some $k$, without changing the span: $\text{SPAN}(\mathcal{B}')=\text{SPAN}(\mathcal{B})$.
\end{frame}
%%%%%%%%%%%%%%%%%%%%%%%%%%%%%%%%%%%%%%%%%%%%%%%%%%%%%%%%%%%%%%%%%%%%%%%%



%\section{Basis, Coordinates and Dimension}
\section{Basis}



%%%%%%%%%%%%%%%%%%%%%%%%%%%%%%%%%%%%%%%%%%%%%%%%%%%%%%%%%%%%%%%%%%%%%%%%
\begin{frame}
\frametitle{Definition - Spanning set}

A set of vectors $\mathcal{A} = \{ \mathbf{v}_1, \, \dots, \mathbf{v}_n \}$ is said to \textit{span} a vector space $V$ if and only if
\begin{align*}
\text{SPAN}(\mathbf{v}_1, \, \dots, \mathbf{v}_n ) = V
\end{align*}
We also call $\mathcal{A}$ a spanning set of $V$. Every vector of $V$ can be written as a linear combination vectors from a spanning set.

\end{frame}
%%%%%%%%%%%%%%%%%%%%%%%%%%%%%%%%%%%%%%%%%%%%%%%%%%%%%%%%%%%%%%%%%%%%%%%%


\ifnum \EXAMPLEVERSION = 1
%%%%%%%%%%%%%%%%%%%%%%%%%%%%%%%%%%%%%%%%%%%%%%%%%%%%%%%%%%%%%%%%%%%%%%%%
\begin{frame}
\frametitle{Examples - Spanning sets}
The following sets are spanning sets of the given vector spaces
\begin{align*}
& \{ (1,0) \, (0,1) \} \quad \text{for} \quad \mathbb{R}^2 \\
& \{ (1,1), \, (0,-10) \} \quad \text{for} \quad \mathbb{R}^2 \\
& \{ (1,1) \, (1,0)\, (-2,1) \, (0,-10) \} \quad \text{for} \quad \mathbb{R}^2 \\
& \{ x^2, \, x, \, 1 \} \quad \text{for} \quad \mathcal{P}_2 \\
& \left\{ \begin{pmatrix} 1 & 0 \\ 0 & 0 \end{pmatrix},\, \begin{pmatrix} 0 & 1 \\ 0 & 0 \end{pmatrix}, \, \begin{pmatrix} 0 & 0 \\ 0 & 1 \end{pmatrix} \right\} \quad \text{for upper triangular 2x2 matrices} \\
\end{align*}
\end{frame}
%%%%%%%%%%%%%%%%%%%%%%%%%%%%%%%%%%%%%%%%%%%%%%%%%%%%%%%%%%%%%%%%%%%%%%%%


%%%%%%%%%%%%%%%%%%%%%%%%%%%%%%%%%%%%%%%%%%%%%%%%%%%%%%%%%%%%%%%%%%%%%%%%
\begin{frame}
\frametitle{Examples - Spanning sets}
The spanning set vectors can be used to express any element of the given vector space
\begin{align*}
& \{ (1,0) \, (0,1) \} \quad \text{for} \quad \mathbb{R}^2 \quad \longrightarrow \quad (x,y) = x(1,0) + y(0,1) \\ \\
& \{ (1,1), \, (0,-10) \} \quad \text{for} \quad \mathbb{R}^2 \quad \longrightarrow \quad (x,y) = x(1,1) + \frac{x-y}{10}(0,-10)  \\ \\
& \{ x^2, \, x, \, 1 \} \quad \text{for} \quad \mathcal{P}_2 \quad \longrightarrow \quad \text{arbitrary polynomial: } ax^2 + bx + c
\end{align*}
\end{frame}
%%%%%%%%%%%%%%%%%%%%%%%%%%%%%%%%%%%%%%%%%%%%%%%%%%%%%%%%%%%%%%%%%%%%%%%%
\fi 


\ifnum \EXAMPLEVERSION = 3
%%%%%%%%%%%%%%%%%%%%%%%%%%%%%%%%%%%%%%%%%%%%%%%%%%%%%%%%%%%%%%%%%%%%%%%%
\begin{frame}
\frametitle{Examples - Spanning sets}
\end{frame}
%%%%%%%%%%%%%%%%%%%%%%%%%%%%%%%%%%%%%%%%%%%%%%%%%%%%%%%%%%%%%%%%%%%%%%%%
\fi 



%%%%%%%%%%%%%%%%%%%%%%%%%%%%%%%%%%%%%%%%%%%%%%%%%%%%%%%%%%%%%%%%%%%%%%%%
\begin{frame}
\frametitle{Definition - Basis}

A \textit{basis of a vector space} $V$ is a minimal set of vectors which spans the vector space. Formally, the set of vectors $\mathcal{B}=\{\mathbf{v}_1, \dots, \mathbf{v}_n\}$ in a vector space $V$ is a basis of $V$ if it is a set of linearly independent vectors and $\text{SPAN}(\mathbf{v}_1, \dots, \mathbf{v}_n) = V$. 

\textit{Note}: bases are not unique, but they always contain the same number of vectors.
\end{frame}
%%%%%%%%%%%%%%%%%%%%%%%%%%%%%%%%%%%%%%%%%%%%%%%%%%%%%%%%%%%%%%%%%%%%%%%%



%%%%%%%%%%%%%%%%%%%%%%%%%%%%%%%%%%%%%%%%%%%%%%%%%%%%%%%%%%%%%%%%%%%%%%%%
\begin{frame}
\frametitle{Theorem - Bases of two dimensional Euclidean space}
Any two linearly independent vectors in $\mathbb{R}^2$ forms a basis of $\mathbb{R}^2$.
\end{frame}
%%%%%%%%%%%%%%%%%%%%%%%%%%%%%%%%%%%%%%%%%%%%%%%%%%%%%%%%%%%%%%%%%%%%%%%%



\ifnum \EXAMPLEVERSION = 1
%%%%%%%%%%%%%%%%%%%%%%%%%%%%%%%%%%%%%%%%%%%%%%%%%%%%%%%%%%%%%%%%%%%%%%%%
\begin{frame}
\frametitle{Example - Basis of $\mathbb{R}^3$}

\noindent Is the set $\mathcal{B}=\{\mathbf{u}, \, \mathbf{v}, \, \mathbf{w} \} = \left\{(1,-2,2), \, (2,1,0), \, (1,1,2) \right\}$ a basis of $\mathbb{R}^3$?
\end{frame}
%%%%%%%%%%%%%%%%%%%%%%%%%%%%%%%%%%%%%%%%%%%%%%%%%%%%%%%%%%%%%%%%%%%%%%%%


%%%%%%%%%%%%%%%%%%%%%%%%%%%%%%%%%%%%%%%%%%%%%%%%%%%%%%%%%%%%%%%%%%%%%%%%
\begin{frame}
We first check the linear independence of these vectors. Let $\alpha \mathbf{u} +  \beta \mathbf{v} + \gamma \mathbf{w} = \mathbf{0}_{\mathbb{R}^3}$. That is
\begin{align*}
& (\alpha + 2\beta + \gamma, \, -2\alpha + \beta + \gamma, \, 2\alpha + 2\gamma) = (0,0,0) \\
%
%
& \implies  \begin{cases}
\alpha + 2\beta + \gamma = 0 \\
-2\alpha + \beta + \gamma = 0 \\
2\alpha + 2\gamma = 0 
\end{cases}
\begin{matrix}
\, \\
L_2 \to L_2 + 2L_1 \\
L_3 \to L_3 - 2L_1 
\end{matrix} \\
%
%
& \implies  \begin{cases}
\alpha + 2\beta + \gamma = 0 \\
5\beta + 3\gamma = 0 \\
-4\beta = 0 
\end{cases} \\
%
%
& \implies \beta = 0 \\
& \implies \gamma = 0 \\
& \implies \alpha = 0
\end{align*}
Hence the set is linearly independent. 
\end{frame}
%%%%%%%%%%%%%%%%%%%%%%%%%%%%%%%%%%%%%%%%%%%%%%%%%%%%%%%%%%%%%%%%%%%%%%%%


%%%%%%%%%%%%%%%%%%%%%%%%%%%%%%%%%%%%%%%%%%%%%%%%%%%%%%%%%%%%%%%%%%%%%%%%
\begin{frame}
Now we need to show that $\mathcal{B}$ spans $\mathbb{R}^3$, that is $\mathbb{R}^3 = \text{SPAN}(\mathcal{B})$. To do this we start with an arbitrary vector $(x,y,z)\in\mathbb{R}^3$. Let $(x,y,z) = \alpha \mathbf{u} +  \beta \mathbf{v} + \gamma \mathbf{w}$. Now the goal is figure out what $\alpha$, $\beta$ and $\gamma$ have to be as functions of $x$, $y$ and $z$ if it is possible. So we develop the relation
\begin{align*}
&\begin{cases}
\alpha + 2\beta + \gamma = x \\
-2\alpha + \beta + \gamma = y \\
2\alpha + 2\gamma = z 
\end{cases}
\begin{matrix}
\, \\
L_2 \to L_2 + 2L_1 \\
L_3 \to L_3 - 2L_1 
\end{matrix}
%
%
\implies  \begin{cases}
\alpha + 2\beta + \gamma = x \\
5\beta + 3\gamma = y+2x \\
-4\beta = z-2x
\end{cases} \\
%
%
& \implies \beta = \frac{x}{2} - \frac{z}{4} \quad\implies\quad \gamma = \frac{y+2x-5\beta}{5} = -\frac{1}{10}x + \frac{1}{5}y+\frac{1}{4}z \\
& \implies \alpha = x - 2\beta - \gamma = \frac{1}{10}x - \frac{1}{5}y+\frac{1}{4}z
\end{align*}
So we can represent every vector in $\mathbb{R}^3$ as a linear combination of vectors in $\mathcal{B}$:
\begin{align*}
(x,y,z) = \left(\frac{1}{10}x - \frac{1}{5}y+\frac{1}{4}z\right) \mathbf{u} +  \left(\frac{1}{2}x - \frac{1}{4}z\right) \mathbf{v} + \left(-\frac{1}{10}x + \frac{1}{5}y+\frac{1}{4}z\right)\mathbf{w}
\end{align*}
and hence $\mathcal{B}$ spans $\mathbb{R}^3$. Therefore $\mathcal{B}$ is a basis of $\mathbb{R}^3$.
\end{frame}
%%%%%%%%%%%%%%%%%%%%%%%%%%%%%%%%%%%%%%%%%%%%%%%%%%%%%%%%%%%%%%%%%%%%%%%%
\fi 



\ifnum \EXAMPLEVERSION = 3
%%%%%%%%%%%%%%%%%%%%%%%%%%%%%%%%%%%%%%%%%%%%%%%%%%%%%%%%%%%%%%%%%%%%%%%%
\begin{frame}
\frametitle{Example - Basis of $\mathbb{R}^3$}

\noindent Is the set $\mathcal{B}=\{\mathbf{u}, \, \mathbf{v}, \, \mathbf{w} \} = \left\{(1,-2,2), \, (2,1,0), \, (1,1,2) \right\}$ a basis of $\mathbb{R}^3$?
\vspace{6cm}
\end{frame}
%%%%%%%%%%%%%%%%%%%%%%%%%%%%%%%%%%%%%%%%%%%%%%%%%%%%%%%%%%%%%%%%%%%%%%%%
\fi 






\section{Dimension}

%%%%%%%%%%%%%%%%%%%%%%%%%%%%%%%%%%%%%%%%%%%%%%%%%%%%%%%%%%%%%%%%%%%%%%%%
\begin{frame}
\frametitle{Definition - Dimension}
A vector space is said to be \textbf{finite-dimensional} if it has a basis with a finite number of vectors. This \textcolor{Purple}{\textit{number of vectors}} (or cardinality) of the basis is called the dimension of the vector space, denoted $\dim(V)$. We may also speak of an \textit{$n$-dimensional vector space} for a finite-dimensional vector space with dimension $n \in \mathbb{Z}$.
\end{frame}
%%%%%%%%%%%%%%%%%%%%%%%%%%%%%%%%%%%%%%%%%%%%%%%%%%%%%%%%%%%%%%%%%%%%%%%%



%%%%%%%%%%%%%%%%%%%%%%%%%%%%%%%%%%%%%%%%%%%%%%%%%%%%%%%%%%%%%%%%%%%%%%%%
\begin{frame}
\frametitle{Dimension and Canonical Basis of Euclidean vector spaces}

The Euclidean vector space $\mathbb{R}^n$ has dimension $n$.

The \textit{canonical basis} of the vector space of real $n$-tuples, $\mathbb{R}^n$, is the ordered set of $n$ $n$-tuples with $k^{th}$ element, $\mathbf{c}_k=(\alpha_1, \dots, \alpha_n)$ such that 
\begin{align*}
\alpha_j = 
\begin{cases} 
1 & \text{for } j= k, \\
0 & \text{for } j\neq k.
\end{cases}
\end{align*}
That is, as a set the canonical basis is
\begin{align*}
\mathcal{C}_n=\{ 
(1, 0, \dots, 0 ), \,
(0, 1, \dots, 0 ), \,
\dots, \,
\underbrace{(0, 0, \dots, 0, \overbrace{1}^{k^{th} \text{ place}}, 0, \dots, 0 )}_{k^{th} \text{ tuple}}, \,
\dots, \,
(0, 0, \dots, 1)
\}.
\end{align*}

\end{frame}
%%%%%%%%%%%%%%%%%%%%%%%%%%%%%%%%%%%%%%%%%%%%%%%%%%%%%%%%%%%%%%%%%%%%%%%%





\ifnum \EXAMPLEVERSION = 1
%%%%%%%%%%%%%%%%%%%%%%%%%%%%%%%%%%%%%%%%%%%%%%%%%%%%%%%%%%%%%%%%%%%%%%%%
\begin{frame}
\frametitle{Examples - Dimension}

$\mathbb{R}^3$ has a basis $\mathcal{A} = \left\{(1,-2,2), \, (2,1,0), \, (1,1,2) \right\}$ and therefore $\dim(\mathbb{R}^3)=3$.

The set $\mathcal{B} = \left\{(1,0,0), \, (0,-1,0), \, (0,0,2), \, (1,1,-1) \right\}$ is not a basis of $\mathbb{R}^3$ because it has more vectors than $\dim(\mathbb{R}^3)$.

The vector space of polynomials of degree up to 3, $\mathcal{P}_3$ has a basis $\mathcal{C} = \left\{1-x, \, 1+x^2, \, x^3, \, 2 \right\}$ and therefore $\dim(\mathcal{P}_3)=4$.

The vector space of 2x2 diagonal matrices, $\mathcal{M}$ has a basis $\mathcal{D} = \left\{\begin{pmatrix} 2 & 0 \\ 0 & 0 \end{pmatrix}, \, \begin{pmatrix} 0 & 0 \\ 0 & -1 \end{pmatrix} \right\}$ and therefore $\dim(\mathcal{M})=2$.

\end{frame}
%%%%%%%%%%%%%%%%%%%%%%%%%%%%%%%%%%%%%%%%%%%%%%%%%%%%%%%%%%%%%%%%%%%%%%%%
\fi 



\ifnum \EXAMPLEVERSION = 3
%%%%%%%%%%%%%%%%%%%%%%%%%%%%%%%%%%%%%%%%%%%%%%%%%%%%%%%%%%%%%%%%%%%%%%%%
\begin{frame}
\frametitle{Examples - Dimension}
$\mathbb{R}^3$ has a basis $\mathcal{A} = \left\{(1,-2,2), \, (2,1,0), \, (1,1,2) \right\}$ and therefore $\dim(\mathbb{R}^3) = \quad$.

The set $\mathcal{B} = \left\{(1,0,0), \, (0,-1,0), \, (0,0,2), \, (1,1,-1) \right\}$ is not a basis of $\mathbb{R}^3$ because \\ ...

The vector space of polynomials of degree up to 3, $\mathcal{P}_3$ has a basis $\mathcal{C} = \left\{1-x, \, 1+x^2, \, x^3, \, 2 \right\}$ and therefore $\dim(\mathcal{P}_3) = \quad$.

The vector space of 2x2 diagonal matrices, $\mathcal{M}$ has a basis $\mathcal{D} = \left\{\begin{pmatrix} 2 & 0 \\ 0 & 0 \end{pmatrix}, \, \begin{pmatrix} 0 & 0 \\ 0 & -1 \end{pmatrix} \right\}$ and therefore $\dim(\mathcal{M}) = \quad$.
\end{frame}
%%%%%%%%%%%%%%%%%%%%%%%%%%%%%%%%%%%%%%%%%%%%%%%%%%%%%%%%%%%%%%%%%%%%%%%%
\fi 





%%%%%%%%%%%%%%%%%%%%%%%%%%%%%%%%%%%%%%%%%%%%%%%%%%%%%%%%%%%%%%%%%%%%%%%%
\begin{frame}
\frametitle{Theorem - Dimension of subspaces}
Let $V$ be a finite-dimensional vector space of dimension $n$. Then every vector subspace, $S$, of $V$ is also finite-dimensional and
\begin{align*}
\dim(S) \leq \dim(V).
\end{align*}
Furthermore, if $\dim(S)=\dim(V)$ then $S=V$.
\end{frame}
%%%%%%%%%%%%%%%%%%%%%%%%%%%%%%%%%%%%%%%%%%%%%%%%%%%%%%%%%%%%%%%%%%%%%%%%



%%%%%%%%%%%%%%%%%%%%%%%%%%%%%%%%%%%%%%%%%%%%%%%%%%%%%%%%%%%%%%%%%%%%%%%%
\begin{frame}
\frametitle{Theorem - Basis demonstration with dimension}

Let $V$ be a vector space and $\mathcal{B}$ be set of vectors in $V$. Then $\mathcal{B}$ forms a basis of $V$ if its vectors are linearly independent and the number of vectors in $\mathcal{B}$ equals the dimension of $V$.

\vspace{1cm}

\textit{Note}: this theorem merely results from the definition of dimension. It seems circular, but if you can use your geometrical intuition to find the dimension of a space (for example we can know that a planar equation will define a 2 dimensional vector space) without finding the basis first, then it cuts the work in half.

\end{frame}
%%%%%%%%%%%%%%%%%%%%%%%%%%%%%%%%%%%%%%%%%%%%%%%%%%%%%%%%%%%%%%%%%%%%%%%%



\ifnum \EXAMPLEVERSION = 1
%%%%%%%%%%%%%%%%%%%%%%%%%%%%%%%%%%%%%%%%%%%%%%%%%%%%%%%%%%%%%%%%%%%%%%%%
\begin{frame}
\frametitle{Example - Dimension of an intersection of Euclidean subspaces}
\noindent Let $A=\text{SPAN}((1,1,0),\,(1,2,-1))$ and $B=\text{SPAN}((0,2,-1),\,(-1,1,-2))$ be two vector subspaces of $\mathbb{R}^3$. What is the dimension of $A \cap B$? \\

\noindent We first represent $A$ and $B$ in Cartesian form. For $A$, let $(x,y,z)=\alpha(1,1,0) + \beta(1,2,-1)$ for some $\alpha$, $\beta \in \mathbb{R}$, giving the system of equations
\begin{align*}
&\begin{cases}
x = \alpha + \beta \\
y = \alpha + 2\beta \\
z = -\beta
\end{cases}
\implies & y-x = \beta = -z
\implies & A = \left\{ (x,y,z)\in\mathbb{R}^3 \, | \, x - y - z = 0\right\}
\end{align*}
For $B$, let $(x,y,z)=\alpha(0,2,-1) + \beta(-1,1,-2)$ for some $\alpha$, $\beta \in \mathbb{R}$, giving the system of equations
\begin{align*}
&\begin{cases}
x = -\beta \\
y = 2\alpha + \beta \\
z = -\alpha - 2\beta
\end{cases}
\implies & y+2z = -3\beta = 3x
\implies & B = \left\{ (x,y,z)\in\mathbb{R}^3 \, | \, 3x - y - 2z = 0\right\}
\end{align*}

\end{frame}
%%%%%%%%%%%%%%%%%%%%%%%%%%%%%%%%%%%%%%%%%%%%%%%%%%%%%%%%%%%%%%%%%%%%%%%%


%%%%%%%%%%%%%%%%%%%%%%%%%%%%%%%%%%%%%%%%%%%%%%%%%%%%%%%%%%%%%%%%%%%%%%%%
\begin{frame}

Now the intersection of $A$ and $B$ has Cartesian form with both defining equations of $A$ and $B$ simultaneously true: $A \cap B \left\{ (x,y,z)\in\mathbb{R}^3 \, | \, x - y - z = 0, \, 3x - y - 2z = 0\right\}$.

Subtracting the first equation from the second eliminates $y$, and subtracting 3 of the first from the second eliminates $x$, giving
\begin{align*}
\begin{cases}
2x - z = 0 \, \implies x = \dfrac{z}{2} \\
2y + z = 0 \, \implies y = -\dfrac{z}{2}
\end{cases}
\end{align*}
As we can write $x$ and $y$ in terms of $z$, $z$ is a free variable $z=t\in\mathbb{R}$. So we have that any vector in $A\cap B$ can be written $(x,y,z)=(t/2,\,-t/2,\,t)=(1/2,\,-1/2,\,1)t$. The intersection space is then
\begin{align*}
A \cap B = \left\{ \left( x,y,z \right) \in \mathbb{R}^3 \, | \, 2x-z = 0, \, 2y + z =0 \right\} = \text{SPAN}\left( \, \left( \frac{1}{2},\,-\frac{1}{2},\,1 \right) \, \right)
\end{align*}
Since this vector $\mathbf{v}$ spans the intersection, and a single vector always forms a linearly independent set, we automatically have a basis of $A \cap B$: $ \mathcal{B} = \left\{ \left( 1/2,\,-1/2,\,1 \right)\right\}$. Since this basis has one member, $\dim(A \cap B)=1$. 
\end{frame}
%%%%%%%%%%%%%%%%%%%%%%%%%%%%%%%%%%%%%%%%%%%%%%%%%%%%%%%%%%%%%%%%%%%%%%%%
\fi 



\ifnum \EXAMPLEVERSION = 3
%%%%%%%%%%%%%%%%%%%%%%%%%%%%%%%%%%%%%%%%%%%%%%%%%%%%%%%%%%%%%%%%%%%%%%%%
\begin{frame}
\frametitle{Example - Dimension of an intersection of Euclidean subspaces}

\noindent Let $A=\text{SPAN}((1,1,0),\,(1,2,-1))$ and $B=\text{SPAN}((0,2,-1),\,(-1,1,-2))$ be two vector subspaces of $\mathbb{R}^3$. What is the dimension of $A \cap B$?

\vspace{5cm}

\end{frame}
%%%%%%%%%%%%%%%%%%%%%%%%%%%%%%%%%%%%%%%%%%%%%%%%%%%%%%%%%%%%%%%%%%%%%%%%

%%%%%%%%%%%%%%%%%%%%%%%%%%%%%%%%%%%%%%%%%%%%%%%%%%%%%%%%%%%%%%%%%%%%%%%%
\begin{frame}
\frametitle{Example - Dimension of an intersection of Euclidean subspaces}

\end{frame}
%%%%%%%%%%%%%%%%%%%%%%%%%%%%%%%%%%%%%%%%%%%%%%%%%%%%%%%%%%%%%%%%%%%%%%%%
\fi 



%%%%%%%%%%%%%%%%%%%%%%%%%%%%%%%%%%%%%%%%%%%%%%%%%%%%%%%%%%%%%%%%%%%%%%%%
\begin{frame}
\frametitle{Dimension of sum of subspaces}

Let $A$ and $B$ be two finite-dimensional subspaces of a vector space $V$ (which is either finite or infinite dimensional). Recall: the sum of these subspaces is the vector space given by $A + B = \{ \mathbf{a} + \mathbf{b} \, | \, \mathbf{a}\in A, \, \mathbf{b}\in B \}$.

The sum space $A+B$ is also a finite-dimensional vector space with
\begin{align*}
\dim(A+B) = \dim(A) + \dim(B) - \dim(A \cap B).
\end{align*}

Particular cases:
\begin{itemize}
\item If $A+B = A \oplus B$ (a direct sum), then $A \cap B = \{\mathbf{0}_V\}$. So we have 
\begin{align*}
\dim(A\oplus B) = \dim(A) + \dim(B)
\end{align*}

\item If $A$ and $B$ are \textit{complementary}, then $A \cap B = \{\mathbf{0}_V\}$ and $A+B = V$. Hence
\begin{align*}
\dim(A\oplus B) = \dim(A) + \dim(B) = \dim(V)
\end{align*}
\end{itemize}

\end{frame}
%%%%%%%%%%%%%%%%%%%%%%%%%%%%%%%%%%%%%%%%%%%%%%%%%%%%%%%%%%%%%%%%%%%%%%%%







\ifnum \EXAMPLEVERSION = 1
%%%%%%%%%%%%%%%%%%%%%%%%%%%%%%%%%%%%%%%%%%%%%%%%%%%%%%%%%%%%%%%%%%%%%%%%
\begin{frame}
\frametitle{Example - Dimension of sum space}

Long ago we showed that the vector space $A=\{(x,y,z)\in\mathbb{R}^3 \, | \, x+2y-3z=0\}$ is equivalent to the span of vectors $\mathbf{u}=(3,0,1)$ and $\mathbf{v}=(0,3,2)$. These vectors are linearly independent (show this!), and so the set $\{\mathbf{u},\mathbf{v}\}$ forms a basis of $A$. Hence $A$ has dimension 2 (it is a plane after all).

Consider another planar subspace $B=\{(x,y,z)\in\mathbb{R}^3 \, | \, x+y+z=0\}$.

\begin{itemize}
\item Find a basis for $B$.
\item Show that $A+B = \mathbb{R}^3$.
\item Is $A+B$ a direct sum?
\item Does $\dim(A+B)=\dim(A)+\dim(B)$?
\end{itemize}

\end{frame}
%%%%%%%%%%%%%%%%%%%%%%%%%%%%%%%%%%%%%%%%%%%%%%%%%%%%%%%%%%%%%%%%%%%%%%%%
\fi 





\ifnum \EXAMPLEVERSION = 3
%%%%%%%%%%%%%%%%%%%%%%%%%%%%%%%%%%%%%%%%%%%%%%%%%%%%%%%%%%%%%%%%%%%%%%%%
\begin{frame}
\frametitle{Example - Dimension of sum space}

For $A=\{(x,y,z)\in\mathbb{R}^3 \, | \, x+2y-3z=0\}$ and $B=\{(x,y,z)\in\mathbb{R}^3 \, | \, x+y+z=0\}$: 1) Find a basis for $B$. 2) Show that $A+B = \mathbb{R}^3$. 3) Is $A+B$ a direct sum? 4) Does $\dim(A+B)=\dim(A)+\dim(B)$?

\vspace{4.5cm}
\end{frame}
%%%%%%%%%%%%%%%%%%%%%%%%%%%%%%%%%%%%%%%%%%%%%%%%%%%%%%%%%%%%%%%%%%%%%%%%

%%%%%%%%%%%%%%%%%%%%%%%%%%%%%%%%%%%%%%%%%%%%%%%%%%%%%%%%%%%%%%%%%%%%%%%%
\begin{frame}

\end{frame}
%%%%%%%%%%%%%%%%%%%%%%%%%%%%%%%%%%%%%%%%%%%%%%%%%%%%%%%%%%%%%%%%%%%%%%%%
\fi 




\section{Coordinates}





%%%%%%%%%%%%%%%%%%%%%%%%%%%%%%%%%%%%%%%%%%%%%%%%%%%%%%%%%%%%%%%%%%%%%%%%
\begin{frame}
\frametitle{Coordinates in a basis}

Let $\mathcal{B}=\{\mathbf{v}_1, \dots, \mathbf{v}_n \}$ be a basis of a vector space $V$. As $\text{SPAN}(\mathcal{B}) = V$, every vector of $V$ can be expressed as a linear combination of the basis vectors
\begin{align*}
\forall \, \mathbf{v}\in V, \quad \exists \, \alpha_1, \dots, \alpha_n, \quad\text{such that} \quad \mathbf{v} = \alpha_1 \mathbf{v}_1 + \cdots + \alpha_n \mathbf{v}_n.
\end{align*}
The \textit{unique} coefficients in this linear combination are called the \textcolor{Mahogany}{\textit{coordinates}} of the vector $\mathbf{v}$ \textit{in the basis} $\mathcal{B}$. We denote these coordinates with a column
\begin{align*}
[\mathbf{v}]_\mathcal{B}
=
\begin{pmatrix}
\alpha_1 \\
\vdots \\
\alpha_n
\end{pmatrix}
\end{align*}

\end{frame}
%%%%%%%%%%%%%%%%%%%%%%%%%%%%%%%%%%%%%%%%%%%%%%%%%%%%%%%%%%%%%%%%%%%%%%%%



\ifnum \EXAMPLEVERSION = 1
%%%%%%%%%%%%%%%%%%%%%%%%%%%%%%%%%%%%%%%%%%%%%%%%%%%%%%%%%%%%%%%%%%%%%%%%
\begin{frame}
\frametitle{Example: coordinates in $\mathbb{R}^2$}

Let's express an abritrary vector $\mathbf{v}\in\mathbb{R}^2$ in terms of the basis $\mathcal{B} = \{ \mathbf{b}_1, \mathbf{b}_2 \} = \{ (1,-1), (1,1) \}$ as shown below

\begin{minipage}{0.45\textwidth}
\begin{figure}[H]
\centering
\begin{tikzpicture}
	% coordinate axis
	\draw[->] (-1, 0) -- (4,0) node[below left] {$x$};
	\draw[->] ( 0,-0.5) -- (0,3) node[below left] {$y$};

	\coordinate (O) at (0,0);
	\coordinate (a) at (1,-1);
	\coordinate (b) at (1,1);
	\coordinate (c) at (3,1);
	
	%basis vectors
	\draw [->, blue] (O) --(a) node[pos=1,right=2pt,black] {$\mathbf{b}_1$};
	\draw [->, blue] (O) --(b) node[pos=1,above,black] {$\mathbf{b}_2$};

	% arbitrary vector
	\draw [->] (O) --(c) node[pos=1,right] {$\mathbf{v}=(x, y)$};
	%\draw [->] (a) --($(a)+1.3*(r)$) node[pos=0.9,above,rotate=26.5] {$\mathbf{a}}$};
	
\end{tikzpicture}
\end{figure}
\end{minipage}\hspace{0.5cm}
\begin{minipage}{0.45\textwidth}
Start with the linear combination 
\begin{align*}
\mathbf{v}=\alpha \mathbf{b}_1 + \beta \mathbf{b}_2
\end{align*}
where the coefficients give the coordinates
\begin{align*}
[\mathbf{v}]_\mathcal{B}
=
\begin{pmatrix}
\alpha \\
\beta
\end{pmatrix}
\end{align*}
\end{minipage}

So we have
\begin{align*}
\mathbf{v} = (x,y) = \alpha(1,-1) + \beta(1,1) = (\alpha + \beta, -\alpha + \beta)
\end{align*}

\end{frame}
%%%%%%%%%%%%%%%%%%%%%%%%%%%%%%%%%%%%%%%%%%%%%%%%%%%%%%%%%%%%%%%%%%%%%%%%



%%%%%%%%%%%%%%%%%%%%%%%%%%%%%%%%%%%%%%%%%%%%%%%%%%%%%%%%%%%%%%%%%%%%%%%%
\begin{frame}
\frametitle{Example: coordinates in $\mathbb{R}^2$}

This gives the system of equations
\begin{align*}
\begin{cases}
\alpha + \beta = x \\
-\alpha + \beta = y 
\end{cases}
%
\quad
%
\begin{matrix}
L_1 \to L_1 - L_2 \\
\iff  \\
L_2 \to L_1 + L_2 \\
\end{matrix}
%
\quad
%
\begin{cases}
2\alpha= x-y \\
2\beta = x+y 
\end{cases}
%
\iff
%
\begin{cases}
\alpha= \frac{x-y}{2} \\
\beta = \frac{x+y}{2} 
\end{cases}
\end{align*}

So we have the linear combination in terms of only $x$ and $y$
\begin{align*}
\mathbf{v} = \frac{x-y}{2} (1,-1) + \frac{x+y}{2} (1,1) 
%
\quad \implies \quad 
%
[\mathbf{v}]_\mathcal{B}
=
\begin{pmatrix}
\frac{x-y}{2}  \\
\frac{x+y}{2}
\end{pmatrix}
\end{align*}

Challenge question: can you show that this is equivalent to rotating the coordinate system clockwise by 45 degrees?

\end{frame}
%%%%%%%%%%%%%%%%%%%%%%%%%%%%%%%%%%%%%%%%%%%%%%%%%%%%%%%%%%%%%%%%%%%%%%%%
\fi 


\ifnum \EXAMPLEVERSION = 3
%%%%%%%%%%%%%%%%%%%%%%%%%%%%%%%%%%%%%%%%%%%%%%%%%%%%%%%%%%%%%%%%%%%%%%%%
\begin{frame}
\frametitle{Example - Coordinates in $\mathbb{R}^2$}
\end{frame}

\begin{frame}
\end{frame}
%%%%%%%%%%%%%%%%%%%%%%%%%%%%%%%%%%%%%%%%%%%%%%%%%%%%%%%%%%%%%%%%%%%%%%%%
\fi 





\ifnum \EXAMPLEVERSION = 1
%%%%%%%%%%%%%%%%%%%%%%%%%%%%%%%%%%%%%%%%%%%%%%%%%%%%%%%%%%%%%%%%%%%%%%%%
\begin{frame}
\frametitle{Example: coordinates in $\mathbb{R}^3$}

Let $\mathcal{B} = \{\mathbf{b}_1, \mathbf{b}_2, \mathbf{b}_3 \} = \{(1,2,3), \, (0,1,4), \, (2,7,17) \}$ be a basis of $\mathbb{R}^3$.

Determine the coordinates of an arbitrary element of $\mathbb{R}^3$ in this basis.
\end{frame}
%%%%%%%%%%%%%%%%%%%%%%%%%%%%%%%%%%%%%%%%%%%%%%%%%%%%%%%%%%%%%%%%%%%%%%%%





%%%%%%%%%%%%%%%%%%%%%%%%%%%%%%%%%%%%%%%%%%%%%%%%%%%%%%%%%%%%%%%%%%%%%%%%
\begin{frame}
\frametitle{Example: coordinates in $\mathbb{R}^3$}
\vspace{0.5cm}
Let $\mathbf{w}=(x,y,z)$ and $[\mathbf{w}]_\mathcal{B}=\begin{pmatrix}\alpha \\ \beta \\ \gamma \end{pmatrix}$.

We're looking to express $\mathbf{w}$ as a linear combination of the basis vectors: $\mathbf{w} = \alpha \mathbf{b}_1 + \beta \mathbf{b}_2  + \gamma \mathbf{b}_3$. Fully writing developing this equation gives a linear system:
\begin{align*}
\mathbf{w} = \alpha(1,2,3) + \beta (0,1,4)  + \gamma (2,7,17) = (\alpha + 2\gamma, \, 2\alpha +  \beta + 7\gamma, \, 3\alpha + 4\beta + 17 \gamma)
\end{align*}

\begin{align*}
\begin{cases}
x =  \alpha +           2\gamma \\
y = 2\alpha +  \beta +  7\gamma \\
z = 3\alpha + 4\beta + 17\gamma
\end{cases}
%
\quad
\begin{matrix}
L_2 \to L_2 - 2L_1 \\
\iff \\
L_3 \to L_3 - 3L_1 
\end{matrix}
%
\quad
\begin{cases}
x =  \alpha + 2\gamma \\
y-2x = \beta + 3\gamma \\
z-3x = 4\beta + 11\gamma
\end{cases}
%
\quad
\begin{matrix}
 \\
\iff \\
L_3 \to L_3 - 4L_2 
\end{matrix}
\end{align*}
\end{frame}
%%%%%%%%%%%%%%%%%%%%%%%%%%%%%%%%%%%%%%%%%%%%%%%%%%%%%%%%%%%%%%%%%%%%%%%%





%%%%%%%%%%%%%%%%%%%%%%%%%%%%%%%%%%%%%%%%%%%%%%%%%%%%%%%%%%%%%%%%%%%%%%%%
\begin{frame}

\begin{align*}
\begin{matrix}
 \\
\iff \\
L_3 \to L_3 - 4L_2 
\end{matrix}
%
\quad
\begin{cases}
x =  \alpha + 2\gamma \\
y-2x = \beta + 3\gamma \\
5x -4y + z = -\gamma
\end{cases}
\iff
%
\quad
\begin{cases}
\alpha = 11x - 8y + 3z \\
\beta = 13x - 11y + 3z \\
\gamma = - 5x + 4y - z
\end{cases}
\end{align*}
So we have the three linear combination coefficients of $\mathbf{w}$ in the basis $\mathcal{B}$ as functions of $x$, $y$ and z. That is:
\begin{align*}
[(x,y,z)]_\mathcal{B} = \begin{pmatrix}
11x - 8y + 3z \\
13x - 11y + 3z \\
- 5x + 4y - z
\end{pmatrix}
\end{align*}
\end{frame}
%%%%%%%%%%%%%%%%%%%%%%%%%%%%%%%%%%%%%%%%%%%%%%%%%%%%%%%%%%%%%%%%%%%%%%%%
\fi 






\ifnum \EXAMPLEVERSION = 3
%%%%%%%%%%%%%%%%%%%%%%%%%%%%%%%%%%%%%%%%%%%%%%%%%%%%%%%%%%%%%%%%%%%%%%%%
\begin{frame}
\frametitle{Example - Coordinates in $\mathbb{R}^3$}
\end{frame}

\begin{frame}
\end{frame}
%%%%%%%%%%%%%%%%%%%%%%%%%%%%%%%%%%%%%%%%%%%%%%%%%%%%%%%%%%%%%%%%%%%%%%%%
\fi 

\end{document}



%%%%%%%%%%%%%%%%%%%%%%%%%%%%%%%%%%%%%%%%%%%%%%%%%%%%%%%%%%%%%%%%%%%%%%%%
\begin{frame}
\frametitle{Definition - }
\end{frame}
%%%%%%%%%%%%%%%%%%%%%%%%%%%%%%%%%%%%%%%%%%%%%%%%%%%%%%%%%%%%%%%%%%%%%%%%




%%%%%%%%%%%%%%%%%%%%%%%%%%%%%%%%%%%%%%%%%%%%%%%%%%%%%%%%%%%%%%%%%%%%%%%%
\begin{frame}
\frametitle{Properties - }
\end{frame}
%%%%%%%%%%%%%%%%%%%%%%%%%%%%%%%%%%%%%%%%%%%%%%%%%%%%%%%%%%%%%%%%%%%%%%%%



%%%%%%%%%%%%%%%%%%%%%%%%%%%%%%%%%%%%%%%%%%%%%%%%%%%%%%%%%%%%%%%%%%%%%%%%
\begin{frame}
\frametitle{Theorem - }
\end{frame}
%%%%%%%%%%%%%%%%%%%%%%%%%%%%%%%%%%%%%%%%%%%%%%%%%%%%%%%%%%%%%%%%%%%%%%%%



%%%%%%%%%%%%%%%%%%%%%%%%%%%%%%%%%%%%%%%%%%%%%%%%%%%%%%%%%%%%%%%%%%%%%%%%
\begin{frame}
\frametitle{Example - }
\end{frame}
%%%%%%%%%%%%%%%%%%%%%%%%%%%%%%%%%%%%%%%%%%%%%%%%%%%%%%%%%%%%%%%%%%%%%%%%





%%%%%%%%%%%%%%%%%%%%%%%%%%%%%%%%%%%%%%%%%%%%%%%%%%%%%%%%%%%%%%%%%%%%%%%%
\begin{frame}
\frametitle{title}
\fontsize{9pt}{10pt}\selectfont
\end{frame}
%%%%%%%%%%%%%%%%%%%%%%%%%%%%%%%%%%%%%%%%%%%%%%%%%%%%%%%%%%%%%%%%%%%%%%%%


%%%% COLOUR CHOICES
% \textcolor{MidnightBlue}{}
% \textcolor{Maroon}{}
% \textcolor{Purple}{}
% \textcolor{BurntOrange}{}
% \textcolor{MidnightBlue}{}
% \textcolor{Mahogany}{}
% \textcolor{ForestGreen}{}
