\documentclass[usenames,dvipsnames,aspectratio=169,10pt]{beamer}
%\usetheme{default}
\usetheme[progressbar=frametitle]{metropolis}

\def \EXAMPLEVERSION {3} % 1 for examples, 2 to hide examples (they are in textbook), 3 to hide examples but leave blank slide

\def \SCHOOLVERSION {2} % 1 for neutral, 2 for ISEP

%\documentclass[12pt]{book}
\usepackage{amsfonts}
\usepackage{amsmath}
\usepackage{amssymb}
\usepackage{graphicx}
\usepackage[authoryear]{natbib}
%\usepackage[margin=2.5cm]{geometry}
%\usepackage{hyperref}
\usepackage[font=footnotesize]{caption}
\usepackage{float}
\usepackage{caption}
\usepackage{subcaption}
\usepackage{setspace}
\usepackage{cleveref}
\usepackage{lscape}
\usepackage{multirow}
\usepackage{nicematrix}

% for tikz
\usepackage{tikz}
\usetikzlibrary{angles, arrows.meta, calc, quotes}
\usetikzlibrary{decorations.pathreplacing,calligraphy}
\usetikzlibrary{patterns}
\usetikzlibrary{bending,matrix,positioning}
\usetikzlibrary{arrows, fit, shapes, backgrounds}

\usepackage{tikz-3dplot}
\usepackage{xcolor}


% for red line canceling diagonally
\usepackage{cancel}
\renewcommand{\CancelColor}{\color{red}}

\captionsetup{font=small,labelfont=bf,singlelinecheck=off,margin=2cm,justification=justified}
\numberwithin{equation}{section}

\newcommand{\defbox}[3]
	{
		\vspace{0.5cm} 
		\noindent \fbox{\begin{minipage}{\linewidth}
		\textbf{{#1}DEFINITION{#2}} #3
		\end{minipage}}
		\vspace{0.5cm}
	}
	
	
\newcommand{\defnobox}[3]
	{
		\vspace{0.5cm} 
		\noindent \begin{minipage}{\linewidth}
		\textbf{{#1}DEFINITION{#2}} #3
		\end{minipage}
		\vspace{0.5cm}
	}
	
%%%% To get a nice colourful box around an equation
\newcommand*{\colourboxed}{}
\def\colourboxed#1#{%
  \colourboxedAux{#1}%
}
\newcommand*{\colourboxedAux}[3]{%
  % #1: optional argument for color model
  % #2: color specification
  % #3: formula
  \begingroup
    \colorlet{cb@saved}{.}%
    \color#1{#2}%
    \boxed{%
      \color{cb@saved}%
      #3%
    }%
  \endgroup
}


%%%%%% COLOURS %%%%%%%%%%
\definecolor{airforceblue}{rgb}{0.36, 0.54, 0.66}
\definecolor{battleshipgrey}{rgb}{0.52, 0.52, 0.51}
\definecolor{brightmaroon}{rgb}{0.76, 0.13, 0.28}

% isep colours
\definecolor{isepblue1}{RGB}{0, 97, 161}      % teinte à 100%
\definecolor{isepblue2}{RGB}{77, 144, 189}    % teinte à 70%
\definecolor{isepblue3}{RGB}{179, 208, 227}   % teinte à 30%
\definecolor{iseporange1}{RGB}{244, 161, 0}   % teinte à 100%
\definecolor{iseporange2}{RGB}{234, 189, 100} % teinte à 70%
\definecolor{iseporange3}{RGB}{252, 227, 179} % teinte à 30%
%%%%%%%%%%%%%%%%

% beamer stuff
\setbeamertemplate{navigation symbols}{}
\setbeamersize{text margin left=1.0cm,text margin right=1.0cm}
\setbeamercolor{background canvas}{bg=white}

\ifnum \SCHOOLVERSION = 2
	%%%% ISEP COLOURS %%%%
	\setbeamercolor{frametitle}{bg=isepblue1, fg=white}
	\setbeamercolor{progress bar}{fg=iseporange1}
	\setbeamercolor{itemize item}{fg=iseporange1,bg=iseporange1}
	%%%%%%%%%%%%%%%%%%%%%%
\fi

\setbeamerfont{frametitle}{family=\fontfamily{qag}\selectfont} % choose font for frame titles
\setbeamerfont{title}{family=\fontfamily{qag}\selectfont} % choose font for title
\setbeamerfont{subtitle}{family=\fontfamily{qag}\selectfont} % choose font for subtitle
\setbeamerfont{section title}{family=\fontfamily{qag}\selectfont} % choose font for titles
%\fontfamily{qag}\selectfont %choose font for main text % put after begin{document}

% to make the progress bar a little thicker
\makeatletter
\setlength{\metropolis@titleseparator@linewidth}{1.5pt}
\setlength{\metropolis@progressonsectionpage@linewidth}{1.5pt}
\setlength{\metropolis@progressinheadfoot@linewidth}{1.5pt}
\makeatother

\begin{document}

\title{Linear Algebra}
\subtitle{Matrix Algebra}
\author{Andrew Lehmann}
\ifnum \SCHOOLVERSION = 2
	\institute{\'{E}cole d'ing\'{e}nieurs du num\'{e}rique}
\fi
\date{\textit{Last updated: \today}}

% logo of university
\ifnum \SCHOOLVERSION = 2
	\titlegraphic{\includegraphics[width=3cm]{/home/andrew/Dropbox/ISEP/admin/logo-isep-2023.png} }
\fi


\begin{frame}
\titlepage
\end{frame}


\section{Basic Concepts}



%%%%%%%%%%%%%%%%%%%%%%%%%%%%%%%%%%%%%%%%%%%%%%%%%%%%%%%%%%%%%%%%%%%%%%%%
\begin{frame}
\frametitle{Definition - Matrix}

A \textcolor{Purple}{matrix} is a collection of numbers usually represented by a rectangular array. For example, an $m \times n$ (said m by n) \textcolor{Purple}{matrix} $A$ with coefficients, $a_{ij}$, from a field $\mathbb{F}$ (e.g. rational numbers) would be represented by an array with $m$ rows and $n$ columns:
\begin{align*}
A =
\begin{pmatrix}
a_{11} & a_{12} & a_{13} & \cdots & a_{1j} & \cdots & a_{1n} \\
a_{21} & a_{22} & a_{23} & \cdots & a_{2j} & \cdots & a_{2n} \\
a_{31} & a_{32} & a_{33} & \cdots & a_{3j} & \cdots & a_{3n} \\
\vdots & \vdots & \vdots & \ddots & \vdots & \ddots & \vdots \\
a_{i1} & a_{12} & a_{13} & \cdots & a_{ij} & \cdots & a_{in} \\
\vdots & \vdots & \vdots & \ddots & \vdots & \ddots & \vdots \\
a_{m1} & a_{m2} & a_{m3} & \cdots & a_{mj} & \cdots & a_{mn}
\end{pmatrix}
=
\left( a_{ij} \right)_{\substack{ 1 \leq i \leq m \\ 1 \leq j \leq n }}
\end{align*}

By convention we denote matrices with capital letters and their coefficients in lowercase. We might sometimes use $a_{ij}=\left(A\right)_{ij}$ when it's clear.
\end{frame}
%%%%%%%%%%%%%%%%%%%%%%%%%%%%%%%%%%%%%%%%%%%%%%%%%%%%%%%%%%%%%%%%%%%%%%%%







%%%%%%%%%%%%%%%%%%%%%%%%%%%%%%%%%%%%%%%%%%%%%%%%%%%%%%%%%%%%%%%%%%%%%%%%
\begin{frame}
\frametitle{Definition - Set of all $n \times m$ matrices}

We write the \textbf{\textcolor{Purple}{set of all $m \times n$ matrices}} with coefficients in $\mathbb{F}$ as
\begin{align*}
\mathcal{M}_{m,n}(\mathbb{F})
\end{align*}

Reminder of fields:
\begin{table}
\begin{tabular}{lcl}
Natural numbers  & $\mathbb{N}$ & $\{1,2,3,4,\dots\}$ and sometimes 0 \\
Integers         & $\mathbb{Z}$ & $\{\dots,-3,-2,0,1,2,3,\dots\}$ \\
Rational numbers & $\mathbb{Q}$ & $\{\dots,-1/2,-1/3,0,1,2,5/2,11/2,\dots\}$ \\
Real numbers     & $\mathbb{R}$ & $\{\dots,-1,-1/2,0,1,22/3,\pi,e,\sqrt{2},\dots\}$ \\
Complex numbers  & $\mathbb{C}$ & $\{\dots,1+i,i,0,\pi,e,\sqrt{2},-1/2,\dots\}$ 
\end{tabular}
\end{table}

For the rest of this course we will almost exclusively use real numbers, and write $\mathcal{M}_{m,n} = \mathcal{M}_{m,n}(\mathbb{R})$.
\end{frame}
%%%%%%%%%%%%%%%%%%%%%%%%%%%%%%%%%%%%%%%%%%%%%%%%%%%%%%%%%%%%%%%%%%%%%%%%







%%%%%%%%%%%%%%%%%%%%%%%%%%%%%%%%%%%%%%%%%%%%%%%%%%%%%%%%%%%%%%%%%%%%%%%%
\begin{frame}
\frametitle{Definition - Columns of a matrix}

We denote the \textbf{columns} of a matrix $A \in \mathcal{M}_{m,n}$ as $A^{(1)}$, $A^{(2)}$, $A^{(3)}$, etc where the j$^\text{th}$ column is

\begin{align*}
A^{(j)} =
\begin{pmatrix}
 a_{1j} \\
 a_{2j} \\
 a_{3j} \\
 \vdots \\
 a_{ij} \\
 \vdots \\
 a_{mj}
\end{pmatrix}
\end{align*}

\end{frame}
%%%%%%%%%%%%%%%%%%%%%%%%%%%%%%%%%%%%%%%%%%%%%%%%%%%%%%%%%%%%%%%%%%%%%%%%







%%%%%%%%%%%%%%%%%%%%%%%%%%%%%%%%%%%%%%%%%%%%%%%%%%%%%%%%%%%%%%%%%%%%%%%%
\begin{frame}
\frametitle{Definition - Rows of a matrix}

We denote the \textbf{rows} of a matrix $A \in \mathcal{M}_{m,n}$ as $A_{(1)}$, $A_{(2)}$, $A_{(3)}$, etc where the i$^\text{th}$ row is

\begin{align*}
A_{(i)} =
\begin{pmatrix}
a_{i1} & a_{12} & a_{13} & \cdots & a_{ij} & \cdots & a_{in}
\end{pmatrix}
\end{align*}

\end{frame}
%%%%%%%%%%%%%%%%%%%%%%%%%%%%%%%%%%%%%%%%%%%%%%%%%%%%%%%%%%%%%%%%%%%%%%%%








\ifnum \EXAMPLEVERSION = 1
%%%%%%%%%%%%%%%%%%%%%%%%%%%%%%%%%%%%%%%%%%%%%%%%%%%%%%%%%%%%%%%%%%%%%%%%
\begin{frame}
\frametitle{Example matrices}

Square matrices
\begin{align*}
A =
\begin{pmatrix}
1 & 3 \\
7 & 2
\end{pmatrix}
\quad\quad
B =
\begin{pmatrix}
1.5 & 3 & \pi \\
7 & 2 & 2i \\
1 & 1 & 1
\end{pmatrix}
\quad
B \in \mathcal{M}_{3,3}(\mathbb{C})
\end{align*}
Identity matrices
\begin{align*}
I_2 =
\begin{pmatrix}
1 & 0 \\
0 & 1
\end{pmatrix}
\quad\quad
I_3 =
\begin{pmatrix}
1 & 0 & 0 \\
0 & 1 & 0 \\
0 & 0 & 1
\end{pmatrix}
\quad
\text{(Once we define matrix multiplication: $I_n A = A$)}
\end{align*}
Empty/null matrices
\begin{align*}
A =
\begin{pmatrix}
0 & 0 & 0 & 0 \\
0 & 0 & 0 & 0
\end{pmatrix}
\end{align*}
\end{frame}
%%%%%%%%%%%%%%%%%%%%%%%%%%%%%%%%%%%%%%%%%%%%%%%%%%%%%%%%%%%%%%%%%%%%%%%%
\fi

\ifnum \EXAMPLEVERSION = 3
%%%%%%%%%%%%%%%%%%%%%%%%%%%%%%%%%%%%%%%%%%%%%%%%%%%%%%%%%%%%%%%%%%%%%%%%
\begin{frame}
\frametitle{Example matrices}
\end{frame}
%%%%%%%%%%%%%%%%%%%%%%%%%%%%%%%%%%%%%%%%%%%%%%%%%%%%%%%%%%%%%%%%%%%%%%%%
\fi





%%%%%%%%%%%%%%%%%%%%%%%%%%%%%%%%%%%%%%%%%%%%%%%%%%%%%%%%%%%%%%%%%%%%%%%%
\begin{frame}
\frametitle{Triangular matrices}

For square matrices, an \textit{upper} triangular matrix only has elements in the upper triangle, e.g.
\begin{align*}
\begin{pmatrix}
1.5 & 3 & \pi \\
0 & 2 & 2 \\
0 & 0 & 1
\end{pmatrix}
\end{align*}
while a \textit{lower} triangular matrix only has elements in the lower triangle, e.g.
\begin{align*}
\begin{pmatrix}
1.5 & 0  & 0 & 0 \\
1   & 2  & 0 & 0 \\
0   & -1 & 1 & 0 \\
0.5 & 3  & 1 & 2
\end{pmatrix}
\end{align*}

\end{frame}
%%%%%%%%%%%%%%%%%%%%%%%%%%%%%%%%%%%%%%%%%%%%%%%%%%%%%%%%%%%%%%%%%%%%%%%%





%%%%%%%%%%%%%%%%%%%%%%%%%%%%%%%%%%%%%%%%%%%%%%%%%%%%%%%%%%%%%%%%%%%%%%%%
\begin{frame}
\frametitle{Canonical matrices}

Canonical matrices have a single 1 amongst zeros, e.g. canonical columns
\begin{align*}
E^{(1)} =
\begin{pmatrix}
 1 \\
 0 \\
 0 \\
 \vdots \\
 0
\end{pmatrix}
, \quad
E^{(2)} =
\begin{pmatrix}
 0 \\
 1 \\
 0 \\
 \vdots \\
 0
\end{pmatrix}
, \quad \cdots
E^{(n)} =
\begin{pmatrix}
 0 \\
 0 \\
 0 \\
 \vdots \\
 1
\end{pmatrix}
\end{align*}

Or we can write this compactly (but maybe harder to understand) $\left( E^{(j)} \right)_{i}= \begin{cases} 1, & i=j \\ 0, & i\neq j \end{cases}$

We could similarly define canonical rows.
\end{frame}
%%%%%%%%%%%%%%%%%%%%%%%%%%%%%%%%%%%%%%%%%%%%%%%%%%%%%%%%%%%%%%%%%%%%%%%%



%%%%%%%%%%%%%%%%%%%%%%%%%%%%%%%%%%%%%%%%%%%%%%%%%%%%%%%%%%%%%%%%%%%%%%%%
\begin{frame}
\frametitle{Definition - Transpose}
The \textbf{transpose} of a matrix $A \in \mathcal{M}_{m,n}$ is another matrix $B \in \mathcal{M}_{m,n}$ with coefficients defined by
\begin{align*}
b_{ij} = a_{ji} \quad \text{for} \quad 1 \leq i \leq m, \, 1 \leq j \leq n
\end{align*}
In essence, the rows of $B$ are the columns of $A$. Similarly, the columns of $B$ are the rows of $A$. We denote the transpose $B = A^T$. Visually:
\begin{align*}
\begin{pmatrix}
a_{11} & a_{12} & \cdots & a_{1j} & \cdots & a_{1n} \\
a_{21} & a_{22} & \cdots & a_{2j} & \cdots & a_{2n} \\
\vdots & \vdots & \ddots & \vdots & \ddots & \vdots \\
a_{i1} & a_{12} & \cdots & a_{ij} & \cdots & a_{in} \\
\vdots & \vdots & \ddots & \vdots & \ddots & \vdots \\
a_{m1} & a_{m2} & \cdots & a_{mj} & \cdots & a_{mn}
\end{pmatrix}^T
=
\begin{pmatrix}
a_{11} & a_{21} & \cdots & a_{i1} & \cdots & a_{m1} \\
a_{12} & a_{22} & \cdots & a_{i2} & \cdots & a_{m2} \\
\vdots & \vdots & \ddots & \vdots & \ddots & \vdots \\
a_{1j} & a_{2j} & \cdots & a_{ij} & \cdots & a_{mj} \\
\vdots & \vdots & \ddots & \vdots & \ddots & \vdots \\
a_{1n} & a_{2n} & \cdots & a_{in} & \cdots & a_{mn}
\end{pmatrix}
\end{align*}
\end{frame}
%%%%%%%%%%%%%%%%%%%%%%%%%%%%%%%%%%%%%%%%%%%%%%%%%%%%%%%%%%%%%%%%%%%%%%%%







\ifnum \EXAMPLEVERSION = 1
%%%%%%%%%%%%%%%%%%%%%%%%%%%%%%%%%%%%%%%%%%%%%%%%%%%%%%%%%%%%%%%%%%%%%%%%
\begin{frame}
\frametitle{Examples - transpose}
\begin{align*}
\begin{pmatrix}
1 & 2 & 1 & 3 \\
0 & 8 & 9 & 1
\end{pmatrix}^T
=
\begin{pmatrix}
1 & 0 \\
2 & 8 \\ 
1 & 9 \\
3 & 1
\end{pmatrix}
\end{align*}
\begin{align*}
\begin{pmatrix}
1 \\
2 \\ 
3 
\end{pmatrix}^T
=
\begin{pmatrix}
1 & 2 & 3 
\end{pmatrix}
\end{align*}
\begin{align*}
\begin{pmatrix}
1 & 2 \\
0 & 8 
\end{pmatrix}^T
=
\begin{pmatrix}
1 & 0 \\
2 & 8 
\end{pmatrix}
\end{align*}
\end{frame}
%%%%%%%%%%%%%%%%%%%%%%%%%%%%%%%%%%%%%%%%%%%%%%%%%%%%%%%%%%%%%%%%%%%%%%%%
\fi 



\ifnum \EXAMPLEVERSION = 3
%%%%%%%%%%%%%%%%%%%%%%%%%%%%%%%%%%%%%%%%%%%%%%%%%%%%%%%%%%%%%%%%%%%%%%%%
\begin{frame}
\frametitle{Examples - Transpose}
\end{frame}
%%%%%%%%%%%%%%%%%%%%%%%%%%%%%%%%%%%%%%%%%%%%%%%%%%%%%%%%%%%%%%%%%%%%%%%%
\fi 

%%%%%%%%%%%%%%%%%%%%%%%%%%%%%%%%%%%%%%%%%%%%%%%%%%%%%%%%%%%%%%%%%%%%%%%%
\begin{frame}
\frametitle{Definition - Symmetric Matrices}

We call a \textit{square} matrix $A \in \mathcal{M}_{n,n}$ \textbf{symmetric} if it is equal to its transpose. That is, if
\begin{align*}
a_{ij} = a_{ji} \quad \text{for} \quad 1 \leq i \leq m, \, 1 \leq j \leq n
\end{align*}

\end{frame}
%%%%%%%%%%%%%%%%%%%%%%%%%%%%%%%%%%%%%%%%%%%%%%%%%%%%%%%%%%%%%%%%%%%%%%%%


\ifnum \EXAMPLEVERSION = 1
%%%%%%%%%%%%%%%%%%%%%%%%%%%%%%%%%%%%%%%%%%%%%%%%%%%%%%%%%%%%%%%%%%%%%%%%
\begin{frame}
\frametitle{Examples - Symmetric matrices}

\begin{align*}
&\begin{pmatrix}
1.5 & 3 & \pi \\
3   & 2 & 2 \\
\pi & 2 & 1
\end{pmatrix}^T
=
\begin{pmatrix}
1.5 & 3 & \pi \\
3   & 2 & 2 \\
\pi & 2 & 1
\end{pmatrix}
\quad \text{symmetric} \\
%
&\begin{pmatrix}
1.5 & 1 & \pi \\
3   & 2 & 2 \\
7 & 2 & 1
\end{pmatrix}^T
=
\begin{pmatrix}
1.5 & 3 & 7 \\
1   & 2 & 2 \\
\pi & 2 & 1
\end{pmatrix}
\quad \text{not symmetric}
\end{align*}
\end{frame}
%%%%%%%%%%%%%%%%%%%%%%%%%%%%%%%%%%%%%%%%%%%%%%%%%%%%%%%%%%%%%%%%%%%%%%%%
\fi 


\ifnum \EXAMPLEVERSION = 3
%%%%%%%%%%%%%%%%%%%%%%%%%%%%%%%%%%%%%%%%%%%%%%%%%%%%%%%%%%%%%%%%%%%%%%%%
\begin{frame}
\frametitle{Examples - Symmetric matrices}
\end{frame}
%%%%%%%%%%%%%%%%%%%%%%%%%%%%%%%%%%%%%%%%%%%%%%%%%%%%%%%%%%%%%%%%%%%%%%%%
\fi 





\section{Algebra of matrices}


%%%%%%%%%%%%%%%%%%%%%%%%%%%%%%%%%%%%%%%%%%%%%%%%%%%%%%%%%%%%%%%%%%%%%%%%
\begin{frame}
\frametitle{Definition - Matrix addition}
Given two matrices \textit{of the same shape}, $A,B \in \mathcal{M}_{m,n}$, we define the \textbf{addition}, $A+B$, to be a third matrix $C \in \mathcal{M}_{m,n}$ with coefficients given by:
\begin{align*}
c_{ij} = a_{ij} + b_{ij},
\end{align*}
that is, we add matrices \textit{coefficient by coefficient}.

\ifnum \EXAMPLEVERSION = 1 {
For example:
\begin{gather*}
\begin{pmatrix}
 1 & 3 \\
 7 & 2 
\end{pmatrix}
+
\begin{pmatrix}
 2 & 2 \\
-1 & 1 
\end{pmatrix}
=
\begin{pmatrix}
 1+2 & 3+2 \\
 7-1 & 2+1 
\end{pmatrix}
=
\begin{pmatrix}
 3 & 5 \\
 6 & 3 
\end{pmatrix}
%%%%%%%%
%%%%%%%%
\\
%%%%%%%%
%%%%%%%%
\begin{pmatrix}
 a & b \\
 c & d \\
 e & f 
\end{pmatrix}
+
\begin{pmatrix}
 2 & 2 \\
 1 & 1 \\
 0 & 3  
\end{pmatrix}
=
\begin{pmatrix}
 a+2 & b+2 \\
 c+1 & d+1 \\
 e   & f+3 
\end{pmatrix}
\end{gather*}
}
\fi

\ifnum \EXAMPLEVERSION = 2
\fi 

\ifnum \EXAMPLEVERSION = 3 {
For example:
\begin{gather*}
\begin{pmatrix}
 1 & 3 \\
 7 & 2 
\end{pmatrix}
+
\begin{pmatrix}
 2 & 2 \\
-1 & 1 
\end{pmatrix}
=
\begin{pmatrix}
 \quad & \quad & \quad & \quad & \quad \\
 \quad & \quad & \quad & \quad & \quad
\end{pmatrix}
=
\begin{pmatrix}
 \quad & \quad & \quad \\
 \quad & \quad & \quad
\end{pmatrix}
%%%%%%%%
%%%%%%%%
\\
%%%%%%%%
%%%%%%%%
\begin{pmatrix}
 a & b \\
 c & d \\
 e & f 
\end{pmatrix}
+
\begin{pmatrix}
 2 & 2 \\
 1 & 1 \\
 0 & 3  
\end{pmatrix}
=
\begin{pmatrix}
 \quad & \quad & \quad & \quad & \quad \\
 \quad & \quad & \quad & \quad & \quad \\
 \quad & \quad & \quad & \quad & \quad
\end{pmatrix}
\end{gather*}
}
\fi 

\end{frame}
%%%%%%%%%%%%%%%%%%%%%%%%%%%%%%%%%%%%%%%%%%%%%%%%%%%%%%%%%%%%%%%%%%%%%%%%





%%%%%%%%%%%%%%%%%%%%%%%%%%%%%%%%%%%%%%%%%%%%%%%%%%%%%%%%%%%%%%%%%%%%%%%%
\begin{frame}
\frametitle{Properties - Matrix addition}

\textbf{Remember}: to add two matrices they must be the same shape!

Given three matrices $A,B,C \in \mathcal{M}_{m,n}$:
\begin{itemize}
\item matrix addition is associative: $A + (B+C) = (A+B) + C$,

\item matrix addition is commutative: $A +B = B + A$.
\end{itemize}

\end{frame}
%%%%%%%%%%%%%%%%%%%%%%%%%%%%%%%%%%%%%%%%%%%%%%%%%%%%%%%%%%%%%%%%%%%%%%%%


%%%%%%%%%%%%%%%%%%%%%%%%%%%%%%%%%%%%%%%%%%%%%%%%%%%%%%%%%%%%%%%%%%%%%%%%
\begin{frame}
\frametitle{Definition - Scalar multiplication}
Given a number $k\in\mathbb{R}$ (called a scalar) and a matrix $A \in \mathcal{M}_{m,n}$, we define the \textbf{product}, $kA$, to be a matrix $B \in \mathcal{M}_{m,n}$ with coefficients given by:
\begin{align*}
b_{ij} = ka_{ij},
\end{align*}
that is, we multiply \textit{every coefficient} by the scalar. 

\ifnum \EXAMPLEVERSION = 1 {
For example:
\begin{gather*}
k
\begin{pmatrix}
 1 & 3 \\
 7 & 2 
\end{pmatrix}
=
\begin{pmatrix}
 k & 3k \\
 7k & 2k 
\end{pmatrix}
%%%%%%%%
%%%%%%%%
\\
%%%%%%%%
%%%%%%%%
\frac{1}{2}
\begin{pmatrix}
 2 & 2 \\
 1 & 1 \\
 0 & 3 
\end{pmatrix}
=
\begin{pmatrix}
 1 & 1 \\
 1/2 & 1/2 \\
 0 & 3/2 
\end{pmatrix}
\end{gather*}
}
\fi

\ifnum \EXAMPLEVERSION = 2
\fi 

\ifnum \EXAMPLEVERSION = 3 {
For example:
\begin{gather*}
k
\begin{pmatrix}
 1 & 3 \\
 7 & 2 
\end{pmatrix}
=
\begin{pmatrix}
 \quad & \quad & \quad & \quad \\
 \quad & \quad & \quad & \quad
\end{pmatrix}
%%%%%%%%
%%%%%%%%
\\
%%%%%%%%
%%%%%%%%
\frac{1}{2}
\begin{pmatrix}
 2 & 2 \\
 1 & 1 \\
 0 & 3 
\end{pmatrix}
=
\begin{pmatrix}
 \quad & \quad & \quad & \quad \\
 \quad & \quad & \quad & \quad \\
 \quad & \quad & \quad & \quad
\end{pmatrix}
\end{gather*}
}
\fi 



\end{frame}
%%%%%%%%%%%%%%%%%%%%%%%%%%%%%%%%%%%%%%%%%%%%%%%%%%%%%%%%%%%%%%%%%%%%%%%%


%%%%%%%%%%%%%%%%%%%%%%%%%%%%%%%%%%%%%%%%%%%%%%%%%%%%%%%%%%%%%%%%%%%%%%%%
\begin{frame}
\frametitle{Properties - Scalar multiplication}

For any $\alpha,\beta \in \mathbb{R}$ and matrices $A,B \in \mathcal{M}_{m,n}$ we have:


\begin{itemize}
\item Scalar multiplication is distributive: 
\begin{align*}
& \alpha(A + B) = \alpha A + \alpha B \\
& (\alpha + \beta)A = \alpha A + \beta A
\end{align*}

\item Scalar multiplication is associative: 
\begin{align*}
& \alpha(\beta A) = (\alpha \beta ) A
\end{align*}
\end{itemize}

\end{frame}
%%%%%%%%%%%%%%%%%%%%%%%%%%%%%%%%%%%%%%%%%%%%%%%%%%%%%%%%%%%%%%%%%%%%%%%%






%%%%%%%%%%%%%%%%%%%%%%%%%%%%%%%%%%%%%%%%%%%%%%%%%%%%%%%%%%%%%%%%%%%%%%%%
\begin{frame}
\frametitle{Definition - Multiplication of matrix by column}

Consider a matrix $A \in \mathcal{M}_{m,n}$ and a column $X \in \mathcal{M}_{m,1}$. We define the product $Y = AX$ to be the column in $\mathcal{M}_{m,1}$ with coefficients
\begin{align*}
y_i = a_{i1}x_1 + a_{i2}x_2 + \cdots + a_{in}x_n = \sum_{k=1}^n a_{ik} x_k 
\end{align*}

Visually
\begin{align*}
\begin{pmatrix}
y_{1} \\
y_{2} \\
\vdots \\
y_{m} 
\end{pmatrix}
%%%
=
%%%
\begin{pmatrix}
a_{11} & a_{12} & \cdots & a_{1n} \\
a_{21} & a_{22} & \cdots & a_{2n} \\
\vdots & \vdots & \ddots & \vdots \\
a_{m1} & a_{m2} & \cdots & a_{mn}
\end{pmatrix}
\begin{pmatrix}
x_{1} \\
x_{2} \\
\vdots \\
x_{n} 
\end{pmatrix}
%%%
&=
%%%
x_{1}
\begin{pmatrix}
a_{11} \\
a_{21} \\
\vdots \\
a_{m1} 
\end{pmatrix}
+
x_{2}
\begin{pmatrix}
a_{12} \\
a_{22} \\
\vdots \\
a_{m2} 
\end{pmatrix}
+ \cdots +
x_m
\begin{pmatrix}
a_{1n} \\
a_{2n} \\
\vdots \\
a_{mn} 
\end{pmatrix}
\\ \\
\implies Y &= x_{1}A^{(1)} + x_{2}A^{(2)} + \cdots + x_{n}A^{(n)}
\end{align*}


\end{frame}
%%%%%%%%%%%%%%%%%%%%%%%%%%%%%%%%%%%%%%%%%%%%%%%%%%%%%%%%%%%%%%%%%%%%%%%%




\ifnum \EXAMPLEVERSION = 1
%%%%%%%%%%%%%%%%%%%%%%%%%%%%%%%%%%%%%%%%%%%%%%%%%%%%%%%%%%%%%%%%%%%%%%%%
\begin{frame}
\frametitle{Examples - Multiplication of matrix by column - Method 1}

\begin{align*}
\begin{pmatrix}
 1 & 2 \\
 3 & 4 
\end{pmatrix}
\begin{pmatrix}
 5 \\
 6 
\end{pmatrix}
=
5
\begin{pmatrix}
 1 \\
 3 
\end{pmatrix}
+
6
\begin{pmatrix}
 2 \\
 4 
\end{pmatrix}
=
\begin{pmatrix}
 5 \\
 15 
\end{pmatrix}
+
\begin{pmatrix}
 12 \\
 24 
\end{pmatrix}
=
\begin{pmatrix}
 17 \\
 39 
\end{pmatrix}
\end{align*}



\begin{align*}
\begin{pmatrix}
 2  & 2 \\
 3  & 0  \\
 -1 & 4 
\end{pmatrix}
\begin{pmatrix}
 4 \\
 3 
\end{pmatrix}
=
4
\begin{pmatrix}
  2 \\
  3 \\
 -1 
\end{pmatrix}
+
3
\begin{pmatrix}
 2 \\
 0  \\
 4 
\end{pmatrix}
=
\begin{pmatrix}
  8 \\
 12 \\
 -4
\end{pmatrix}
+
\begin{pmatrix}
 6 \\
 0  \\
 12 
\end{pmatrix}
=
\begin{pmatrix}
 14 \\
 12  \\
  8 
\end{pmatrix}
\end{align*}

\end{frame}
%%%%%%%%%%%%%%%%%%%%%%%%%%%%%%%%%%%%%%%%%%%%%%%%%%%%%%%%%%%%%%%%%%%%%%%%
\fi 

\ifnum \EXAMPLEVERSION = 3
%%%%%%%%%%%%%%%%%%%%%%%%%%%%%%%%%%%%%%%%%%%%%%%%%%%%%%%%%%%%%%%%%%%%%%%%
\begin{frame}
\frametitle{Examples - Multiplication of matrix by column - Method 1}
\end{frame}
%%%%%%%%%%%%%%%%%%%%%%%%%%%%%%%%%%%%%%%%%%%%%%%%%%%%%%%%%%%%%%%%%%%%%%%%
\fi 




\ifnum \EXAMPLEVERSION = 1
%%%%%%%%%%%%%%%%%%%%%%%%%%%%%%%%%%%%%%%%%%%%%%%%%%%%%%%%%%%%%%%%%%%%%%%%
\begin{frame}
\frametitle{Example - Multiplication of matrix by column - Method 2}


\begin{center}
\begin{tikzpicture}
	% top row
	\matrix  at (0,1.5) (A)[matrix of math nodes,left delimiter=(, right delimiter=),
    ampersand replacement=\&, inner sep=1pt, column sep=6pt, row sep=6pt]
    { 
      1 \& 2  \\
      3 \& 4  \\
    };
	\matrix[right=0.6 of A] (X)[matrix of math nodes,left delimiter=(, right delimiter=),
    ampersand replacement=\&, inner sep=1pt, column sep=6pt, row sep=6pt]
    { 
      5 \\
      6 \\
    };
    \node[right=0.2 of X] (eq) {$=$};
	\matrix[right=0.2 of eq] (Y)[matrix of math nodes,left delimiter=(, right delimiter=),
    ampersand replacement=\&, inner sep=1pt, column sep=6pt, row sep=6pt]
    { 
      1\times 5 + 2\times 6  \\
      -  \\
    };
    \node[left=1 of A] {1st row};
    
    %% colour A matrix row - pink
    \scoped[on background layer]
    \node[fill=pink!50,inner xsep=0mm,inner ysep=0.6mm,
          fit=(A-1-1) (A-1-2)]   {};
    
    %% colour X column - green
    \scoped[on background layer]
    \node[fill=ForestGreen!30,inner xsep=0mm,
          fit=(X-1-1) (X-2-1)]   {};
          
    %% in solution, colour a11 - pink
    \scoped[on background layer]
    \node[fill=pink!50,inner xsep=-8.1mm, xshift=-8.3mm,inner ysep=0.6mm,
          fit=(Y-1-1) (Y-1-1)]   {};
          
    %% in solution, colour x1 - green
    \scoped[on background layer]
    \node[fill=ForestGreen!30,inner xsep=-8.1mm, xshift=-3.8mm,inner ysep=0.6mm,
          fit=(Y-1-1) (Y-1-1)]   {};
          
    %% in solution, colour a12 - pink
    \scoped[on background layer]
    \node[fill=pink!50,inner xsep=-8.1mm, xshift=+3.8mm,inner ysep=0.6mm,
          fit=(Y-1-1) (Y-1-1)]   {};
          
    %% in solution, colour x2 - green
    \scoped[on background layer]
    \node[fill=ForestGreen!30,inner xsep=-8.1mm, xshift=+8.3mm,inner ysep=0.6mm,
          fit=(Y-1-1) (Y-1-1)]   {};
    

	% bottom row
	\matrix  at (0,0) (A)[matrix of math nodes,left delimiter=(, right delimiter=),
    ampersand replacement=\&, inner sep=1pt, column sep=6pt, row sep=6pt]
    { 
      1 \& 2  \\
      3 \& 4  \\
    };
	\matrix[right=0.6 of A] (X)[matrix of math nodes,left delimiter=(, right delimiter=),
    ampersand replacement=\&, inner sep=1pt, column sep=6pt, row sep=6pt]
    { 
      5 \\
      6 \\
    };
    \node[right=0.2 of X] (eq) {$=$};
	\matrix[right=0.2 of eq] (Y)[matrix of math nodes,left delimiter=(, right delimiter=),
    ampersand replacement=\&, inner sep=1pt, column sep=6pt, row sep=6pt]
    { 
      - \\
      3\times 5 + 4\times 6   \\
    };
    \node[left=1 of A] {2nd row};
    
    
    
    %% colour A matrix row - blue
    \scoped[on background layer]
    \node[fill=airforceblue!50,inner xsep=0mm,inner ysep=0.6mm,
          fit=(A-2-1) (A-2-2)]   {};
    
    %% colour X column - green
    \scoped[on background layer]
    \node[fill=ForestGreen!30,inner xsep=0mm,
          fit=(X-1-1) (X-2-1)]   {};
          
    %% in solution, colour a11 - blue
    \scoped[on background layer]
    \node[fill=airforceblue!50,inner xsep=-8.1mm, xshift=-8.3mm,inner ysep=0.6mm,
          fit=(Y-2-1) (Y-2-1)]   {};
          
    %% in solution, colour x1 - green
    \scoped[on background layer]
    \node[fill=ForestGreen!30,inner xsep=-8.1mm, xshift=-3.8mm,inner ysep=0.6mm,
          fit=(Y-2-1) (Y-2-1)]   {};
          
    %% in solution, colour a12 - blue
    \scoped[on background layer]
    \node[fill=airforceblue!50,inner xsep=-8.1mm, xshift=+3.8mm,inner ysep=0.6mm,
          fit=(Y-2-1) (Y-2-1)]   {};
          
    %% in solution, colour x2 - green
    \scoped[on background layer]
    \node[fill=ForestGreen!30,inner xsep=-8.1mm, xshift=+8.3mm,inner ysep=0.6mm,
          fit=(Y-2-1) (Y-2-1)]   {};
        
\end{tikzpicture}
\end{center}

\centering
Therefore
\begin{align*}
\begin{pmatrix}
 1 & 2 \\
 3 & 4 
\end{pmatrix}
\begin{pmatrix}
 5 \\
 6 
\end{pmatrix}
=
\begin{pmatrix}
 17 \\
 39 
\end{pmatrix}
\end{align*}

\end{frame}
%%%%%%%%%%%%%%%%%%%%%%%%%%%%%%%%%%%%%%%%%%%%%%%%%%%%%%%%%%%%%%%%%%%%%%%%
\fi 



\ifnum \EXAMPLEVERSION = 3
%%%%%%%%%%%%%%%%%%%%%%%%%%%%%%%%%%%%%%%%%%%%%%%%%%%%%%%%%%%%%%%%%%%%%%%%
\begin{frame}
\frametitle{Examples - Multiplication of matrix by column - Method 2}
\end{frame}
%%%%%%%%%%%%%%%%%%%%%%%%%%%%%%%%%%%%%%%%%%%%%%%%%%%%%%%%%%%%%%%%%%%%%%%%
\fi 


%%%%%%%%%%%%%%%%%%%%%%%%%%%%%%%%%%%%%%%%%%%%%%%%%%%%%%%%%%%%%%%%%%%%%%%%
\begin{frame}
\frametitle{Properties}

To be able to multiply a matrix by a column, the matrix must have the same number of columns as the elements of the column.
\begin{align*}
\underbrace{A}_{(\colourboxed{Mahogany}{m},\colourboxed{airforceblue}{n})} \underbrace{X}_{(\colourboxed{airforceblue}{n},1)} = \underbrace{Y}_{(\colourboxed{Mahogany}{m},1)}
\end{align*}

For any $k\in\mathbb{R}$, matrices $A,B \in \mathcal{M}_{m,n}$ and columns $X,X' \in \mathcal{M}_{n,1}$ we have the following:

\begin{align*}
& \text{Distributivity:} \quad 
\begin{array}{l}
\alpha(A + B)X = AX + BX \\
A(X + X') = AX + AX'
\end{array}
\\ \\
& \text{Associativity:} \quad
k(AX) = (kA) X
\end{align*}

\end{frame}
%%%%%%%%%%%%%%%%%%%%%%%%%%%%%%%%%%%%%%%%%%%%%%%%%%%%%%%%%%%%%%%%%%%%%%%%


\ifnum \EXAMPLEVERSION = 1
%%%%%%%%%%%%%%%%%%%%%%%%%%%%%%%%%%%%%%%%%%%%%%%%%%%%%%%%%%%%%%%%%%%%%%%%
\begin{frame}
\frametitle{Exercise}
\centering
\begin{align*}
\text{Let} \quad 
A = \begin{pmatrix} 1 & 2 \\ 2 & 4 \end{pmatrix},
\quad
X = \begin{pmatrix} 3 \\ 1 \end{pmatrix},
\quad
Y = \begin{pmatrix} 1 \\ 2 \end{pmatrix}
\end{align*}
Calculate $AX$, $AY$ and $AX+AY$. \\


\begin{align*}
\text{Let} \quad 
A = \begin{pmatrix} 3 & 2 \\ 6 & 4 \end{pmatrix},
\quad
B = \begin{pmatrix} 5 & 0 \\ 2 & 8 \end{pmatrix},
\quad
X = \begin{pmatrix} 1 \\ 1 \end{pmatrix}
\end{align*}
Calculate $AX$, $BX$ and $AX+BX$.
\end{frame}
%%%%%%%%%%%%%%%%%%%%%%%%%%%%%%%%%%%%%%%%%%%%%%%%%%%%%%%%%%%%%%%%%%%%%%%%
\fi 


\ifnum \EXAMPLEVERSION = 3
%%%%%%%%%%%%%%%%%%%%%%%%%%%%%%%%%%%%%%%%%%%%%%%%%%%%%%%%%%%%%%%%%%%%%%%%
\begin{frame}
\frametitle{Exercise}
\begin{minipage}{0.5\textwidth}
\begin{align*}
\text{Let} \quad 
A = \begin{pmatrix} 1 & 2 \\ 2 & 4 \end{pmatrix},
\quad
X = \begin{pmatrix} 3 \\ 1 \end{pmatrix},
\quad
Y = \begin{pmatrix} 1 \\ 2 \end{pmatrix}
\end{align*}
Calculate $AX$, $AY$ and $AX+AY$. \\


\begin{align*}
\text{Let} \quad 
A = \begin{pmatrix} 3 & 2 \\ 6 & 4 \end{pmatrix},
\quad
B = \begin{pmatrix} 5 & 0 \\ 2 & 8 \end{pmatrix},
\quad
X = \begin{pmatrix} 1 \\ 1 \end{pmatrix}
\end{align*}
Calculate $AX$, $BX$ and $AX+BX$.
\end{minipage}
\end{frame}
%%%%%%%%%%%%%%%%%%%%%%%%%%%%%%%%%%%%%%%%%%%%%%%%%%%%%%%%%%%%%%%%%%%%%%%%
\fi 


%%%%%%%%%%%%%%%%%%%%%%%%%%%%%%%%%%%%%%%%%%%%%%%%%%%%%%%%%%%%%%%%%%%%%%%%
\begin{frame}
\frametitle{Multiplication of two matrices}
Suppose we have two matrices $A$ and $B$. We would like to define the matrix multiplication $AB=C$ so that the following associative law holds:
\begin{align*}
CX = (AB)X = A(BX)
\end{align*}
for a column $X$ of an appropriate size.

\begin{minipage}{0.6\textwidth}
Let $B\in \mathcal{M}_{m,n}$. This forces the size of $X$: $X\in\mathcal{M}_{n,1}$ \\
and we have a new vector $Y\in\mathcal{M}_{m,1}$. 

Matrix $A$ must have the same number of columns as the elements of the column it multiplies, so let $A\in\mathcal{M}_{q,m}$ which gives a new vector $Z\in\mathcal{M}_{q,1}$. 

Finally we must have $CX = Z$. This forces $C\in\mathcal{M}_{q,n}$.

\end{minipage}\hspace{1cm}
\begin{minipage}{0.3\textwidth}
$\underbrace{B}_{(m,n)} \underbrace{X}_{(n,1)} = \underbrace{Y}_{(m,1)}$ 
\\
\vspace{0.1cm}
\\
$\underbrace{A}_{(q,m)} \underbrace{Y}_{(m,1)} = \underbrace{Z}_{(q,1)}$
\\
\vspace{0.1cm}
\\
$\underbrace{C}_{(q,n)} \underbrace{X}_{(n,1)} = \underbrace{Z}_{(q,1)}$
\end{minipage}

\end{frame}
%%%%%%%%%%%%%%%%%%%%%%%%%%%%%%%%%%%%%%%%%%%%%%%%%%%%%%%%%%%%%%%%%%%%%%%%



%%%%%%%%%%%%%%%%%%%%%%%%%%%%%%%%%%%%%%%%%%%%%%%%%%%%%%%%%%%%%%%%%%%%%%%%
\begin{frame}
\frametitle{Multiplication of two matrices}
So, \textit{in order that the multiplication is well defined}, the number of columns of the left matrix must match the number rows of the right matrix.That is:
\begin{align*}
\underbrace{A}_{(\colorbox{Mahogany!20}{q},\colorbox{airforceblue!20}{m})} \underbrace{B}_{(\colorbox{airforceblue!20}{m},\colorbox{ForestGreen!30}{n})} = \underbrace{C}_{(\colorbox{Mahogany!20}{q},\colorbox{ForestGreen!30}{n})}
\end{align*}

I say ``the \textit{inner} indices of $A$ and $B$ must match and their \textit{outer} indices gives the shape of the result''. For example
\begin{gather*}
\begin{pmatrix}
 1 & 2 & 1 \\
 3 & 4 & 2
\end{pmatrix}
\begin{pmatrix}
 1 & 2 &  0 \\
 2 & 1 & 10 \\
 2 & 3 &  1
\end{pmatrix}
\quad \text{well defined} \\
\begin{pmatrix}
 1 & 2 \\
 3 & 4 
\end{pmatrix}
\begin{pmatrix}
 1 & 2 \\
 2 & 1 \\
 2 & 3 
\end{pmatrix}
\quad \text{not defined}
\end{gather*}
\end{frame}
%%%%%%%%%%%%%%%%%%%%%%%%%%%%%%%%%%%%%%%%%%%%%%%%%%%%%%%%%%%%%%%%%%%%%%%%

%%%%%%%%%%%%%%%%%%%%%%%%%%%%%%%%%%%%%%%%%%%%%%%%%%%%%%%%%%%%%%%%%%%%%%%%
\begin{frame}
\frametitle{Definition - Multiplication of two matrices}
Consider two matrices $A \in \mathcal{M}_{m,n}$ and $B \in \mathcal{M}_{n,q}$. We define the product $AB$ to be the matrix $C\in\mathcal{M}_{m,q}$ with coefficients
\begin{gather*}
c_{ij} = a_{i1}b_{1j} + a_{i2}b_{2j} + \cdots + a_{in}b_{nj} = \sum_{k=1}^n a_{ik} b_{kj} \\
%%%
%%%
%%%
\implies
\begin{pmatrix}
a_{11} & a_{12} & \cdots & a_{1n} \\
a_{21} & a_{22} & \cdots & a_{2n} \\
\vdots & \vdots & \ddots & \vdots \\
a_{m1} & a_{m2} & \cdots & a_{mn}
\end{pmatrix}
\begin{pmatrix}
b_{11} & b_{12} & \cdots & b_{1q} \\
b_{21} & b_{22} & \cdots & b_{2q} \\
\vdots & \vdots & \ddots & \vdots \\
b_{n1} & b_{n2} & \cdots & b_{nq}
\end{pmatrix} \\
%%%
%%%
%%%
=
\left(
b_{11}
\begin{pmatrix}
a_{11} \\
a_{21} \\
\vdots \\
a_{m1} 
\end{pmatrix}
+ \cdots +
b_{n1}
\begin{pmatrix}
a_{1n} \\
a_{2n} \\
\vdots \\
a_{mn} 
\end{pmatrix}
%%
\quad \cdots \quad 
%%
b_{1q}
\begin{pmatrix}
a_{11} \\
a_{21} \\
\vdots \\
a_{m1} 
\end{pmatrix}
+ \cdots +
b_{nq}
\begin{pmatrix}
a_{1n} \\
a_{2n} \\
\vdots \\
a_{mn} 
\end{pmatrix}
\right)
\end{gather*}

\end{frame}
%%%%%%%%%%%%%%%%%%%%%%%%%%%%%%%%%%%%%%%%%%%%%%%%%%%%%%%%%%%%%%%%%%%%%%%%




\ifnum \EXAMPLEVERSION = 1
%%%%%%%%%%%%%%%%%%%%%%%%%%%%%%%%%%%%%%%%%%%%%%%%%%%%%%%%%%%%%%%%%%%%%%%%
\begin{frame}
\frametitle{Example - Method 1}



\begin{align*}
\begin{pNiceMatrix}
\Block[fill=Purple!20,rounded-corners]{3-1}{} 1 & \Block[fill=Mahogany!20,rounded-corners]{3-1}{} 2 \\
  3 & 0  \\
 -1 & 4 
\end{pNiceMatrix}
\begin{pmatrix}
  \colorbox{airforceblue!30}{1} & \colorbox{pink!30}{-2} \\
  \colorbox{ForestGreen!30}{3} &  \colorbox{BurntOrange!30}{2} 
\end{pmatrix}
%%%
%%%
&=
\left(
\colorbox{airforceblue!30}{1}
\begin{pNiceMatrix}
  \Block[fill=Purple!20,rounded-corners]{3-1}{} 1 \\
  3 \\
 -1 
\end{pNiceMatrix}
+
\colorbox{ForestGreen!30}{3}
\begin{pNiceMatrix}
\Block[fill=Mahogany!20,rounded-corners]{3-1}{} 2 \\
  0  \\
  4 
\end{pNiceMatrix}
\quad 
\colorbox{pink!30}{-2}
\begin{pNiceMatrix}
  \Block[fill=Purple!20,rounded-corners]{3-1}{} 1 \\
  3 \\
 -1 
\end{pNiceMatrix}
+
\colorbox{BurntOrange!30}{2}
\begin{pNiceMatrix}
\Block[fill=Mahogany!20,rounded-corners]{3-1}{} 2 \\
  0  \\
  4 
\end{pNiceMatrix}
\right)\\
%%%
%%%
&=
\begin{pmatrix}
  5 &  2 \\
  3 & -6 \\
 11 & 10
\end{pmatrix}
\end{align*}

%why not work on home computer?

\end{frame}
%%%%%%%%%%%%%%%%%%%%%%%%%%%%%%%%%%%%%%%%%%%%%%%%%%%%%%%%%%%%%%%%%%%%%%%%
\fi 


\ifnum \EXAMPLEVERSION = 3
%%%%%%%%%%%%%%%%%%%%%%%%%%%%%%%%%%%%%%%%%%%%%%%%%%%%%%%%%%%%%%%%%%%%%%%%
\begin{frame}
\frametitle{Example - Method 1}
\end{frame}
%%%%%%%%%%%%%%%%%%%%%%%%%%%%%%%%%%%%%%%%%%%%%%%%%%%%%%%%%%%%%%%%%%%%%%%%
\fi 



\ifnum \EXAMPLEVERSION = 1
%%%%%%%%%%%%%%%%%%%%%%%%%%%%%%%%%%%%%%%%%%%%%%%%%%%%%%%%%%%%%%%%%%%%%%%%
\begin{frame}
\frametitle{Example - Method 2}
\begin{center}
\begin{tikzpicture}
	% top row
	\matrix  at (0,2) (A1)[matrix of math nodes,left delimiter=(, right delimiter=),
    ampersand replacement=\&, inner sep=1pt, column sep=6pt, row sep=6pt]
    { 
      1 \&  2 \& 4  \\
      3 \& -1 \& 0  \\
    };
    
	\matrix[right=0.6 of A1] (B1)[matrix of math nodes,left delimiter=(, right delimiter=),
    ampersand replacement=\&, inner sep=1pt, column sep=6pt, row sep=6pt]
    { 
      2 \&  5  \\
      1 \&  3  \\
     -1 \& -4  \\
    };
    
    \node[right=0.2 of B1] (eq) {$=$};
	\matrix[right=0.2 of eq] (C1)[matrix of math nodes,left delimiter=(, right delimiter=),
    ampersand replacement=\&, inner sep=1pt, column sep=15pt, row sep=6pt]
    { 
      1\times 2 + 2\times 1 + 4\times -1 \& 1\times 5 + 2\times 3 + 4\times -4  \\
      - \& -  \\
    };
    \node[left=1 of A1] {1st row};
    
    %% colour A matrix row - pink
    \scoped[on background layer]
    \node[fill=pink!50,inner xsep=0mm,inner ysep=0.6mm,
          fit=(A1-1-1) (A1-1-3)]   {};
    
    %% colour X column - green
    \scoped[on background layer]
    \node[fill=ForestGreen!30,inner xsep=0mm,
          fit=(B1-1-1) (B1-3-1)]   {};
    
    %% colour X column - BurntOrange
    \scoped[on background layer]
    \node[fill=BurntOrange!30,inner xsep=0mm,
          fit=(B1-1-2) (B1-3-2)]   {};
          
    %%%%% in solution C11
    %colour A11 - pink
    \scoped[on background layer]
    \node[fill=pink!50,inner xsep=-15.6mm, xshift=-15.7mm,inner ysep=0.6mm,
          fit=(C1-1-1) (C1-1-1)]   {};    
    %colour B11 - green
    \scoped[on background layer]
    \node[fill=ForestGreen!30,inner xsep=-15.6mm, xshift=-11.27mm,inner ysep=0.6mm,
          fit=(C1-1-1) (C1-1-1)]   {};
    %colour A12 - pink
    \scoped[on background layer]
    \node[fill=pink!50,inner xsep=-15.6mm, xshift=-3.6mm,inner ysep=0.6mm,
          fit=(C1-1-1) (C1-1-1)]   {};    
    %colour B21 - green
    \scoped[on background layer]
    \node[fill=ForestGreen!30,inner xsep=-15.6mm, xshift=0.8mm,inner ysep=0.6mm,
          fit=(C1-1-1) (C1-1-1)]   {};
    %colour A13 - pink
    \scoped[on background layer]
    \node[fill=pink!50,inner xsep=-15.6mm, xshift=+8.5mm,inner ysep=0.6mm,
          fit=(C1-1-1) (C1-1-1)]   {};    
    %colour B31 - green
    \scoped[on background layer]
    \node[fill=ForestGreen!30,inner xsep=-14.0mm, xshift=+14.55mm,inner ysep=0.6mm,
          fit=(C1-1-1) (C1-1-1)]   {};
    %%%%%
          
    %%%%% in solution C12
    %colour A11 - pink
    \scoped[on background layer]
    \node[fill=pink!50,inner xsep=-15.6mm, xshift=-15.7mm,inner ysep=0.6mm,
          fit=(C1-1-2) (C1-1-2)]   {};    
    %colour B11 - BurntOrange
    \scoped[on background layer]
    \node[fill=BurntOrange!30,inner xsep=-15.6mm, xshift=-11.27mm,inner ysep=0.6mm,
          fit=(C1-1-2) (C1-1-2)]   {};
    %colour A12 - pink
    \scoped[on background layer]
    \node[fill=pink!50,inner xsep=-15.6mm, xshift=-3.6mm,inner ysep=0.6mm,
          fit=(C1-1-2) (C1-1-2)]   {};    
    %colour B21 - BurntOrange
    \scoped[on background layer]
    \node[fill=BurntOrange!30,inner xsep=-15.6mm, xshift=0.8mm,inner ysep=0.6mm,
          fit=(C1-1-2) (C1-1-2)]   {};
    %colour A13 - pink
    \scoped[on background layer]
    \node[fill=pink!50,inner xsep=-15.6mm, xshift=+8.5mm,inner ysep=0.6mm,
          fit=(C1-1-2) (C1-1-2)]   {};    
    %colour B31 - BurntOrange
    \scoped[on background layer]
    \node[fill=BurntOrange!30,inner xsep=-14.0mm, xshift=+14.55mm,inner ysep=0.6mm,
          fit=(C1-1-2) (C1-1-2)]   {};
    %%%%%
    

	% bottom row
	\matrix  at (0,0) (A2)[matrix of math nodes,left delimiter=(, right delimiter=),
    ampersand replacement=\&, inner sep=1pt, column sep=6pt, row sep=6pt]
    { 
      1 \&  2 \& 4  \\
      3 \& -1 \& 0  \\
    };
    
	\matrix[right=0.6 of A2] (B2)[matrix of math nodes,left delimiter=(, right delimiter=),
    ampersand replacement=\&, inner sep=1pt, column sep=6pt, row sep=6pt]
    { 
      2 \&  5  \\
      1 \&  3  \\
     -1 \& -4  \\
    };
    
    \node[right=0.2 of B2] (eq) {$=$};
    
	\matrix[right=0.2 of eq] (C2)[matrix of math nodes,left delimiter=(, right delimiter=),
    ampersand replacement=\&, inner sep=1pt, column sep=15pt, row sep=6pt]
    { 
      - \& - \\
      3\times 2 -1\times 1 + 0\times -1 \& 3\times 5 -1\times 3 + 0\times -4  \\
    };
    \node[left=1 of A2] {2nd row};
    
    
    
    %% colour A matrix row - blue
    \scoped[on background layer]
    \node[fill=airforceblue!50,inner xsep=0mm,inner ysep=0.6mm,
          fit=(A2-2-1) (A2-2-3)]   {};
    
    %% colour X column - green
    \scoped[on background layer]
    \node[fill=ForestGreen!30,inner xsep=0mm,
          fit=(B2-1-1) (B2-3-1)]   {};
    
    %% colour X column - BurntOrange
    \scoped[on background layer]
    \node[fill=BurntOrange!30,inner xsep=0mm,
          fit=(B2-1-2) (B2-3-2)]   {};
          
          
    %% in solution C11
    %colour A11 - airforceblue
    \scoped[on background layer]
    \node[fill=airforceblue!50,inner xsep=-15.6mm, xshift=-15.7mm,inner ysep=0.6mm,
          fit=(C2-2-1) (C2-2-1)]   {};    
    %colour B11 - green
    \scoped[on background layer]
    \node[fill=ForestGreen!30,inner xsep=-15.6mm, xshift=-11.27mm,inner ysep=0.6mm,
          fit=(C2-2-1) (C2-2-1)]   {};
    %colour A12 - airforceblue
    \scoped[on background layer]
    \node[fill=airforceblue!50,inner xsep=-15.6mm, xshift=-3.6mm,inner ysep=0.6mm,
          fit=(C2-2-1) (C2-2-1)]   {};    
    %colour B21 - green
    \scoped[on background layer]
    \node[fill=ForestGreen!30,inner xsep=-15.6mm, xshift=0.8mm,inner ysep=0.6mm,
          fit=(C2-2-1) (C2-2-1)]   {};
    %colour A13 - airforceblue
    \scoped[on background layer]
    \node[fill=airforceblue!50,inner xsep=-15.6mm, xshift=+8.5mm,inner ysep=0.6mm,
          fit=(C2-2-1) (C2-2-1)]   {};    
    %colour B31 - green
    \scoped[on background layer]
    \node[fill=ForestGreen!30,inner xsep=-14.0mm, xshift=+14.55mm,inner ysep=0.6mm,
          fit=(C2-2-1) (C2-2-1)]   {};
    %%%%%%%%%%
          
    %%%%% in solution C22
    %colour A21 - airforceblue
    \scoped[on background layer]
    \node[fill=airforceblue!50,inner xsep=-15.6mm, xshift=-15.7mm,inner ysep=0.6mm,
          fit=(C2-2-2) (C2-2-2)]   {};    
    %colour B12 - BurntOrange
    \scoped[on background layer]
    \node[fill=BurntOrange!30,inner xsep=-15.6mm, xshift=-11.27mm,inner ysep=0.6mm,
          fit=(C2-2-2) (C2-2-2)]   {};
    %colour A22 - airforceblue
    \scoped[on background layer]
    \node[fill=airforceblue!50,inner xsep=-15.6mm, xshift=-3.6mm,inner ysep=0.6mm,
          fit=(C2-2-2) (C2-2-2)]   {};    
    %colour B22 - BurntOrange
    \scoped[on background layer]
    \node[fill=BurntOrange!30,inner xsep=-15.6mm, xshift=0.8mm,inner ysep=0.6mm,
          fit=(C2-2-2) (C2-2-2)]   {};
    %colour A23 - airforceblue
    \scoped[on background layer]
    \node[fill=airforceblue!50,inner xsep=-15.6mm, xshift=+8.5mm,inner ysep=0.6mm,
          fit=(C2-2-2) (C2-2-2)]   {};    
    %colour B32 - BurntOrange
    \scoped[on background layer]
    \node[fill=BurntOrange!30,inner xsep=-14.0mm, xshift=+14.55mm,inner ysep=0.6mm,
          fit=(C2-2-2) (C2-2-2)]   {};
    %%%%%
\end{tikzpicture}
\end{center}

\centering
Therefore
\begin{align*}
\begin{pmatrix}
      1 &  2 & 4  \\
      3 & -1 & 0  \\
\end{pmatrix}
\begin{pmatrix}
      2 &  5  \\
      1 &  3  \\
     -1 & -4  \\
\end{pmatrix}
=
\begin{pmatrix}
 0 & -5 \\
 5 & 12
\end{pmatrix}
\end{align*}

\end{frame}
%%%%%%%%%%%%%%%%%%%%%%%%%%%%%%%%%%%%%%%%%%%%%%%%%%%%%%%%%%%%%%%%%%%%%%%%
\fi 



\ifnum \EXAMPLEVERSION = 3
%%%%%%%%%%%%%%%%%%%%%%%%%%%%%%%%%%%%%%%%%%%%%%%%%%%%%%%%%%%%%%%%%%%%%%%%
\begin{frame}
\frametitle{Example - Method 2}
\end{frame}
%%%%%%%%%%%%%%%%%%%%%%%%%%%%%%%%%%%%%%%%%%%%%%%%%%%%%%%%%%%%%%%%%%%%%%%%
\fi 


\ifnum \EXAMPLEVERSION = 1
%%%%%%%%%%%%%%%%%%%%%%%%%%%%%%%%%%%%%%%%%%%%%%%%%%%%%%%%%%%%%%%%%%%%%%%%
\begin{frame}
\frametitle{Example - Method 3}
\vspace{-0.7cm}
\begin{align*}
\text{Let } A =
\begin{pmatrix}
      a_{11} & a_{12}  \\
      a_{21} & a_{22}  \\
      a_{31} & a_{32}
\end{pmatrix},
\quad\quad
 B =
\begin{pmatrix}
      b_{11} & b_{12}  & b_{13} \\
      b_{21} & b_{22}  & b_{23}
\end{pmatrix}
\quad \text{and} \quad C = AB
\end{align*}

\begin{center}
\begin{tikzpicture}
	% top row
	\matrix  at (0,0) (C)[matrix of math nodes,left delimiter=(, right delimiter=),
    ampersand replacement=\&, inner sep=1pt, column sep=6pt, row sep=6pt]
    { 
      c_{11} \& c_{12} \& c_{13}  \\
      c_{21} \& c_{22} \& c_{23}  \\
      c_{31} \& c_{32} \& c_{33}  \\
    };
    
	\matrix[left=1 of C] (A)[matrix of math nodes,left delimiter=(, right delimiter=),
    ampersand replacement=\&, inner sep=1pt, column sep=6pt, row sep=6pt]
    { 
      a_{11} \& a_{12}  \\
      a_{21} \& a_{22}  \\
      a_{31} \& a_{32}  \\
    };
    
	\matrix[above=0.8 of C] (B)[matrix of math nodes,left delimiter=(, right delimiter=),
    ampersand replacement=\&, inner sep=1pt, column sep=6, row sep=6pt]
    { 
      b_{11} \& b_{12} \& b_{13}  \\
      b_{21} \& b_{22} \& b_{23}  \\
    };
    

    %colour A row 1 - airforceblue
    \scoped[on background layer]
    \node[fill=airforceblue!30,inner xsep=-0.1mm,inner ysep=+0.5mm,
          fit=(A-1-1) (A-1-2)]   {}; 

    %colour A row 3 - forestgreen
    \scoped[on background layer]
    \node[fill=ForestGreen!30,inner xsep=-0.1mm,inner ysep=+0.5mm,
          fit=(A-3-1) (A-3-2)]   {}; 
          

    %colour B column 2 - airforceblue
    \scoped[on background layer]
    \node[fill=airforceblue!30,inner xsep=-0.1mm,inner ysep=+0.5mm,
          fit=(B-1-2) (B-2-2)]   {}; 

    %colour B column 3 - forestgreen
    \scoped[on background layer]
    \node[fill=ForestGreen!30,inner xsep=-0.1mm,inner ysep=+0.5mm,
          fit=(B-1-3) (B-2-3)]   {}; 
          

    %colour C row 1 column 2 - airforceblue
    \scoped[on background layer]
    \node[fill=airforceblue!30,inner xsep=-0.1mm,inner ysep=+0.5mm,
          fit=(C-1-2) (C-1-2)]   {}; 

    %colour C row 3 column 3 - forestgreen
    \scoped[on background layer]
    \node[fill=ForestGreen!30,inner xsep=-0.1mm,inner ysep=+0.5mm,
          fit=(C-3-3) (C-3-3)]   {}; 
    %%%%%
    \draw[->,line width=1mm,airforceblue!30] (-2,0.5)--(-0.4,0.5);
    \draw[->,line width=1mm,airforceblue!30] (0.0,1.4)--(0.0,0.8);
    \draw[->,line width=1mm,ForestGreen!30] (-2,-0.5)--(0.4,-0.5);
    \draw[->,line width=1mm,ForestGreen!30] (0.7,1.4)--(0.7,-0.2);
\end{tikzpicture}
\end{center}

The $i,j$ element of $C=AB$ comes from multiplying i$^{th}$ row of $A$ by the j$^{th}$ column of $B$: 
\begin{align*}
c_{ij} = A_{(i)}B^{(j)}
\end{align*}

\end{frame}
%%%%%%%%%%%%%%%%%%%%%%%%%%%%%%%%%%%%%%%%%%%%%%%%%%%%%%%%%%%%%%%%%%%%%%%%
\fi 


\ifnum \EXAMPLEVERSION = 3
%%%%%%%%%%%%%%%%%%%%%%%%%%%%%%%%%%%%%%%%%%%%%%%%%%%%%%%%%%%%%%%%%%%%%%%%
\begin{frame}
\frametitle{Example - Method 3}
\end{frame}
%%%%%%%%%%%%%%%%%%%%%%%%%%%%%%%%%%%%%%%%%%%%%%%%%%%%%%%%%%%%%%%%%%%%%%%%
\fi 



%%%%%%%%%%%%%%%%%%%%%%%%%%%%%%%%%%%%%%%%%%%%%%%%%%%%%%%%%%%%%%%%%%%%%%%%
\begin{frame}
\frametitle{Properties - Matrix multiplication}

Given matrices $A,A' \in \mathcal{M}_{m,n}$, $B,B' \in \mathcal{M}_{n,q}$ and constant $k\in\mathbb{R}$ we have the following properties
\begin{itemize}
\item $A(B + B')=AB + AB'$ and $(A + A')B=AB + A'B$
\item $A(kB)=k(AB)= (kA)B$
\item $(AB)^T= B^T A^T$
\end{itemize}
\end{frame}
%%%%%%%%%%%%%%%%%%%%%%%%%%%%%%%%%%%%%%%%%%%%%%%%%%%%%%%%%%%%%%%%%%%%%%%%



%%%%%%%%%%%%%%%%%%%%%%%%%%%%%%%%%%%%%%%%%%%%%%%%%%%%%%%%%%%%%%%%%%%%%%%%
\begin{frame}
\frametitle{Definition - Identity matrix}
The identity matrix, $I_k$, is a square matrix satisfying: $I_n A = A = A I_m$ for any $A\in\mathcal{M}_{m,n}$. $I_k$ is a $k\times k$ square matrix with 1s on the diagonal and 0s everywhere else:
\begin{align*}
(I_k)_{ij} = 
\begin{cases}
1, & \text{when } i=j\\
0, & \text{when } i\neq j.
\end{cases}
\end{align*}
For example
\begin{align*}
I_3 =
\begin{pmatrix}
1 & 0 & 0 \\
0 & 1 & 0 \\
0 & 0 & 1
\end{pmatrix}
\quad \quad
I_5 =
\begin{pmatrix}
1 & 0 & 0 & 0 & 0 \\
0 & 1 & 0 & 0 & 0 \\
0 & 0 & 1 & 0 & 0 \\
0 & 0 & 0 & 1 & 0 \\
0 & 0 & 0 & 0 & 1
\end{pmatrix}
\end{align*}
\end{frame}
%%%%%%%%%%%%%%%%%%%%%%%%%%%%%%%%%%%%%%%%%%%%%%%%%%%%%%%%%%%%%%%%%%%%%%%%





%%%%%%%%%%%%%%%%%%%%%%%%%%%%%%%%%%%%%%%%%%%%%%%%%%%%%%%%%%%%%%%%%%%%%%%%
\begin{frame}
\frametitle{Properties}
The multiplication of two matrices can give the null matrix, even if both matrices are non-null. For example

\begin{align*}
\begin{pmatrix}
1 & 2 & 0 \\
3 & 0 & 0
\end{pmatrix}
\begin{pmatrix}
0 & 0 & 0 \\
0 & 0 & 0 \\
2 & 3 & 1
\end{pmatrix}
=
\begin{pmatrix}
0 & 0 & 0 \\
0 & 0 & 0
\end{pmatrix}
\end{align*}
\end{frame}
%%%%%%%%%%%%%%%%%%%%%%%%%%%%%%%%%%%%%%%%%%%%%%%%%%%%%%%%%%%%%%%%%%%%%%%%





%%%%%%%%%%%%%%%%%%%%%%%%%%%%%%%%%%%%%%%%%%%%%%%%%%%%%%%%%%%%%%%%%%%%%%%%
\begin{frame}
\frametitle{Properties}
Given matrices $A\in\mathcal{M}_{m,n}$ and $B\in\mathcal{M}_{n,q}$:
\begin{itemize}
\item The product $C=AB$ is a matrix of size $(m,q)$.

\item If we want to define the product $C'=BA$, what must be true?
	\begin{itemize}
	\item $C'$ must be a square matrix of size $(n,n)$
	\item $C$ must be a square matrix of size $(m,m)=(q,q)$
	\item If $n \neq m$, then $C$ and $C'$ have different sizes, and we say that $C$ and $C'$ are not compatible.
	\item If $n = m$, the products $AB$ and $BA$ are the same size, but are not necessarily the same element by element. Matrix multiplication \textit{is not generally commutative}.
	\end{itemize}
\end{itemize}
\end{frame}
%%%%%%%%%%%%%%%%%%%%%%%%%%%%%%%%%%%%%%%%%%%%%%%%%%%%%%%%%%%%%%%%%%%%%%%%





\ifnum \EXAMPLEVERSION = 1
%%%%%%%%%%%%%%%%%%%%%%%%%%%%%%%%%%%%%%%%%%%%%%%%%%%%%%%%%%%%%%%%%%%%%%%%
\begin{frame}
\frametitle{Example - Non-commutative multiplication}\centering
Consider
\begin{align*}
A=
\begin{pmatrix}
1 & 2 \\
3 & 4
\end{pmatrix}
\qquad
%%
B=
\begin{pmatrix}
1 & -1 \\
2 &  3
\end{pmatrix}
\end{align*}
Then we have the multiplications
\begin{align*}
& AB=
\begin{pmatrix}
1 & 2 \\
3 & 4
\end{pmatrix}
\begin{pmatrix}
1 & -1 \\
2 &  3
\end{pmatrix}
%%%
=
\begin{pmatrix}
  5 &  5 \\
 11 &  9
\end{pmatrix}
\\
%%%%
%%%%
& BA=
\begin{pmatrix}
1 & -1 \\
2 &  3
\end{pmatrix}
\begin{pmatrix}
1 & 2 \\
3 & 4
\end{pmatrix}
%%%
=
\begin{pmatrix}
 -2 & -2 \\
 11 & 16
\end{pmatrix}
\end{align*}

Therefore $AB \neq BA$.
\end{frame}
%%%%%%%%%%%%%%%%%%%%%%%%%%%%%%%%%%%%%%%%%%%%%%%%%%%%%%%%%%%%%%%%%%%%%%%%
\fi


\ifnum \EXAMPLEVERSION = 3
%%%%%%%%%%%%%%%%%%%%%%%%%%%%%%%%%%%%%%%%%%%%%%%%%%%%%%%%%%%%%%%%%%%%%%%%
\begin{frame}
\frametitle{Example - Non-commutative matrix multiplication}
\end{frame}
%%%%%%%%%%%%%%%%%%%%%%%%%%%%%%%%%%%%%%%%%%%%%%%%%%%%%%%%%%%%%%%%%%%%%%%%
\fi 


%%%%%%%%%%%%%%%%%%%%%%%%%%%%%%%%%%%%%%%%%%%%%%%%%%%%%%%%%%%%%%%%%%%%%%%%
\begin{frame}
\frametitle{Multiplication by diagonal matrices}
Given two square matrices $A,\Lambda \in\mathcal{M}_{m,m}$ where $\Lambda$ is diagonal:
\begin{align*}
\Lambda A = 
\begin{pmatrix}
\lambda_1 &    0      & \cdots & 0 \\
0         & \lambda_2 & \cdots & 0 \\
\vdots    & \vdots    & \ddots & \vdots \\
0         &    0      & \cdots & \lambda_m
\end{pmatrix}
\begin{pmatrix}
a_{11} & a_{12} & \cdots & a_{1m} \\
a_{21} & a_{22} & \cdots & a_{2m} \\
\vdots & \vdots & \ddots & \vdots \\
a_{m1} & a_{m2} & \cdots & a_{mm}
\end{pmatrix}
=
\begin{pmatrix}
\lambda_1 a_{11} & \lambda_1 a_{12} & \cdots & \lambda_1 a_{1m} \\
\lambda_2 a_{21} & \lambda_2 a_{22} & \cdots & \lambda_2 a_{2m} \\
\vdots & \vdots & \ddots & \vdots \\
\lambda_m a_{m1} & \lambda_m a_{m2} & \cdots & \lambda_m a_{mm}
\end{pmatrix}
\end{align*}
The rows of $A$ are multiplied, respectively, by $\lambda_1, \lambda_2, \dots, \lambda_m$.
\begin{align*}
A \Lambda = 
\begin{pmatrix}
a_{11} & a_{12} & \cdots & a_{1m} \\
a_{21} & a_{22} & \cdots & a_{2m} \\
\vdots & \vdots & \ddots & \vdots \\
a_{m1} & a_{m2} & \cdots & a_{mm}
\end{pmatrix}
\begin{pmatrix}
\lambda_1 &    0      & \cdots & 0 \\
0         & \lambda_2 & \cdots & 0 \\
\vdots    & \vdots    & \ddots & \vdots \\
0         &    0      & \cdots & \lambda_m
\end{pmatrix}
=
\begin{pmatrix}
\lambda_1 a_{11} & \lambda_2 a_{12} & \cdots & \lambda_m a_{1m} \\
\lambda_1 a_{21} & \lambda_2 a_{22} & \cdots & \lambda_m a_{2m} \\
\vdots & \vdots & \ddots & \vdots \\
\lambda_1 a_{m1} & \lambda_2 a_{m2} & \cdots & \lambda_m a_{mm}
\end{pmatrix}
\end{align*}
The columns of $A$ are multiplied, respectively, by $\lambda_1, \lambda_2, \dots, \lambda_m$.
\end{frame}
%%%%%%%%%%%%%%%%%%%%%%%%%%%%%%%%%%%%%%%%%%%%%%%%%%%%%%%%%%%%%%%%%%%%%%%%




%%%%%%%%%%%%%%%%%%%%%%%%%%%%%%%%%%%%%%%%%%%%%%%%%%%%%%%%%%%%%%%%%%%%%%%%
\begin{frame}
\frametitle{Definition - Invertible matrix}

Consider a square matrix $A \in\mathcal{M}_{n,n}$. We say that $A$ is \textbf{\textcolor{MidnightBlue}{invertible}} if there exists a matrix, denoted $A^{-1}$, which satisfies the following
\begin{align*}
A^{-1}A = I_n = AA^{-1}
\end{align*}

Note:
\begin{itemize}
\item If one of the equalities can be shown, the other follows automatically.
\item If a matrix is invertible, the inverse is unique.
\end{itemize}

We will develop methods to find inverses of matrices in future lessons.
\end{frame}
%%%%%%%%%%%%%%%%%%%%%%%%%%%%%%%%%%%%%%%%%%%%%%%%%%%%%%%%%%%%%%%%%%%%%%%%
\end{document}

%%%%%%%%%%%%%%%%%%%%%%%%%%%%%%%%%%%%%%%%%%%%%%%%%%%%%%%%%%%%%%%%%%%%%%%%
\begin{frame}
\frametitle{Definition - }
\end{frame}
%%%%%%%%%%%%%%%%%%%%%%%%%%%%%%%%%%%%%%%%%%%%%%%%%%%%%%%%%%%%%%%%%%%%%%%%



%%%%%%%%%%%%%%%%%%%%%%%%%%%%%%%%%%%%%%%%%%%%%%%%%%%%%%%%%%%%%%%%%%%%%%%%
\begin{frame}
\frametitle{Properties - }
\end{frame}
%%%%%%%%%%%%%%%%%%%%%%%%%%%%%%%%%%%%%%%%%%%%%%%%%%%%%%%%%%%%%%%%%%%%%%%%



%%%%%%%%%%%%%%%%%%%%%%%%%%%%%%%%%%%%%%%%%%%%%%%%%%%%%%%%%%%%%%%%%%%%%%%%
\begin{frame}
\frametitle{Example - }
\end{frame}
%%%%%%%%%%%%%%%%%%%%%%%%%%%%%%%%%%%%%%%%%%%%%%%%%%%%%%%%%%%%%%%%%%%%%%%%

%%%%%%%%%%%%%%%%%%%%%%%%%%%%%%%%%%%%%%%%%%%%%%%%%%%%%%%%%%%%%%%%%%%%%%%%
\begin{frame}
\frametitle{title}
\fontsize{9pt}{10pt}\selectfont
\end{frame}
%%%%%%%%%%%%%%%%%%%%%%%%%%%%%%%%%%%%%%%%%%%%%%%%%%%%%%%%%%%%%%%%%%%%%%%%


%%%% COLOUR CHOICES
% \textcolor{MidnightBlue}{}
% \textcolor{Maroon}{}
% \textcolor{Purple}{matrix}
% \textcolor{BurntOrange}{}
% \textcolor{MidnightBlue}{}
% \textcolor{Mahogany}{}
% \textcolor{ForestGreen}{}
