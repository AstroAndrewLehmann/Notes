\documentclass[usenames,dvipsnames,aspectratio=169,10pt]{beamer}
%\usetheme{default}
\usetheme[progressbar=frametitle]{metropolis}

\def \EXAMPLEVERSION {1} % 1 for examples, 2 to hide examples (they are in textbook), 3 to hide examples but leave blank slide

\def \SCHOOLVERSION {1} % 1 for neutral, 2 for ISEP

%\documentclass[12pt]{book}
\usepackage{amsfonts}
\usepackage{amsmath}
\usepackage{amssymb}
\usepackage{graphicx}
\usepackage[authoryear]{natbib}
%\usepackage[margin=2.5cm]{geometry}
%\usepackage{hyperref}
\usepackage[font=footnotesize]{caption}
\usepackage{float}
\usepackage{caption}
\usepackage{subcaption}
\usepackage{setspace}
\usepackage{cleveref}
\usepackage{lscape}
\usepackage{multirow}
\usepackage{nicematrix}

% for tikz
\usepackage{tikz}
\usetikzlibrary{angles, arrows.meta, calc, quotes}
\usetikzlibrary{decorations.pathreplacing,calligraphy}
\usetikzlibrary{patterns}
\usetikzlibrary{bending,matrix,positioning}
\usetikzlibrary{arrows, fit, shapes, backgrounds}

\usepackage{tikz-3dplot}
\usepackage{xcolor}


% for red line canceling diagonally
\usepackage{cancel}
\renewcommand{\CancelColor}{\color{red}}

\captionsetup{font=small,labelfont=bf,singlelinecheck=off,margin=2cm,justification=justified}
\numberwithin{equation}{section}

\newcommand{\defbox}[3]
	{
		\vspace{0.5cm} 
		\noindent \fbox{\begin{minipage}{\linewidth}
		\textbf{{#1}DEFINITION{#2}} #3
		\end{minipage}}
		\vspace{0.5cm}
	}
	
	
\newcommand{\defnobox}[3]
	{
		\vspace{0.5cm} 
		\noindent \begin{minipage}{\linewidth}
		\textbf{{#1}DEFINITION{#2}} #3
		\end{minipage}
		\vspace{0.5cm}
	}
	
%%%% To get a nice colourful box around an equation
\newcommand*{\colourboxed}{}
\def\colourboxed#1#{%
  \colourboxedAux{#1}%
}
\newcommand*{\colourboxedAux}[3]{%
  % #1: optional argument for color model
  % #2: color specification
  % #3: formula
  \begingroup
    \colorlet{cb@saved}{.}%
    \color#1{#2}%
    \boxed{%
      \color{cb@saved}%
      #3%
    }%
  \endgroup
}


%%%%%% COLOURS %%%%%%%%%%
\definecolor{airforceblue}{rgb}{0.36, 0.54, 0.66}
\definecolor{battleshipgrey}{rgb}{0.52, 0.52, 0.51}
\definecolor{brightmaroon}{rgb}{0.76, 0.13, 0.28}
\definecolor{nicegreen}{RGB}{133, 204, 111}

% isep colours
\definecolor{isepblue1}{RGB}{0, 97, 161}      % teinte à 100%
\definecolor{isepblue2}{RGB}{77, 144, 189}    % teinte à 70%
\definecolor{isepblue3}{RGB}{179, 208, 227}   % teinte à 30%
\definecolor{iseporange1}{RGB}{244, 161, 0}   % teinte à 100%
\definecolor{iseporange2}{RGB}{234, 189, 100} % teinte à 70%
\definecolor{iseporange3}{RGB}{252, 227, 179} % teinte à 30%
%%%%%%%%%%%%%%%%

% beamer stuff
\setbeamertemplate{navigation symbols}{}
\setbeamersize{text margin left=1.0cm,text margin right=1.0cm}
\setbeamercolor{background canvas}{bg=white}

\ifnum \SCHOOLVERSION = 2
	%%%% ISEP COLOURS %%%%
	\setbeamercolor{frametitle}{bg=isepblue1, fg=white}
	\setbeamercolor{progress bar}{fg=iseporange1}
	\setbeamercolor{itemize item}{fg=iseporange1,bg=iseporange1}
	%%%%%%%%%%%%%%%%%%%%%%
\fi

\setbeamerfont{frametitle}{family=\fontfamily{qag}\selectfont} % choose font for frame titles
\setbeamerfont{title}{family=\fontfamily{qag}\selectfont} % choose font for title
\setbeamerfont{subtitle}{family=\fontfamily{qag}\selectfont} % choose font for subtitle
\setbeamerfont{section title}{family=\fontfamily{qag}\selectfont} % choose font for titles
%\fontfamily{qag}\selectfont %choose font for main text % put after begin{document}

% to make the progress bar a little thicker
\makeatletter
\setlength{\metropolis@titleseparator@linewidth}{1.5pt}
\setlength{\metropolis@progressonsectionpage@linewidth}{1.5pt}
\setlength{\metropolis@progressinheadfoot@linewidth}{1.5pt}
\makeatother

\begin{document}

\title{Linear Algebra}
\subtitle{Determinants}
\author{Andrew Lehmann}
\ifnum \SCHOOLVERSION = 2
	\institute{\'{E}cole d'ing\'{e}nieurs du num\'{e}rique}
\fi
\date{\textit{Last updated: \today}}

% logo of university
\ifnum \SCHOOLVERSION = 2
	\titlegraphic{\includegraphics[width=3cm]{/home/andrew/Dropbox/ISEP/admin/logo-isep-2023.png} }
\fi


\begin{frame}
\titlepage
\end{frame}




%%%%%%%%%%%%%%%%%%%%%%%%%%%%%%%%%%%%%%%%%%%%%%%%%%%%%%%%%%%%%%%%%%%%%%%%
\begin{frame}

Suppose we have some square matrix $A\in \mathcal{M}_{n,n}$
\begin{align*}
A =
\begin{pmatrix}
a_{11} & a_{12} & \cdots & a_{1n} \\
a_{21} & a_{22} & \cdots & a_{2n} \\
\vdots & \vdots & \ddots & \vdots \\
a_{n1} & a_{n2} & \cdots & a_{nn}
\end{pmatrix}
\end{align*}
Then the \textit{determinant of $A$}, denoted $\det(A)$ or $|A|$, is some number that, among other uses, tells us if $A$ is invertible. If the determinant of a matrix is non-zero, then it is invertible. Recall that only square matrices are invertible.

\end{frame}
%%%%%%%%%%%%%%%%%%%%%%%%%%%%%%%%%%%%%%%%%%%%%%%%%%%%%%%%%%%%%%%%%%%%%%%%



%%%%%%%%%%%%%%%%%%%%%%%%%%%%%%%%%%%%%%%%%%%%%%%%%%%%%%%%%%%%%%%%%%%%%%%%
\begin{frame}
\frametitle{Definition - Determinant of a 1 by 1 matrix}

Suppose we have a general 1 by 1 matrix $A = \begin{pmatrix} a \end{pmatrix}$. It shouldn't be too difficult to see that the inverse matrix must be $A^{-1} = \begin{pmatrix} \frac{1}{a} \end{pmatrix}$ because the multiplication $\begin{pmatrix} a \end{pmatrix}\begin{pmatrix} \frac{1}{a} \end{pmatrix} = \begin{pmatrix} 1 \end{pmatrix}$, the identity matrix. Since $1/a$ is well defined only if $a\neq 0$, the value of $a$ itself satisfies this idea ``if the determinant is non-zero, the matrix is invertible''. So we can say, the determinant of any 1 by 1 matrix is given by its only value:

\begin{align*}
\det \begin{pmatrix} a \end{pmatrix} = a
\end{align*}

Before we define how to calculate determinants of larger square matrices, we must first introduce the concept of a submatrix.
\end{frame}
%%%%%%%%%%%%%%%%%%%%%%%%%%%%%%%%%%%%%%%%%%%%%%%%%%%%%%%%%%%%%%%%%%%%%%%%

%%%%%%%%%%%%%%%%%%%%%%%%%%%%%%%%%%%%%%%%%%%%%%%%%%%%%%%%%%%%%%%%%%%%%%%%
\begin{frame}
\frametitle{Definition - Submatrix}
From a matrix $A$ we generate the \textit{submatrix} $A_{ij}$ by deleting the $ith$ row and $jth$ column:

\begin{columns}\column{\dimexpr\paperwidth-1pt}
\fontsize{7.8pt}{12pt}\selectfont
\begin{align*}
A =
\begin{pmatrix}
a_{1,1}   & \cdots & a_{1,j-1}   & a_{1,j}   & a_{1,j+1}   & \cdots & a_{1,n}   \\
\vdots    & \cdots & \vdots      & \vdots    & \vdots      & \cdots & \vdots    \\
a_{i-1,1} & \cdots & a_{i-1,j-1} & a_{i-1,j} & a_{i-1,j+1} & \cdots & a_{i-1,n} \\
a_{i,1}   & \cdots & a_{i,j-1}   & a_{i,j}   & a_{i,j+1}   & \cdots & a_{i,n}   \\
a_{i+1,1} & \cdots & a_{i+1,j-1} & a_{i+1,j} & a_{i+1,j+1} & \cdots & a_{i+1,n} \\
\vdots    & \cdots & \vdots      & \vdots    & \vdots      & \cdots & \vdots    \\
a_{m,1}   & \cdots & a_{m,j-1}   & a_{m,j}   & a_{m,j+1}   & \cdots & a_{m,n} 
\end{pmatrix} 
\to \, A_{ij} =
\begin{pmatrix}
a_{1,1}   & \cdots & a_{1,j-1}   & a_{1,j+1}   & \cdots & a_{1,n}   \\
\vdots    & \cdots & \vdots      & \vdots      & \cdots & \vdots    \\
a_{i-1,1} & \cdots & a_{i-1,j-1} & a_{i-1,j+1} & \cdots & a_{i-1,n} \\
a_{i+1,1} & \cdots & a_{i+1,j-1} & a_{i+1,j+1} & \cdots & a_{i+1,n} \\
\vdots    & \cdots & \vdots      & \vdots      & \cdots & \vdots    \\
a_{m,1}   & \cdots & a_{m,j-1}   & a_{m,j+1}   & \cdots & a_{m,n} 
\end{pmatrix}
\end{align*}
\fontsize{10pt}{10pt}\selectfont
\end{columns}

\textit{Note}: we generally have to specify in words that we are creating a submatrix. The notation $A_{ij}$ is a little ambiguous without being explicit. 
\end{frame}
%%%%%%%%%%%%%%%%%%%%%%%%%%%%%%%%%%%%%%%%%%%%%%%%%%%%%%%%%%%%%%%%%%%%%%%%


\ifnum \EXAMPLEVERSION = 1
%%%%%%%%%%%%%%%%%%%%%%%%%%%%%%%%%%%%%%%%%%%%%%%%%%%%%%%%%%%%%%%%%%%%%%%%
\begin{frame}
\frametitle{Examples - Submatrices}

If we have the matrix
\begin{align*}
C=
\begin{pmatrix}
 2 &  3 & 1 \\
-1 &  1 & 1 \\
 0 & -2 & 0 
\end{pmatrix}
\end{align*}
then we have submatrices
\begin{align*}
C_{12}=
\begin{pmatrix}
-1 &  1 \\
 0 &  0 
\end{pmatrix}
, \quad
C_{22}=
\begin{pmatrix}
 2 & 1 \\
 0 & 0 
\end{pmatrix}
, \quad
C_{31}=
\begin{pmatrix}
 3 & 1 \\
 1 & 1
\end{pmatrix}
\end{align*}

\end{frame}
%%%%%%%%%%%%%%%%%%%%%%%%%%%%%%%%%%%%%%%%%%%%%%%%%%%%%%%%%%%%%%%%%%%%%%%%
\fi 


\ifnum \EXAMPLEVERSION = 3
%%%%%%%%%%%%%%%%%%%%%%%%%%%%%%%%%%%%%%%%%%%%%%%%%%%%%%%%%%%%%%%%%%%%%%%%
\begin{frame}
\frametitle{Examples - Submatrices}
\end{frame}
%%%%%%%%%%%%%%%%%%%%%%%%%%%%%%%%%%%%%%%%%%%%%%%%%%%%%%%%%%%%%%%%%%%%%%%%
\fi 



%%%%%%%%%%%%%%%%%%%%%%%%%%%%%%%%%%%%%%%%%%%%%%%%%%%%%%%%%%%%%%%%%%%%%%%%
\begin{frame}
\frametitle{Definition - Determinant}
The determinant of a $n$ by $n$ square matrix $A$ is given by either of the following equivalent summations
\begin{align*}
\det(A) = \underbrace{\sum_{i=1}^n (-1)^{i+j} a_{ij} \det(A_{ij})}_{\text{for any choice of }j} = \underbrace{\sum_{j=1}^n (-1)^{i+j} a_{ij} \det(A_{ij})}_{\text{for any choice of }i}
\end{align*}

We will soon learn why the left summation will be ``along a column'' and the right summation will be ``along a row''.

This definition should strike you as odd. It defines the determinant in terms of the addition of other determinants! Let's try to understand this with an example.
\end{frame}
%%%%%%%%%%%%%%%%%%%%%%%%%%%%%%%%%%%%%%%%%%%%%%%%%%%%%%%%%%%%%%%%%%%%%%%%



%%%%%%%%%%%%%%%%%%%%%%%%%%%%%%%%%%%%%%%%%%%%%%%%%%%%%%%%%%%%%%%%%%%%%%%%
\begin{frame}
\frametitle{Definition - Determinant of a 2 by 2 matrix}

Suppose we have a general 2 by 2 matrix $A = \begin{pmatrix} a & b \\ c & d\end{pmatrix}$.

The determinant is therefore (choosing $j=1$ for the first summation)
\begin{align*}
\det(A) = \sum_{i=1}^n (-1)^{i+1} a_{i1} \det(A_{i1}) &= (-1)^{2} a_{11} \det(A_{11}) + (-1)^{3} a_{21} \det(A_{21})\\
 &= a \det\begin{pmatrix} d\end{pmatrix} - c \det\begin{pmatrix} b \end{pmatrix}.
\end{align*}
We defined earlier the determinants of 1 by 1 matrices, so we have a formula worth remembering
\begin{align*}
\colourboxed{nicegreen}{\det \begin{pmatrix} a & b \\ c & d\end{pmatrix} = ad - bc}
\end{align*}
I think of this formula as ``Multiplication of on-diagonal minus multiplication of off-diagonal''.

\end{frame}
%%%%%%%%%%%%%%%%%%%%%%%%%%%%%%%%%%%%%%%%%%%%%%%%%%%%%%%%%%%%%%%%%%%%%%%%


\ifnum \EXAMPLEVERSION = 1
%%%%%%%%%%%%%%%%%%%%%%%%%%%%%%%%%%%%%%%%%%%%%%%%%%%%%%%%%%%%%%%%%%%%%%%%
\begin{frame}
\frametitle{Example - Determinant of a 3 by 3 matrix}
Consider the matrix 
$A=
\begin{pmatrix}
 2 &  3 & 1 \\
-1 &  1 & 1 \\
 0 & -2 & 0 
\end{pmatrix}
$.
Let's take the first summation and choose $j=1$. Then
\begin{align*}
\det(A) = \sum_{i=1}^n (-1)^{i+1} a_{i1} \det(A_{i1}) &= (-1)^{2} a_{11} \det(A_{11}) + (-1)^{3} a_{21} \det(A_{21}) + (-1)^{4} a_{31} \det(A_{31}) \\
 &= 2 \det(A_{11}) - -1 \det(A_{21}) + 0 \det(A_{31}) \\
 &= 2 \det\begin{pmatrix}
  1 & 1 \\
 -2 & 0 
\end{pmatrix} 
+ \det
\begin{pmatrix}
  3 & 1 \\
 -2 & 0 
\end{pmatrix} 
+ 0\det
\begin{pmatrix}
  3 & 1 \\
  1 & 1 \\
\end{pmatrix} \\
 &= 2 \left(1\times 0 - -2\times 1 \right) +  \left(3\times 0 --2\times 1 \right) \\
 &= 6
\end{align*}
We see that the determinant of this matrix was ``reduced'' to determinants of smaller matrices.
\end{frame}
%%%%%%%%%%%%%%%%%%%%%%%%%%%%%%%%%%%%%%%%%%%%%%%%%%%%%%%%%%%%%%%%%%%%%%%%
\fi 

\ifnum \EXAMPLEVERSION = 3
%%%%%%%%%%%%%%%%%%%%%%%%%%%%%%%%%%%%%%%%%%%%%%%%%%%%%%%%%%%%%%%%%%%%%%%%
\begin{frame}
\frametitle{Example - Determinant of a 3 by 3 matrix}
\end{frame}
%%%%%%%%%%%%%%%%%%%%%%%%%%%%%%%%%%%%%%%%%%%%%%%%%%%%%%%%%%%%%%%%%%%%%%%%
\fi 


\ifnum \EXAMPLEVERSION = 1
%%%%%%%%%%%%%%%%%%%%%%%%%%%%%%%%%%%%%%%%%%%%%%%%%%%%%%%%%%%%%%%%%%%%%%%%
\begin{frame}
\frametitle{Example - Determinant of a 3 by 3 matrix along a row}
Let's take the determinant of the same matrix, but instead use the row summation by choose $i=2$. Then
\begin{align*}
\det(A) &= \sum_{j=1}^n (-1)^{2+j} a_{2j} \det(A_{2j}) \\
 &= (-1)^{3} a_{21} \det(A_{21}) + (-1)^{4} a_{22} \det(A_{22}) + (-1)^{5} a_{23} \det(A_{23}) \\
 &= - -1 
 \det\begin{pmatrix}
  3 & 1 \\
 -2 & 0 
\end{pmatrix}
 + 1 
 \det\begin{pmatrix}
 2 &  1 \\
 0 &  0 
\end{pmatrix} 
- 1 
\det\begin{pmatrix}
 2 &  3 \\
 0 & -2 
\end{pmatrix} \\
 &= (3\times 0 - -2\times 1) + (0 - 0)- (2\times -2 - 3\times 0 ) \\
 &= 6
\end{align*}
We get the same answer!
\end{frame}
%%%%%%%%%%%%%%%%%%%%%%%%%%%%%%%%%%%%%%%%%%%%%%%%%%%%%%%%%%%%%%%%%%%%%%%%
\fi 


\ifnum \EXAMPLEVERSION = 3
%%%%%%%%%%%%%%%%%%%%%%%%%%%%%%%%%%%%%%%%%%%%%%%%%%%%%%%%%%%%%%%%%%%%%%%%
\begin{frame}
\frametitle{Example - Determinant of a 3 by 3 matrix along a row}
\end{frame}
%%%%%%%%%%%%%%%%%%%%%%%%%%%%%%%%%%%%%%%%%%%%%%%%%%%%%%%%%%%%%%%%%%%%%%%%
\fi 



%%%%%%%%%%%%%%%%%%%%%%%%%%%%%%%%%%%%%%%%%%%%%%%%%%%%%%%%%%%%%%%%%%%%%%%%
\begin{frame}
\frametitle{Properties}
In the determinant expression there is this term $(-1)^{i+j}$ that appears in both summations:
\begin{align*}
(-1)^{i+j} = 
\begin{cases}
+1 & \text{if $i+j$ is even} \\
-1 & \text{if $i+j$ is odd}
\end{cases}
\end{align*}
This determines the following $\pm$ pattern to the matrix\begin{align*}
\begin{pmatrix}
+ & - & + & - & \cdots \\
- & + & - & + & \cdots \\
+ & - & + & - & \cdots \\
\vdots & \vdots & \vdots & \vdots & \ddots
\end{pmatrix}
\end{align*}
Noticing this pattern lets you avoid having to explicitly write $(-1)^{i+j}$ during the calculations.
\end{frame}
%%%%%%%%%%%%%%%%%%%%%%%%%%%%%%%%%%%%%%%%%%%%%%%%%%%%%%%%%%%%%%%%%%%%%%%%



\ifnum \EXAMPLEVERSION = 1
%%%%%%%%%%%%%%%%%%%%%%%%%%%%%%%%%%%%%%%%%%%%%%%%%%%%%%%%%%%%%%%%%%%%%%%%
\begin{frame}
\frametitle{Examples}
\begin{align*}
\left|\begin{matrix}
2 & 5 \\ 3 & 1
\end{matrix}\right|=-13
,\qquad
\left|\begin{matrix}
5 & 10 \\ 1 & 2
\end{matrix}\right|=0
,\qquad
\left|\begin{matrix}
2 & 4 & 0 \\ 1 & 3 & 0 \\ 0 & 0 & 1
\end{matrix}\right|=2
,\qquad
\left|\begin{matrix}
2 & 4 & 0 \\ 1 & 2 & -3 \\ 0 & 0 & 13
\end{matrix}\right|=0
\end{align*}
\end{frame}
%%%%%%%%%%%%%%%%%%%%%%%%%%%%%%%%%%%%%%%%%%%%%%%%%%%%%%%%%%%%%%%%%%%%%%%%
\fi 


\ifnum \EXAMPLEVERSION = 3
%%%%%%%%%%%%%%%%%%%%%%%%%%%%%%%%%%%%%%%%%%%%%%%%%%%%%%%%%%%%%%%%%%%%%%%%
\begin{frame}
\frametitle{Exercices}
\begin{minipage}{0.4\textwidth}
Calculate the following determinants
\begin{align*}
& \left|\begin{matrix}
2 & 5 \\ 3 & 1
\end{matrix}\right| = 
\\
& \left|\begin{matrix}
5 & 10 \\ 1 & 2
\end{matrix}\right| = 
\\
& \left|\begin{matrix}
2 & 4 & 0 \\ 1 & 3 & 0 \\ 0 & 0 & 1
\end{matrix}\right| = 
\\
& \left|\begin{matrix}
2 & 4 & 0 \\ 1 & 2 & -3 \\ 0 & 0 & 13
\end{matrix}\right| =
\end{align*}
\end{minipage}
\end{frame}
%%%%%%%%%%%%%%%%%%%%%%%%%%%%%%%%%%%%%%%%%%%%%%%%%%%%%%%%%%%%%%%%%%%%%%%%
\fi 



%%%%%%%%%%%%%%%%%%%%%%%%%%%%%%%%%%%%%%%%%%%%%%%%%%%%%%%%%%%%%%%%%%%%%%%%
\begin{frame}
\frametitle{Theorem - Determinant when column is multiplied by a constant}

If we multiply a column by a constant, $k$, then the determinant is multiplied by that constant. 
\vspace{-0.1cm}\begin{center} \textcolor{airforceblue}{\rule{0.7\textwidth}{0.3mm}} \end{center}

\textbf{Proof}: Let's multiply the j$^{th}$ column of a matrix $A$ by $k$ to get the new matrix
\begin{align*}
A' =
\begin{pmatrix}
a_{11} & \cdots & ka_{1j} & \cdots & a_{1n} \\
a_{21} & \cdots & ka_{2j} & \cdots & a_{2n} \\
\vdots & \ddots & \vdots & \cdots & \vdots \\
a_{n1} & \cdots & ka_{nj} & \cdots & a_{nn}
\end{pmatrix}
\end{align*}
Then, calculate the determinant of $A'$, choosing to compute it along the j$^{th}$ column
\begin{align*}
\det(A') = \sum_{i=1}^n (-1)^{i+j} (ka_{ij}) \det(A'_{ij}) = k\sum_{i=1}^n (-1)^{i+j} a_{ij} \det(A'_{ij}) = k \det (A)
\end{align*}
\end{frame}
%%%%%%%%%%%%%%%%%%%%%%%%%%%%%%%%%%%%%%%%%%%%%%%%%%%%%%%%%%%%%%%%%%%%%%%%





%%%%%%%%%%%%%%%%%%%%%%%%%%%%%%%%%%%%%%%%%%%%%%%%%%%%%%%%%%%%%%%%%%%%%%%%
\begin{frame}
\frametitle{Theorems}
\centering
Given two matrices $A,B\in \mathcal{M}_{n,n}$, we have
\begin{align*}
\det(AB)=\det(A)\det(B)
\end{align*}
\vspace{-1.3cm}\begin{center} \textcolor{airforceblue}{\rule{0.7\textwidth}{0.3mm}} \end{center}\vspace{-0.2cm}

Let $A\in\mathcal{M}_{n,n}$ be a diagonal matrix. Then
\begin{align*}
\det(A)=a_{11}\times a_{22} \times \cdots \times a_{nn}
\end{align*}
\vspace{-1.3cm}\begin{center} \textcolor{airforceblue}{\rule{0.7\textwidth}{0.3mm}} \end{center}\vspace{-0.2cm}

Let $A\in\mathcal{M}_{n,n}$ be a triangular matrix. Then
\begin{align*}
\det(A)=a_{11}\times a_{22} \times \cdots \times a_{nn}
\end{align*}
\vspace{-1.3cm}\begin{center} \textcolor{airforceblue}{\rule{0.7\textwidth}{0.3mm}} \end{center}\vspace{-0.2cm}

Let $\Lambda, A\in\mathcal{M}_{n,n}$ with $\Lambda=\lambda I$ a diagonal matrix. Then
\begin{align*}
\det(\Lambda A)=\lambda^n \det(A)
\end{align*}

\end{frame}
%%%%%%%%%%%%%%%%%%%%%%%%%%%%%%%%%%%%%%%%%%%%%%%%%%%%%%%%%%%%%%%%%%%%%%%%



%%%%%%%%%%%%%%%%%%%%%%%%%%%%%%%%%%%%%%%%%%%%%%%%%%%%%%%%%%%%%%%%%%%%%%%%
\begin{frame}
\frametitle{Theorem Proof}

\begin{center} If $A$ is triangular then $\det(A)=a_{11}\times a_{22} \times \cdots \times a_{nn}$ \end{center}
\vspace{-0.5cm}\begin{center} \textcolor{airforceblue}{\rule{0.7\textwidth}{0.3mm}} \end{center}\vspace{-0.2cm}

\textbf{Partial Proof}: Consider an upper triangular matrix
\begin{align*}
A =
\begin{pmatrix}
a_{11} & a_{12} & \cdots & a_{1n} \\
0 & a_{22} & \cdots & a_{2n} \\
\vdots & \vdots & \ddots & \vdots \\
0 & 0 & \cdots & a_{nn}
\end{pmatrix}
\end{align*}
If we take the determinant along the first column at each step we get
\begin{align*}
\det(A) 
= a_{11} 
\det\begin{pmatrix}
a_{22} & a_{23} & \cdots & a_{2n} \\
0 & a_{33} & \cdots & a_{3n} \\
\vdots & \vdots & \ddots & \vdots \\
0 & 0 & \cdots & a_{nn}
\end{pmatrix} 
= a_{11}a_{22} 
\det\begin{pmatrix}
a_{33} & a_{34} & \cdots & a_{3n} \\
0 & a_{44} & \cdots & a_{4n} \\
\vdots & \vdots & \ddots & \vdots \\
0 & 0 & \cdots & a_{nn}
\end{pmatrix}
= \cdots = a_{11}a_{22} \cdots  a_{nn}
\end{align*}
\end{frame}
%%%%%%%%%%%%%%%%%%%%%%%%%%%%%%%%%%%%%%%%%%%%%%%%%%%%%%%%%%%%%%%%%%%%%%%%





%%%%%%%%%%%%%%%%%%%%%%%%%%%%%%%%%%%%%%%%%%%%%%%%%%%%%%%%%%%%%%%%%%%%%%%%
\begin{frame}
\frametitle{Theorem - Determinant of Inverse}
Let $A\in\mathcal{M}_{n,n}$ be an invertible matrix. Then
\begin{align*}
\det(A) \neq 0 
\quad \text{and} \quad 
\det(A^{-1})=\frac{1}{\det(A)}
\end{align*}
\vspace{-0.5cm}\begin{center} \textcolor{airforceblue}{\rule{0.7\textwidth}{0.3mm}} \end{center}\vspace{-0.2cm}

\textbf{Proof of the second}: Since $A$ is invertible, its inverse exists and is defined by $AA^{-1}=I$. The determinant of this relation is
\begin{align*}
\det(AA^{-1}) &= \det(I) \\
\det(A)\det(A^{-1}) &= 1 \\
\det(A^{-1}) &=\frac{1}{\det(A)}
\end{align*}
\end{frame}
%%%%%%%%%%%%%%%%%%%%%%%%%%%%%%%%%%%%%%%%%%%%%%%%%%%%%%%%%%%%%%%%%%%%%%%%





%%%%%%%%%%%%%%%%%%%%%%%%%%%%%%%%%%%%%%%%%%%%%%%%%%%%%%%%%%%%%%%%%%%%%%%%
\begin{frame}
\frametitle{Theorem}
Let $P\in\mathcal{M}_{n,n}$ be an invertible matrix. Then for any matrix $A\in\mathcal{M}_{n,n}$ we have
\begin{align*}
\det(PAP^{-1}) = \det(A).
\end{align*}

Can you prove it?
\end{frame}
%%%%%%%%%%%%%%%%%%%%%%%%%%%%%%%%%%%%%%%%%%%%%%%%%%%%%%%%%%%%%%%%%%%%%%%%





%%%%%%%%%%%%%%%%%%%%%%%%%%%%%%%%%%%%%%%%%%%%%%%%%%%%%%%%%%%%%%%%%%%%%%%%
\begin{frame}
\frametitle{Theorems - determinants and column properties}
Let $A\in\mathcal{M}_{n,n}$ be a square matrix. If any two columns of $A$ are equal to each other then $\det(A)=0$. For example
\begin{align*}
\det\begin{pmatrix}
 2 &  1 & 2 \\
 4 &  2 & 4 \\
 6 &  0 & 6 
\end{pmatrix}
=0.
\end{align*}
If any column is a multiple of another, then $\det(A)=0$. For example
\begin{align*}
\det\begin{pmatrix}
 1 &  3 & 0 \\
 2 &  6 & 3 \\
 2 &  6 & 3 
\end{pmatrix}
=0.
\end{align*}
\end{frame}
%%%%%%%%%%%%%%%%%%%%%%%%%%%%%%%%%%%%%%%%%%%%%%%%%%%%%%%%%%%%%%%%%%%%%%%%





%%%%%%%%%%%%%%%%%%%%%%%%%%%%%%%%%%%%%%%%%%%%%%%%%%%%%%%%%%%%%%%%%%%%%%%%
\begin{frame}
\frametitle{Theorems - determinants and column operations}
Let $A\in\mathcal{M}_{n,n}$ be a square matrix. If we exchange any two columns, then the determinant of the new matrix is $-1$ times the old. For example
\begin{align*}
\det\begin{pmatrix}
 2 &  1 & 7 \\
 4 &  2 & 4 \\
 6 &  0 & 3 
\end{pmatrix}
=-1
\det\begin{pmatrix}
 2 &  7 & 1 \\
 4 &  4 & 2 \\
 6 &  3 & 0 
\end{pmatrix}
\end{align*}
The determinant is unchanged if we add to a column a linear combination of other columns. For example
\begin{align*}
\det\begin{pmatrix}
 2 &  1 & 7 \\
 4 &  2 & 4 \\
 6 &  0 & 3 
\end{pmatrix}
=
\det\begin{pmatrix}
 2 &  1 & 9 \\
 4 &  2 & 8 \\
 6 &  0 & 3 
\end{pmatrix}
\end{align*}
(the new third column is the old third column + 2 times the second column)
\end{frame}
%%%%%%%%%%%%%%%%%%%%%%%%%%%%%%%%%%%%%%%%%%%%%%%%%%%%%%%%%%%%%%%%%%%%%%%%



%%%%%%%%%%%%%%%%%%%%%%%%%%%%%%%%%%%%%%%%%%%%%%%%%%%%%%%%%%%%%%%%%%%%%%%%
\begin{frame}
\frametitle{Theorems - determinants and row properties and operations}
For every square matrix $A$, the previous results also hold for rows! For example:
\begin{itemize}
\item If two rows are equal to each other, then $\det(A)=0$.
\item If any row is a multiple of another, then $\det(A)=0$.
\item If we exchange any two rows, the new determinant is $-1$ times the old.
\item The determinant is unchanged if we add to a row a linear combination of other rows.
\end{itemize}
\end{frame}
%%%%%%%%%%%%%%%%%%%%%%%%%%%%%%%%%%%%%%%%%%%%%%%%%%%%%%%%%%%%%%%%%%%%%%%%






%%%%%%%%%%%%%%%%%%%%%%%%%%%%%%%%%%%%%%%%%%%%%%%%%%%%%%%%%%%%%%%%%%%%%%%%
\begin{frame}
\frametitle{Theorem - Existence of a triangular with the same determinant}
For every square matrix, $A$, there exists a triangular matrix, $T$, such that $\det(A)=\det(T)$.

\vspace{-0.5cm}\begin{center} \textcolor{airforceblue}{\rule{0.7\textwidth}{0.3mm}} \end{center}\vspace{-0.2cm}

\textbf{Proof}:

\noindent \begin{minipage}{0.3\textwidth}
\begin{align*}
A =
\begin{pmatrix}
a_{11} & a_{12} & \cdots & a_{1n} \\
a_{21} & a_{22} & \cdots & a_{2n} \\
\vdots & \vdots & \ddots & \vdots \\
a_{n1} & a_{n2} & \cdots & a_{nn}
\end{pmatrix}
\end{align*}
\end{minipage}\hfill
\begin{minipage}{0.68\textwidth}
Consider the first column. There are 2 possibilities:
\begin{enumerate}
\item All the $a_{i1}=0$. Then $\det(A)=0$ and the zero matrix is a triangular matrix with the same determinant.
\item There is a non-zero element in the first column, say $a_{k1}\neq 0$. We can add row $k$ to the first row, giving a new matrix $A'$ without changing the determinant. In this way we can always generate a matrix with a non-zero upper left element, $a'_{11}$, and the same determinant.
\end{enumerate}
\end{minipage}
\end{frame}
%%%%%%%%%%%%%%%%%%%%%%%%%%%%%%%%%%%%%%%%%%%%%%%%%%%%%%%%%%%%%%%%%%%%%%%%



%%%%%%%%%%%%%%%%%%%%%%%%%%%%%%%%%%%%%%%%%%%%%%%%%%%%%%%%%%%%%%%%%%%%%%%%
\begin{frame}
\frametitle{Proof continued}

Now we can use this $a'_{11}$ to guarantee it is the \textit{only} non-zero element in the first column. To do this we subtract from every row other than the first this particular multiple of the first row:
\begin{align*}
R_i \to R_i - \frac{a_{i1}}{a'_{11}}R_1
\end{align*}
This operation does not change the determinant, so we have:
\begin{align*}
\det(A)=\det\begin{pmatrix}
a'_{11} & a'_{12} & \cdots & a'_{1n} \\
a_{21} & a_{22} & \cdots & a_{2n} \\
\vdots & \vdots & \ddots & \vdots \\
a_{n1} & a_{n2} & \cdots & a_{nn}
\end{pmatrix}
=
\begin{pmatrix}
a'_{11} & a'_{12} & \cdots & a'_{1n} \\
0 & a^{(1)}_{22} & \cdots & a^{(1)}_{2n} \\
\vdots & \vdots & \ddots & \vdots \\
0 & a^{(1)}_{n2} & \cdots & a^{(1)}_{nn}
\end{pmatrix}
\end{align*}
\end{frame}
%%%%%%%%%%%%%%%%%%%%%%%%%%%%%%%%%%%%%%%%%%%%%%%%%%%%%%%%%%%%%%%%%%%%%%%%



%%%%%%%%%%%%%%%%%%%%%%%%%%%%%%%%%%%%%%%%%%%%%%%%%%%%%%%%%%%%%%%%%%%%%%%%
\begin{frame}
\frametitle{Proof continued}

We can repeat this procedure for the second column, ignoring the first row, to successively generate an upper triangular matrix
\begin{align*}
\det(A)=
\det\begin{pmatrix}
a'_{11} & a'_{12} & a'_{13} & \cdots & a'_{1n} \\
0 & a^{(1)}_{22} & a^{(1)}_{23} & \cdots & a^{(1)}_{2n} \\
0 & 0 & a^{(2)}_{33} & \cdots & a^{(2)}_{2n} \\
\vdots & \vdots & \vdots & \ddots & \vdots \\
0 & 0 & a^{(2)}_{3n} & \cdots & a^{(2)}_{nn}
\end{pmatrix}
=
\cdots
=
\det\begin{pmatrix}
a'_{11} & a'_{12} & a'_{13} & \cdots & a'_{1n} \\
0 & a^{(1)}_{22}  & a^{(1)}_{23} & \cdots & a^{(1)}_{2n} \\
0 & 0 & a^{(2)}_{33} & \cdots & a^{(2)}_{2n} \\
\vdots & \vdots & \vdots & \ddots & \vdots \\
0 & 0 & 0 & \cdots & a^{(n-1)}_{nn}
\end{pmatrix}
\end{align*}
Now it's much easier to calculate the determinant of this triangular matrix (multiply the diagonal). In this way, at the end of the Gaussian reduction we can know immediately if the matrix is invertible or not (whether there is a zero on the diagonal).
\end{frame}
%%%%%%%%%%%%%%%%%%%%%%%%%%%%%%%%%%%%%%%%%%%%%%%%%%%%%%%%%%%%%%%%%%%%%%%%

\end{document}



%%%%%%%%%%%%%%%%%%%%%%%%%%%%%%%%%%%%%%%%%%%%%%%%%%%%%%%%%%%%%%%%%%%%%%%%
\begin{frame}
\frametitle{Definition - }
\end{frame}
%%%%%%%%%%%%%%%%%%%%%%%%%%%%%%%%%%%%%%%%%%%%%%%%%%%%%%%%%%%%%%%%%%%%%%%%



%%%%%%%%%%%%%%%%%%%%%%%%%%%%%%%%%%%%%%%%%%%%%%%%%%%%%%%%%%%%%%%%%%%%%%%%
\begin{frame}
\frametitle{Properties - }
\end{frame}
%%%%%%%%%%%%%%%%%%%%%%%%%%%%%%%%%%%%%%%%%%%%%%%%%%%%%%%%%%%%%%%%%%%%%%%%



%%%%%%%%%%%%%%%%%%%%%%%%%%%%%%%%%%%%%%%%%%%%%%%%%%%%%%%%%%%%%%%%%%%%%%%%
\begin{frame}
\frametitle{Theorem - }
\end{frame}
%%%%%%%%%%%%%%%%%%%%%%%%%%%%%%%%%%%%%%%%%%%%%%%%%%%%%%%%%%%%%%%%%%%%%%%%



%%%%%%%%%%%%%%%%%%%%%%%%%%%%%%%%%%%%%%%%%%%%%%%%%%%%%%%%%%%%%%%%%%%%%%%%
\begin{frame}
\frametitle{Example - }
\end{frame}
%%%%%%%%%%%%%%%%%%%%%%%%%%%%%%%%%%%%%%%%%%%%%%%%%%%%%%%%%%%%%%%%%%%%%%%%

%%%%%%%%%%%%%%%%%%%%%%%%%%%%%%%%%%%%%%%%%%%%%%%%%%%%%%%%%%%%%%%%%%%%%%%%
\begin{frame}
\frametitle{title}
\fontsize{9pt}{10pt}\selectfont
\end{frame}
%%%%%%%%%%%%%%%%%%%%%%%%%%%%%%%%%%%%%%%%%%%%%%%%%%%%%%%%%%%%%%%%%%%%%%%%


%%%% COLOUR CHOICES
% \textcolor{MidnightBlue}{}
% \textcolor{Maroon}{}
% \textcolor{Purple}{matrix}
% \textcolor{BurntOrange}{}
% \textcolor{MidnightBlue}{}
% \textcolor{Mahogany}{}
% \textcolor{ForestGreen}{}
