\documentclass[usenames,dvipsnames,aspectratio=169,10pt]{beamer}
%\usetheme{default}
\usetheme[progressbar=frametitle]{metropolis}

\def \EXAMPLEVERSION {3} % 1 for examples, 2 to hide examples (they are in textbook), 3 to hide examples but leave blank slide

\def \SCHOOLVERSION {2} % 1 for neutral, 2 for ISEP

%\documentclass[12pt]{book}
\usepackage{amsfonts}
\usepackage{amsmath}
\usepackage{amssymb}
\usepackage{graphicx}
\usepackage[authoryear]{natbib}
%\usepackage[margin=2.5cm]{geometry}
%\usepackage{hyperref}
\usepackage[font=footnotesize]{caption}
\usepackage{float}
\usepackage{caption}
\usepackage{subcaption}
\usepackage{setspace}
\usepackage{cleveref}
\usepackage{lscape}
\usepackage{multirow}
\usepackage{nicematrix}

% for tikz
\usepackage{tikz}
\usetikzlibrary{angles, arrows.meta, calc, quotes}
\usetikzlibrary{decorations.pathreplacing,calligraphy}
\usetikzlibrary{patterns}
\usetikzlibrary{bending,matrix,positioning}
\usetikzlibrary{arrows, fit, shapes, backgrounds}

\usepackage{tikz-3dplot}
\usepackage{xcolor,colortbl}


% for red line canceling diagonally
\usepackage{cancel}
\renewcommand{\CancelColor}{\color{red}}

\captionsetup{font=small,labelfont=bf,singlelinecheck=off,margin=2cm,justification=justified}
\numberwithin{equation}{section}

\newcommand{\defbox}[3]
	{
		\vspace{0.5cm} 
		\noindent \fbox{\begin{minipage}{\linewidth}
		\textbf{{#1}DEFINITION{#2}} #3
		\end{minipage}}
		\vspace{0.5cm}
	}
	
	
\newcommand{\defnobox}[3]
	{
		\vspace{0.5cm} 
		\noindent \begin{minipage}{\linewidth}
		\textbf{{#1}DEFINITION{#2}} #3
		\end{minipage}
		\vspace{0.5cm}
	}
	
%%%% To get a nice colourful box around an equation
\newcommand*{\colourboxed}{}
\def\colourboxed#1#{%
  \colourboxedAux{#1}%
}
\newcommand*{\colourboxedAux}[3]{%
  % #1: optional argument for color model
  % #2: color specification
  % #3: formula
  \begingroup
    \colorlet{cb@saved}{.}%
    \color#1{#2}%
    \boxed{%
      \color{cb@saved}%
      #3%
    }%
  \endgroup
}


%%%%%% COLOURS %%%%%%%%%%
\definecolor{airforceblue}{rgb}{0.36, 0.54, 0.66}
\definecolor{battleshipgrey}{rgb}{0.52, 0.52, 0.51}
\definecolor{brightmaroon}{rgb}{0.76, 0.13, 0.28}
\definecolor{nicegreen}{RGB}{133, 204, 111}

% isep colours
\definecolor{isepblue1}{RGB}{0, 97, 161}      % teinte à 100%
\definecolor{isepblue2}{RGB}{77, 144, 189}    % teinte à 70%
\definecolor{isepblue3}{RGB}{179, 208, 227}   % teinte à 30%
\definecolor{iseporange1}{RGB}{244, 161, 0}   % teinte à 100%
\definecolor{iseporange2}{RGB}{234, 189, 100} % teinte à 70%
\definecolor{iseporange3}{RGB}{252, 227, 179} % teinte à 30%
%%%%%%%%%%%%%%%%

% beamer stuff
\setbeamertemplate{navigation symbols}{}
\setbeamersize{text margin left=1.0cm,text margin right=1.0cm}
\setbeamercolor{background canvas}{bg=white}

\ifnum \SCHOOLVERSION = 2
	%%%% ISEP COLOURS %%%%
	\setbeamercolor{frametitle}{bg=isepblue1, fg=white}
	\setbeamercolor{progress bar}{fg=iseporange1}
	\setbeamercolor{itemize item}{fg=iseporange1,bg=iseporange1}
	%%%%%%%%%%%%%%%%%%%%%%
\fi

\setbeamerfont{frametitle}{family=\fontfamily{qag}\selectfont} % choose font for frame titles
\setbeamerfont{title}{family=\fontfamily{qag}\selectfont} % choose font for title
\setbeamerfont{subtitle}{family=\fontfamily{qag}\selectfont} % choose font for subtitle
\setbeamerfont{section title}{family=\fontfamily{qag}\selectfont} % choose font for titles
%\fontfamily{qag}\selectfont %choose font for main text % put after begin{document}

% to make the progress bar a little thicker
\makeatletter
\setlength{\metropolis@titleseparator@linewidth}{1.5pt}
\setlength{\metropolis@progressonsectionpage@linewidth}{1.5pt}
\setlength{\metropolis@progressinheadfoot@linewidth}{1.5pt}
\makeatother

\begin{document}

\title{Linear Algebra}
\subtitle{Matrix Inversion}
\author{Andrew Lehmann}
\ifnum \SCHOOLVERSION = 2
	\institute{\'{E}cole d'ing\'{e}nieurs du num\'{e}rique}
\fi
\date{\textit{Last updated: \today}}

% logo of university
\ifnum \SCHOOLVERSION = 2
	\titlegraphic{\includegraphics[width=3cm]{/home/andrew/Dropbox/ISEP/admin/logo-isep-2023.png} }
\fi


\begin{frame}
\titlepage
\end{frame}




%%%%%%%%%%%%%%%%%%%%%%%%%%%%%%%%%%%%%%%%%%%%%%%%%%%%%%%%%%%%%%%%%%%%%%%%
\begin{frame}
\frametitle{Reminder}

For a square matrix $M$, the inverse matrix $M^{-1}$ is defined by the relation:
\begin{align*}
M M^{-1} = M^{-1} M = I
\end{align*}
Where $I$ is the identity matrix of the same size as $M$. This inverse matrix is useful for solving systems of equations (with equal number of unknowns as equations). For example, for a system written in matrix form
\begin{align*}
AX = Y
\end{align*}
if the matrix $A$ has an inverse and we can compute it, then the unique solution to the
system is given by:
\begin{align*}
X = A^{-1}Y.
\end{align*}


\end{frame}
%%%%%%%%%%%%%%%%%%%%%%%%%%%%%%%%%%%%%%%%%%%%%%%%%%%%%%%%%%%%%%%%%%%%%%%%


\section{Cramer's rule and Cramer systems}


%%%%%%%%%%%%%%%%%%%%%%%%%%%%%%%%%%%%%%%%%%%%%%%%%%%%%%%%%%%%%%%%%%%%%%%%
\begin{frame}
\frametitle{Cramer's rule and Cramer systems}

Suppose we have the following linear system of equations (with unknowns equal to equations)
\begin{align*}
\begin{cases}
a_{11} x_1  + a_{12} x_2 + \cdots + a_{1n} x_n = y_1 \\
a_{21} x_1  + a_{22} x_2 + \cdots + a_{2n} x_n = y_2 \\
\vdots \\
a_{n1} x_1  + a_{n2} x_2 + \cdots + a_{nn} x_n = y_n
\end{cases}
\qquad (S)
\end{align*}
with the associated matrix form
\begin{align*}
\underbrace{
\begin{pmatrix}
a_{11} & a_{12} & \cdots & a_{1n} \\
a_{21} & a_{22} & \cdots & a_{2n} \\
\vdots & \vdots & \ddots & \vdots \\
a_{n1} & a_{n2} & \cdots & a_{nn}
\end{pmatrix}}_A
%%
%%
\underbrace{
\begin{pmatrix}
x_{1} \\
x_{2} \\
\vdots \\
x_{n}
\end{pmatrix}}_X
%%
=
\underbrace{
\begin{pmatrix}
y_{1} \\
y_{2} \\
\vdots \\
y_{n}
\end{pmatrix}}_Y
\end{align*}
\end{frame}
%%%%%%%%%%%%%%%%%%%%%%%%%%%%%%%%%%%%%%%%%%%%%%%%%%%%%%%%%%%%%%%%%%%%%%%%




%%%%%%%%%%%%%%%%%%%%%%%%%%%%%%%%%%%%%%%%%%%%%%%%%%%%%%%%%%%%%%%%%%%%%%%%
\begin{frame}
\frametitle{Theorem - Cramer's rule and Cramer systems}

The system $(S)$ admits a unique solution if and only if $\det(A) \neq 0$. In this case we say that we have a \textit{\textcolor{airforceblue}{Cramer system}}, and a solution can be found as follows. 

The coefficients of the solution vector $X$ are given by \textit{\textcolor{airforceblue}{Cramer's rule}}:
\begin{align*}
x_i = \dfrac{1}{\det(A)}
\underbrace{
\left|
\begin{matrix}
a_{11} & \cdots & a_{1,i-1} & y_1 & a_{1,i+1} & \cdots & a_{1n} \\
a_{21} & \cdots & a_{2,i-1} & y_2 & a_{2,i+1} & \cdots & a_{2n} \\
\vdots & \cdots & \vdots    & \vdots & \vdots & \ddots & \vdots \\
a_{n1} & \cdots & a_{n,i-1} & y_n & a_{n,i+1} & \cdots & a_{nn}
\end{matrix}
\right|
}_{\text{Replace the ith column with Y}}
\quad \text{for every $i$}
\end{align*}

Note that this means we must take $n$ different determinants of this style, one for each unknown in the $X$ column.

\end{frame}
%%%%%%%%%%%%%%%%%%%%%%%%%%%%%%%%%%%%%%%%%%%%%%%%%%%%%%%%%%%%%%%%%%%%%%%%





\ifnum \EXAMPLEVERSION = 1
%%%%%%%%%%%%%%%%%%%%%%%%%%%%%%%%%%%%%%%%%%%%%%%%%%%%%%%%%%%%%%%%%%%%%%%%
\begin{frame}
\frametitle{Examples - Cramer's rule}

Use Cramer’s rule to find the solutions to the systems $(S)$ and $(S’)$ with the following associated coefficient matrices and constant columns
\begin{align*}
A =& 
\begin{pmatrix}
3 & 1 & 5 \\
2 & 1 & 2 \\
0 & 3 & 7 
\end{pmatrix}
\quad
Y = 
\begin{pmatrix}
2 \\ 1 \\ 3
\end{pmatrix}
\qquad (S)
\\
A =& 
\begin{pmatrix}
3 & 1 & 5 \\
2 & 1 & 2 \\
0 & 3 & 7 
\end{pmatrix}
\quad
Y = 
\begin{pmatrix}
3 \\ 0 \\ -1
\end{pmatrix}
\qquad (S')
\end{align*}
\end{frame}
%%%%%%%%%%%%%%%%%%%%%%%%%%%%%%%%%%%%%%%%%%%%%%%%%%%%%%%%%%%%%%%%%%%%%%%%





%%%%%%%%%%%%%%%%%%%%%%%%%%%%%%%%%%%%%%%%%%%%%%%%%%%%%%%%%%%%%%%%%%%%%%%%
\begin{frame}
\fontsize{9pt}{10pt}\selectfont
\vspace{0.5cm}

Solution: $A = 
\begin{pmatrix}
3 & 1 & 5 \\
2 & 1 & 2 \\
0 & 3 & 7 
\end{pmatrix}
\quad
Y = 
\begin{pmatrix}
2 \\ 1 \\ 3
\end{pmatrix}
\qquad (S)$
\begin{align*}
\det(A) = 3\left| \begin{matrix} 1 & 2 \\ 3 & 7 \end{matrix} \right|
- 2\left| \begin{matrix} 1 & 5 \\ 3 & 7 \end{matrix} \right|
= 3(7-6) - 2(7-15) = 19
\end{align*}
\vspace{-0.5cm}
\begin{align*}
x_1 &= \frac{1}{\det(A)}
  \left|\begin{array}{>{\columncolor{airforceblue!20}}ccc}
	2 & 1 & 5 \\
	1 & 1 & 2 \\
	3 & 3 & 7 
  \end{array}\right|
  = \frac{1}{19}\left(
  	  2\left|\begin{matrix} 1 & 2 \\ 3 & 7 \end{matrix}\right| 
  	 -5\left|\begin{matrix} 1 & 2 \\ 3 & 7 \end{matrix}\right|
  	 +5\left|\begin{matrix} 1 & 1 \\ 3 & 3 \end{matrix}\right|
  	\right)
  = \frac{1}{19}
%%
\\
%%
x_2 &= \frac{1}{\det(A)}
  \left|\begin{array}{c>{\columncolor{airforceblue!20}}cc}
	3 & 2 & 5 \\
	2 & 1 & 2 \\
	0 & 3 & 7 
  \end{array}\right|
  = \frac{1}{19}\left(
  	  3\left|\begin{matrix} 1 & 2 \\ 3 & 7 \end{matrix}\right| 
  	 -2\left|\begin{matrix} 2 & 2 \\ 0 & 7 \end{matrix}\right|
  	 +5\left|\begin{matrix} 2 & 1 \\ 0 & 3 \end{matrix}\right|
  	\right)
  = \frac{5}{19}
%%
\\
%%
x_3 &= \frac{1}{\det(A)}
  \left|\begin{array}{cc>{\columncolor{airforceblue!20}}c}
	3 & 1 & 2 \\
	2 & 1 & 1 \\
	0 & 3 & 3 
  \end{array}\right|
  = \frac{1}{19}\left(
  	  3\left|\begin{matrix} 1 & 1 \\ 3 & 3 \end{matrix}\right| 
  	 -1\left|\begin{matrix} 2 & 1 \\ 0 & 3 \end{matrix}\right|
  	 +2\left|\begin{matrix} 2 & 1 \\ 0 & 3 \end{matrix}\right|
  	\right)
  = \frac{6}{19}
\end{align*}
So the unique solution to $(S)$ is $(x_1,x_2,x_3)=(1/19)(1,5,6)$
\end{frame}
%%%%%%%%%%%%%%%%%%%%%%%%%%%%%%%%%%%%%%%%%%%%%%%%%%%%%%%%%%%%%%%%%%%%%%%%
\fi 


\ifnum \EXAMPLEVERSION = 3
%%%%%%%%%%%%%%%%%%%%%%%%%%%%%%%%%%%%%%%%%%%%%%%%%%%%%%%%%%%%%%%%%%%%%%%%
\begin{frame}
\frametitle{Example - Cramer's rule}

\begin{minipage}{0.4\textwidth}
Use Cramer’s rule to solve
\begin{align*}
A =& 
\begin{pmatrix}
3 & 1 & 5 \\
2 & 1 & 2 \\
0 & 3 & 7 
\end{pmatrix}
\quad
Y = 
\begin{pmatrix}
2 \\ 1 \\ 3
\end{pmatrix}
\end{align*}
\end{minipage}
\vspace{4cm}
\end{frame}
%%%%%%%%%%%%%%%%%%%%%%%%%%%%%%%%%%%%%%%%%%%%%%%%%%%%%%%%%%%%%%%%%%%%%%%%

%%%%%%%%%%%%%%%%%%%%%%%%%%%%%%%%%%%%%%%%%%%%%%%%%%%%%%%%%%%%%%%%%%%%%%%%
\begin{frame}
\frametitle{Example - Cramer's rule}

\begin{minipage}{0.4\textwidth}
Use Cramer’s rule to solve
\begin{align*}
A =& 
\begin{pmatrix}
3 & 1 & 5 \\
2 & 1 & 2 \\
0 & 3 & 7 
\end{pmatrix}
\quad
Y = 
\begin{pmatrix}
3 \\ 0 \\ -1
\end{pmatrix}
\end{align*}
\end{minipage}
\vspace{4cm}
\end{frame}
%%%%%%%%%%%%%%%%%%%%%%%%%%%%%%%%%%%%%%%%%%%%%%%%%%%%%%%%%%%%%%%%%%%%%%%%
\fi 


\section{Cofactor method for finding inverses}

%%%%%%%%%%%%%%%%%%%%%%%%%%%%%%%%%%%%%%%%%%%%%%%%%%%%%%%%%%%%%%%%%%%%%%%%
\begin{frame}
\frametitle{Cofactor method for finding inverses}
For an invertible matrix $A$, it's inverse is given by
\begin{align*}
A^{-1} = \frac{1}{\det(A)}C_A^T
\end{align*}
where $C_A$ is called the cofactor matrix (or comatrix) of $A$:
\begin{align*}
C_A = \begin{pmatrix}
C_{11} & C_{12} & \cdots & C_{1n} \\
C_{21} & C_{22} & \cdots & C_{2n} \\
\vdots & \vdots & \ddots & \vdots \\
C_{n1} & C_{n2} & \cdots & C_{nn}
\end{pmatrix}
\end{align*}
Each coefficient of the comatrix, $C_{ij}$, is plus or minus the determinant of the submatrix of $A$ generated by removing the i$^{th}$ row and j$^{th}$ column:
\begin{align*}
C_{ij} = (-1)^{i+j}\det(A_{ij})
\end{align*}

\end{frame}
%%%%%%%%%%%%%%%%%%%%%%%%%%%%%%%%%%%%%%%%%%%%%%%%%%%%%%%%%%%%%%%%%%%%%%%%



\ifnum \EXAMPLEVERSION = 1
%%%%%%%%%%%%%%%%%%%%%%%%%%%%%%%%%%%%%%%%%%%%%%%%%%%%%%%%%%%%%%%%%%%%%%%%
\begin{frame}
\frametitle{Examples - Cofactor method}

Use the cofactor method to find the inverses of the following matrices:

\begin{align*}
A =& 
\begin{pmatrix}
  0 & 1 & -1 \\
  5 & 4 &  3 \\
  3 & 0 & -1
\end{pmatrix}
\\ \\
B =& 
\begin{pmatrix}
  1 & 2 &  0 \\
  5 & 2 & -3 \\
  1 & 0 & -1
\end{pmatrix}
\end{align*}

\end{frame}
%%%%%%%%%%%%%%%%%%%%%%%%%%%%%%%%%%%%%%%%%%%%%%%%%%%%%%%%%%%%%%%%%%%%%%%%



%%%%%%%%%%%%%%%%%%%%%%%%%%%%%%%%%%%%%%%%%%%%%%%%%%%%%%%%%%%%%%%%%%%%%%%%
\begin{frame}

Solution: 
$A = 
\begin{pmatrix}
  0 & 1 & -1 \\
  5 & 4 &  3 \\
  3 & 0 & -1
\end{pmatrix}$.
We need the determinant $\det(A)=26$.

There are nine elements to the comatrix:
\begin{align*}
C_{11} &= (-1)^{1+1}\left|\begin{matrix} 4 & 3 \\ 0 & -1 \end{matrix}\right|=-4, & 
C_{12} &= (-1)^{1+2}\left|\begin{matrix} 5 & 3 \\ 3 & -1 \end{matrix}\right|=14, & 
C_{13} &= (-1)^{1+3}\left|\begin{matrix} 5 & 4 \\ 3 & 0 \end{matrix}\right|=-12 \\
C_{21} &= (-1)^{2+1}\left|\begin{matrix} 1 & -1 \\ 0 & -1 \end{matrix}\right|=1,
&
C_{22} &= (-1)^{2+2}\left|\begin{matrix} 0 & -1 \\ 3 & -1 \end{matrix}\right|=3, &
C_{23} &= (-1)^{2+3}\left|\begin{matrix} 0 & 1 \\ 3 & 0 \end{matrix}\right|=3
\\
C_{31} &= (-1)^{3+1}\left|\begin{matrix} 1 & -1 \\ 4 & 3 \end{matrix}\right|=7,
&
C_{32} &= (-1)^{3+2}\left|\begin{matrix} 0 & -1 \\ 5 & 3 \end{matrix}\right|=-5, 
&
C_{33} &= (-1)^{3+3}\left|\begin{matrix} 0 & 1 \\ 5 & 4 \end{matrix}\right|=-5
\end{align*}

\end{frame}
%%%%%%%%%%%%%%%%%%%%%%%%%%%%%%%%%%%%%%%%%%%%%%%%%%%%%%%%%%%%%%%%%%%%%%%%

%%%%%%%%%%%%%%%%%%%%%%%%%%%%%%%%%%%%%%%%%%%%%%%%%%%%%%%%%%%%%%%%%%%%%%%%
\begin{frame}
So we have the comatrix
\begin{align*}
C_A = 
\begin{pmatrix}
 -4 & 14 & -12 \\
  1 &  3 &   3 \\
  7 & -5 &  -5
\end{pmatrix}
\end{align*}
Transpose it, divide by the determinant of $A$ and we have the inverse of $A$
\begin{align*}
\begin{pmatrix}
  0 & 1 & -1 \\
  5 & 4 &  3 \\
  3 & 0 & -1
\end{pmatrix}^{-1}
=
\frac{1}{26}
\begin{pmatrix}
  -4 &  1 &  7 \\
  14 &  3 & -5 \\
 -12 &  3 & -5
\end{pmatrix}
\end{align*}
You should verify that $AA^{-1}=A^{-1}A=I$
\end{frame}
%%%%%%%%%%%%%%%%%%%%%%%%%%%%%%%%%%%%%%%%%%%%%%%%%%%%%%%%%%%%%%%%%%%%%%%%




%%%%%%%%%%%%%%%%%%%%%%%%%%%%%%%%%%%%%%%%%%%%%%%%%%%%%%%%%%%%%%%%%%%%%%%%
\begin{frame}
\frametitle{Solutions}

\begin{align*}
A&=
\begin{pmatrix}
  0 & 1 & -1 \\
  5 & 4 &  3 \\
  3 & 0 & -1
\end{pmatrix}
\quad\implies\quad
A^{-1}
=
\frac{1}{26}
\begin{pmatrix}
  -4 &  1 &  7 \\
  14 &  3 & -5 \\
 -12 &  3 & -5
\end{pmatrix}
%%
\\ \\
%%
B&=
\begin{pmatrix}
  1 & 2 &  0 \\
  5 & 2 & -3 \\
  1 & 0 & -1
\end{pmatrix}
\quad\implies\quad
B^{-1}
=
\frac{1}{2}
\begin{pmatrix}
  -2 &  2 & -6 \\
   2 & -1 &  3 \\
  -2 &  2 & -8
\end{pmatrix}
\end{align*}
\end{frame}
%%%%%%%%%%%%%%%%%%%%%%%%%%%%%%%%%%%%%%%%%%%%%%%%%%%%%%%%%%%%%%%%%%%%%%%%
\fi


\ifnum \EXAMPLEVERSION = 3
%%%%%%%%%%%%%%%%%%%%%%%%%%%%%%%%%%%%%%%%%%%%%%%%%%%%%%%%%%%%%%%%%%%%%%%%
\begin{frame}
\frametitle{Examples - Cofactor method}

Use the cofactor method to find the inverse of: $A = \begin{pmatrix}
  0 & 1 & -1 \\
  5 & 4 &  3 \\
  3 & 0 & -1
\end{pmatrix}$
\vspace{5cm}
\end{frame}
%%%%%%%%%%%%%%%%%%%%%%%%%%%%%%%%%%%%%%%%%%%%%%%%%%%%%%%%%%%%%%%%%%%%%%%%

%%%%%%%%%%%%%%%%%%%%%%%%%%%%%%%%%%%%%%%%%%%%%%%%%%%%%%%%%%%%%%%%%%%%%%%%
\begin{frame}
\end{frame}
%%%%%%%%%%%%%%%%%%%%%%%%%%%%%%%%%%%%%%%%%%%%%%%%%%%%%%%%%%%%%%%%%%%%%%%%


%%%%%%%%%%%%%%%%%%%%%%%%%%%%%%%%%%%%%%%%%%%%%%%%%%%%%%%%%%%%%%%%%%%%%%%%
\begin{frame}
\frametitle{Examples - Cofactor method}

Use the cofactor method to find the inverse of: $B = \begin{pmatrix}
  1 & 2 &  0 \\
  5 & 2 & -3 \\
  1 & 0 & -1
\end{pmatrix}$
\vspace{5cm}
\end{frame}
%%%%%%%%%%%%%%%%%%%%%%%%%%%%%%%%%%%%%%%%%%%%%%%%%%%%%%%%%%%%%%%%%%%%%%%%

%%%%%%%%%%%%%%%%%%%%%%%%%%%%%%%%%%%%%%%%%%%%%%%%%%%%%%%%%%%%%%%%%%%%%%%%
\begin{frame}
\end{frame}
%%%%%%%%%%%%%%%%%%%%%%%%%%%%%%%%%%%%%%%%%%%%%%%%%%%%%%%%%%%%%%%%%%%%%%%%
\fi 



%%%%%%%%%%%%%%%%%%%%%%%%%%%%%%%%%%%%%%%%%%%%%%%%%%%%%%%%%%%%%%%%%%%%%%%%
\begin{frame}

\frametitle{Theorem - Inverse of a $2\times 2$ matrix}

With the cofactor method, the inverse of a $2\times 2$ matrix is given by
\begin{align*}
\begin{pmatrix}
a & b \\
c & d
\end{pmatrix}^{-1}
=
\frac{1}{ad - bc}
\begin{pmatrix}
d & -c \\
-b & a
\end{pmatrix}
\end{align*}
\end{frame}
%%%%%%%%%%%%%%%%%%%%%%%%%%%%%%%%%%%%%%%%%%%%%%%%%%%%%%%%%%%%%%%%%%%%%%%%



\section{Gauss' method for finding inverses}



%%%%%%%%%%%%%%%%%%%%%%%%%%%%%%%%%%%%%%%%%%%%%%%%%%%%%%%%%%%%%%%%%%%%%%%%
\begin{frame}
Suppose we have a linear system of equations $(S)$ with associated matrices
\begin{align*}
A=
\begin{pmatrix}
   1 &  -2 &  2 \\
   8 & -15 & 15 \\
   2 &  -2 &  3
\end{pmatrix}, \qquad
%%
%%
X = 
\begin{pmatrix}
x \\
y \\
z
\end{pmatrix}, \qquad
%%
Y
=
\begin{pmatrix}
a \\
b \\
c
\end{pmatrix}
\end{align*}
Recall: a matrix is invertible if it's associated system can be solved for arbitrary constant vector $Y$. So, we solve this system as usual with row reduction.
\begin{align*}
&
\left(
	\begin{matrix}
   1 &  -2 &  2 \\
   8 & -15 & 15 \\
   2 &  -2 &  3
	\end{matrix}
  \left|
	\begin{matrix}
		a \\
		b \\
		c
	\end{matrix}
  \right.
\right)
\begin{array}{l}
   \\
 R_2 - 8R_1 \\
 R_3 - 2R_1
\end{array}
%%
%%
\\
%%
%%
&
\left(
	\begin{matrix}
   1 &  -2 &  2 \\
   0 &   1 & -1 \\
   0 &   2 & -1
	\end{matrix}
  \left|
	\begin{matrix}
		a \\
		b-8a \\
		c-2a
	\end{matrix}
  \right.
\right)
\begin{array}{l}
   \\
   \\
 R_3 - 2R_2
\end{array}
%%
%%
\to
%%
%%
\left(
	\begin{matrix}
   1 &  -2 &  2 \\
   0 &   1 & -1 \\
   0 &   0 &  1
	\end{matrix}
  \left|
	\begin{matrix}
		a \\
		b-8a \\
		c-2a-2b+16a
	\end{matrix}
  \right.
\right)
\end{align*}
\end{frame}
%%%%%%%%%%%%%%%%%%%%%%%%%%%%%%%%%%%%%%%%%%%%%%%%%%%%%%%%%%%%%%%%%%%%%%%%



%%%%%%%%%%%%%%%%%%%%%%%%%%%%%%%%%%%%%%%%%%%%%%%%%%%%%%%%%%%%%%%%%%%%%%%%
\begin{frame}
\vspace{-0.6cm}
\begin{align*}
\left(
	\begin{matrix}
   1 &  -2 &  2 \\
   0 &   1 & -1 \\
   0 &   0 &  1
	\end{matrix}
  \left|
	\begin{matrix}
		a \\
		-8a + b \\
		14a -2b + c
	\end{matrix}
  \right.
\right)
\end{align*}
At this step there are 2 methods to get to the inverse.

\textbf{Method 1) Classical method}. We return to the system of equations and rearrange to solve for $x$, $y$ and $z$:
\vspace{-0.3cm}
\begin{align*}
(S) 
\begin{cases}
z = 14a -2b + c\\
y = b - 8a + z = 6a - b + c\\
x = a + 2y - 2z = -15a + 2b
\end{cases}
\end{align*}
The inverse matrix is made up of the coefficients of $a$, $b$ and $c$:
\begin{align*}
\begin{pmatrix}
x \\
y \\
z
\end{pmatrix}
%%
=
\underbrace{
\begin{pmatrix}
 -15 &  2 & 0 \\
   6 & -1 & 1 \\
  14 &  2 & 1
\end{pmatrix}}_{A^{-1}}
%%
%%
\begin{pmatrix}
a \\
b \\
c
\end{pmatrix}
\end{align*}
\end{frame}
%%%%%%%%%%%%%%%%%%%%%%%%%%%%%%%%%%%%%%%%%%%%%%%%%%%%%%%%%%%%%%%%%%%%%%%%


%%%%%%%%%%%%%%%%%%%%%%%%%%%%%%%%%%%%%%%%%%%%%%%%%%%%%%%%%%%%%%%%%%%%%%%%
\begin{frame}\vspace{-0.6cm}
\begin{align*}
\left(
	\begin{matrix}
   1 &  -2 &  2 \\
   0 &   1 & -1 \\
   0 &   0 &  1
	\end{matrix}
  \left|
	\begin{matrix}
		a \\
		-8a + b \\
		14a -2b + c
	\end{matrix}
  \right.
\right)
\end{align*}
\textbf{Method 2) Gauss-Jordan method}. We continue row reduction, moving upwards to zero the columns above the pivots.
\begin{align*}
&
\begin{array}{l}
 R_1 - 2R_3  \\
 R_2 +  R_3 \\
 \\
\end{array}
\left(
	\begin{matrix}
   1 &  -2 &  0 \\
   0 &   1 &  0 \\
   0 &   0 &  1
	\end{matrix}
  \left|
	\begin{matrix}
		-27a + 4b -2c \\
		6a - b + c \\
		14a -2b + c
	\end{matrix}
  \right.
\right)
%%
%%
\\
%%
%%
&
\begin{array}{l}
 R_1 + 2R_2  \\
 \\
 \\
\end{array}
\left(
	\begin{matrix}
   1 &  0 &  0 \\
   0 &  1 &  0 \\
   0 &  0 &  1
	\end{matrix}
  \left|
	\begin{matrix}
		-15a + 2b  \\
		6a - b + c \\
		14a -2b + c
	\end{matrix}
  \right.
\right)
\end{align*}
Unpacking we have:
$(S) 
\begin{cases}
x = -15a + 2b \\
y = 6a - b + c \\
z = 14a -2b + c
\end{cases}
\implies
A^{-1} =
\begin{pmatrix}
	-15 &  2 &  0 \\
	  6 & -1 &  1 \\
	 14 & -2 &  1
\end{pmatrix}$
\end{frame}
%%%%%%%%%%%%%%%%%%%%%%%%%%%%%%%%%%%%%%%%%%%%%%%%%%%%%%%%%%%%%%%%%%%%%%%%


%%%%%%%%%%%%%%%%%%%%%%%%%%%%%%%%%%%%%%%%%%%%%%%%%%%%%%%%%%%%%%%%%%%%%%%%
\begin{frame}
\frametitle{Magic table method}
This method is merely a rewriting of the Gauss-Jordan method. It abstracts away all but the most important information. We create an augmented matrix with the identity on the right. We then use elementary row operations to make the left side the identity.

\begin{minipage}{0.48\textwidth}
\begin{align*}
&
\left(
	\begin{matrix}
		1 &  -2 &  2 \\
		8 & -15 & 15 \\
		2 &  -2 &  3
	\end{matrix}
  \left|
	\begin{matrix}
		1 &  0 &  0 \\
		0 &  1 &  0 \\
		0 &  0 &  1
	\end{matrix}
  \right.
\right)
\begin{array}{l}
   \\
 R_2 - 8R_1 \\
 R_3 - 2R_1
\end{array}
%%
%%
\\
%%
%%
&
\left(
	\begin{matrix}
		1 &  -2 &  2 \\
		0 &   1 & -1 \\
		0 &   2 & -1
	\end{matrix}
  \left|
	\begin{matrix}
		 1 &  0 &  0 \\
		-8 &  1 &  0 \\
		-2 &  0 &  1
	\end{matrix}
  \right.
\right)
\begin{array}{l}
   \\
   \\
 R_3 - 2R_1
\end{array}
%%
%%
\\
%%
%%
&
\left(
	\begin{matrix}
		1 &  -2 &  2 \\
		0 &   1 & -1 \\
		0 &   0 &  1
	\end{matrix}
  \left|
	\begin{matrix}
		 1 &  0 &  0 \\
		-8 &  1 &  0 \\
		14 & -2 &  1
	\end{matrix}
  \right.
\right)
\begin{array}{l}
 R_1 - 2R_3  \\
 R_2 +  R_3 \\
 \\
\end{array}
\end{align*}
\end{minipage}\hfill
\begin{minipage}{0.48\textwidth}
\begin{align*}
&
\left(
	\begin{matrix}
		1 &  -2 &  0 \\
		0 &   1 &  0 \\
		0 &   0 &  1
	\end{matrix}
  \left|
	\begin{matrix}
		-27 &  4 & -2 \\
		  6 & -1 &  1 \\
		 14 & -2 &  1
	\end{matrix}
  \right.
\right)
\begin{array}{l}
 R_1 + 2R_2  \\
   \\
  \\
\end{array}
%%
%%
\\
%%
%%
&
\left(
	\begin{matrix}
		1 &   0 &  0 \\
		0 &   1 &  0 \\
		0 &   0 &  1
	\end{matrix}
  \left|
	\begin{matrix}
		-15 &  2 &  0 \\
		  6 & -1 &  1 \\
		 14 & -2 &  1
	\end{matrix}
  \right.
\right)
%%
%%
\\
%%
%%
& \implies
A^{-1} =
\begin{pmatrix}
	-15 &  2 &  0 \\
	  6 & -1 &  1 \\
	 14 & -2 &  1
\end{pmatrix}
\end{align*}
\end{minipage}

\end{frame}
%%%%%%%%%%%%%%%%%%%%%%%%%%%%%%%%%%%%%%%%%%%%%%%%%%%%%%%%%%%%%%%%%%%%%%%%







\ifnum \EXAMPLEVERSION = 1
%%%%%%%%%%%%%%%%%%%%%%%%%%%%%%%%%%%%%%%%%%%%%%%%%%%%%%%%%%%%%%%%%%%%%%%%
\begin{frame}
\frametitle{Examples - Magic table method}

Use the magic table method to find the inverses of the following matrices:

\begin{align*}
A =& 
\begin{pmatrix}
  1 &  2 & -3 \\
  2 &  5 & -7 \\
 -1 & -2 &  4
\end{pmatrix}
\\ \\
B =& 
\begin{pmatrix}
  1 & -3 & -3 \\
  0 &  1 &  1 \\
 -1 &  3 &  4
\end{pmatrix}
\end{align*}

\end{frame}
%%%%%%%%%%%%%%%%%%%%%%%%%%%%%%%%%%%%%%%%%%%%%%%%%%%%%%%%%%%%%%%%%%%%%%%%



%%%%%%%%%%%%%%%%%%%%%%%%%%%%%%%%%%%%%%%%%%%%%%%%%%%%%%%%%%%%%%%%%%%%%%%%
\begin{frame}
\frametitle{Worked solution for $A$}

\begin{minipage}{0.48\textwidth}
\begin{align*}
&
\left(
	\begin{matrix}
	  1 &  2 & -3 \\
	  2 &  5 & -7 \\
	 -1 & -2 &  4
	\end{matrix}
  \left|
	\begin{matrix}
		1 &  0 &  0 \\
		0 &  1 &  0 \\
		0 &  0 &  1
	\end{matrix}
  \right.
\right)
\begin{array}{l}
   \\
 R_2 - 2R_1 \\
 R_3 + 1R_1
\end{array}
%%
%%
\\
%%
%%
&
\left(
	\begin{matrix}
	  1 &  2 & -3 \\
	  0 &  1 & -1 \\
	  0 &  0 &  1
	\end{matrix}
  \left|
	\begin{matrix}
		 1 &  0 &  0 \\
		-2 &  1 &  0 \\
		 1 &  0 &  1
	\end{matrix}
  \right.
\right)
\begin{array}{l}
  R_1 + 3R_3 \\
  R_2 +  R_3 \\
   \\
\end{array}
\end{align*}
\end{minipage}\hfill
\begin{minipage}{0.48\textwidth}
\begin{align*}
&
\left(
	\begin{matrix}
	  1 &  2 &  0 \\
	  0 &  1 &  0 \\
	  0 &  0 &  1
	\end{matrix}
  \left|
	\begin{matrix}
		 4 &  0 &  3 \\
		-1 &  1 &  1 \\
		 1 &  0 &  1
	\end{matrix}
  \right.
\right)
\begin{array}{l}
 R_1 - 2R_2  \\
   \\
  \\
\end{array}
%%
%%
\\
%%
%%
&
\left(
	\begin{matrix}
	  1 &  0 &  0 \\
	  0 &  1 &  0 \\
	  0 &  0 &  1
	\end{matrix}
  \left|
	\begin{matrix}
		 6 & -2 &  1 \\
		-1 &  1 &  1 \\
		 1 &  0 &  1
	\end{matrix}
  \right.
\right)
\end{align*}
\end{minipage}

And so we have the inverse: $
A^{-1} =
\begin{pmatrix}
	 6 & -2 &  1 \\
	-1 &  1 &  1 \\
	 1 &  0 &  1
\end{pmatrix}
$
\end{frame}
%%%%%%%%%%%%%%%%%%%%%%%%%%%%%%%%%%%%%%%%%%%%%%%%%%%%%%%%%%%%%%%%%%%%%%%%






%%%%%%%%%%%%%%%%%%%%%%%%%%%%%%%%%%%%%%%%%%%%%%%%%%%%%%%%%%%%%%%%%%%%%%%%
\begin{frame}
\frametitle{Solutions}

\begin{align*}
A =& 
\begin{pmatrix}
  1 &  2 & -3 \\
  2 &  5 & -7 \\
 -1 & -2 &  4
\end{pmatrix}
\qquad
A^{-1} =
\begin{pmatrix}
	 6 & -2 &  1 \\
	-1 &  1 &  1 \\
	 1 &  0 &  1
\end{pmatrix}
\\ \\
B =& 
\begin{pmatrix}
  1 & -3 & -3 \\
  0 &  1 &  1 \\
 -1 &  3 &  4
\end{pmatrix}
\qquad
B^{-1} =
\begin{pmatrix}
	 1 &  3 &  0 \\
	-1 &  1 & -1 \\
	 1 &  0 &  1
\end{pmatrix}
\end{align*}

\end{frame}
%%%%%%%%%%%%%%%%%%%%%%%%%%%%%%%%%%%%%%%%%%%%%%%%%%%%%%%%%%%%%%%%%%%%%%%%
\fi 


\ifnum \EXAMPLEVERSION = 3
%%%%%%%%%%%%%%%%%%%%%%%%%%%%%%%%%%%%%%%%%%%%%%%%%%%%%%%%%%%%%%%%%%%%%%%%
\begin{frame}
\frametitle{Example - Magic table method}
Find the inverse of: $A = 
\begin{pmatrix}
  1 &  2 & -3 \\
  2 &  5 & -7 \\
 -1 & -2 &  4
\end{pmatrix}$
\vspace{5cm}
\end{frame}
%%%%%%%%%%%%%%%%%%%%%%%%%%%%%%%%%%%%%%%%%%%%%%%%%%%%%%%%%%%%%%%%%%%%%%%%

%%%%%%%%%%%%%%%%%%%%%%%%%%%%%%%%%%%%%%%%%%%%%%%%%%%%%%%%%%%%%%%%%%%%%%%%
\begin{frame}
\end{frame}
%%%%%%%%%%%%%%%%%%%%%%%%%%%%%%%%%%%%%%%%%%%%%%%%%%%%%%%%%%%%%%%%%%%%%%%%


%%%%%%%%%%%%%%%%%%%%%%%%%%%%%%%%%%%%%%%%%%%%%%%%%%%%%%%%%%%%%%%%%%%%%%%%
\begin{frame}
\frametitle{Example - Magic table method}
Find the inverse of: $B = 
\begin{pmatrix}
  1 & -3 & -3 \\
  0 &  1 &  1 \\
 -1 &  3 &  4
\end{pmatrix}$
\vspace{5cm}
\end{frame}
%%%%%%%%%%%%%%%%%%%%%%%%%%%%%%%%%%%%%%%%%%%%%%%%%%%%%%%%%%%%%%%%%%%%%%%%

%%%%%%%%%%%%%%%%%%%%%%%%%%%%%%%%%%%%%%%%%%%%%%%%%%%%%%%%%%%%%%%%%%%%%%%%
\begin{frame}
\end{frame}
%%%%%%%%%%%%%%%%%%%%%%%%%%%%%%%%%%%%%%%%%%%%%%%%%%%%%%%%%%%%%%%%%%%%%%%%
\fi 



\section{Matrix rank}



%%%%%%%%%%%%%%%%%%%%%%%%%%%%%%%%%%%%%%%%%%%%%%%%%%%%%%%%%%%%%%%%%%%%%%%%
\begin{frame}
\frametitle{Row echelon form}

Row echelon form is the resulting upper triangle matrix at the end of a Gaussian reduction process: 

\begin{align*}
\begin{pmatrix}
a_{11} & a_{12} & \cdots & a_{1m} \\
a_{21} & a_{22} & \cdots & a_{2m} \\
\vdots & \vdots & \ddots & \vdots \\
a_{n1} & a_{n2} & \cdots & a_{nm}
\end{pmatrix}
\xrightarrow{\text{Gaussian reduction}}
\begin{pmatrix}
a_{11} & a_{12}  & \cdots & a_{1n}  & \cdots & a_{1m}  \\
0      & a'_{22} & \cdots & a'_{2n} & \cdots & a'_{2m} \\
\vdots & \vdots  & \ddots & \vdots  & \cdots & \vdots  \\
0      & 0       & \cdots & a'_{nn} & \cdots & a'_{nm} 
\end{pmatrix}
\end{align*}


\end{frame}
%%%%%%%%%%%%%%%%%%%%%%%%%%%%%%%%%%%%%%%%%%%%%%%%%%%%%%%%%%%%%%%%%%%%%%%%



%%%%%%%%%%%%%%%%%%%%%%%%%%%%%%%%%%%%%%%%%%%%%%%%%%%%%%%%%%%%%%%%%%%%%%%%
\begin{frame}
\frametitle{Matrix rank}

The rank of a matrix can be read from the row echelon Ifform. The rank of a matrix is the number of pivots in the row echelon form of that matrix. For example
\begin{align*}
\text{rank}\begin{pmatrix}
1 & 2 & 3 \\
0 & 0 & 2 \\
0 & 0 & 1
\end{pmatrix}
=2,
%%%
\quad
\text{rank}\begin{pmatrix}
1 & 2 & 3 & 2\\
0 & 1 & 2 & 2 \\
0 & 0 & 1 & 2
\end{pmatrix}
=3,
%%%
\quad
\text{rank}\begin{pmatrix}
2 & 3 \\
4 & 6
\end{pmatrix}
=1
\end{align*}

For a linear system $AX=Y$, if the rank of the matrix is equal to the number of unknown variables (size of $X$), then there will be 1 unique solution. Otherwise, the rank is less than the number of unknowns (impossible to be more) and there could be infinite or zero solutions, depending on the right side of the system, $Y$.
\end{frame}
%%%%%%%%%%%%%%%%%%%%%%%%%%%%%%%%%%%%%%%%%%%%%%%%%%%%%%%%%%%%%%%%%%%%%%%%


\end{document}



%%%%%%%%%%%%%%%%%%%%%%%%%%%%%%%%%%%%%%%%%%%%%%%%%%%%%%%%%%%%%%%%%%%%%%%%
\begin{frame}
\frametitle{Definition - }
\end{frame}
%%%%%%%%%%%%%%%%%%%%%%%%%%%%%%%%%%%%%%%%%%%%%%%%%%%%%%%%%%%%%%%%%%%%%%%%



%%%%%%%%%%%%%%%%%%%%%%%%%%%%%%%%%%%%%%%%%%%%%%%%%%%%%%%%%%%%%%%%%%%%%%%%
\begin{frame}
\frametitle{Properties - }
\end{frame}
%%%%%%%%%%%%%%%%%%%%%%%%%%%%%%%%%%%%%%%%%%%%%%%%%%%%%%%%%%%%%%%%%%%%%%%%



%%%%%%%%%%%%%%%%%%%%%%%%%%%%%%%%%%%%%%%%%%%%%%%%%%%%%%%%%%%%%%%%%%%%%%%%
\begin{frame}
\frametitle{Theorem - }
\end{frame}
%%%%%%%%%%%%%%%%%%%%%%%%%%%%%%%%%%%%%%%%%%%%%%%%%%%%%%%%%%%%%%%%%%%%%%%%



%%%%%%%%%%%%%%%%%%%%%%%%%%%%%%%%%%%%%%%%%%%%%%%%%%%%%%%%%%%%%%%%%%%%%%%%
\begin{frame}
\frametitle{Example - }
\end{frame}
%%%%%%%%%%%%%%%%%%%%%%%%%%%%%%%%%%%%%%%%%%%%%%%%%%%%%%%%%%%%%%%%%%%%%%%%

%%%%%%%%%%%%%%%%%%%%%%%%%%%%%%%%%%%%%%%%%%%%%%%%%%%%%%%%%%%%%%%%%%%%%%%%
\begin{frame}
\frametitle{title}
\fontsize{9pt}{10pt}\selectfont
\end{frame}
%%%%%%%%%%%%%%%%%%%%%%%%%%%%%%%%%%%%%%%%%%%%%%%%%%%%%%%%%%%%%%%%%%%%%%%%


%%%% COLOUR CHOICES
% \textcolor{MidnightBlue}{}
% \textcolor{Maroon}{}
% \textcolor{Purple}{matrix}
% \textcolor{BurntOrange}{}
% \textcolor{MidnightBlue}{}
% \textcolor{Mahogany}{}
% \textcolor{ForestGreen}{}
